%% LyX 2.0.5.1 created this file.  For more info, see http://www.lyx.org/.
%% Do not edit unless you really know what you are doing.
\documentclass[11pt,english,ngerman]{scrreprt}
\usepackage[T1]{fontenc}
\usepackage[utf8]{inputenc}
\usepackage[a4paper]{geometry}
\geometry{verbose,tmargin=2.6cm,bmargin=3.5cm,lmargin=2.6cm,rmargin=2.6cm}
\usepackage{fancyhdr}
\pagestyle{fancy}
\setlength{\parskip}{\smallskipamount}
\setlength{\parindent}{0pt}
\usepackage{color}
\usepackage{babel}
\usepackage{array}
\usepackage{textcomp}
\usepackage{url}
\usepackage{multirow}
\usepackage{amsmath}
\usepackage{amssymb}
\usepackage{graphicx}
\usepackage{esint}
\usepackage[unicode=true,
 bookmarks=true,bookmarksnumbered=true,bookmarksopen=false,
 breaklinks=true,pdfborder={0 0 0},backref=page,colorlinks=false]
 {hyperref}
\hypersetup{pdftitle={Thermodynamik und Festkörperphysik},
 pdfauthor={Andreas Völklein},
 pdfkeywords={Thermodynamik, Festkörperphysik, Physik}}

\makeatletter

%%%%%%%%%%%%%%%%%%%%%%%%%%%%%% LyX specific LaTeX commands.
\providecommand{\LyX}{\texorpdfstring%
  {L\kern-.1667em\lower.25em\hbox{Y}\kern-.125emX\@}
  {LyX}}
\newcommand{\noun}[1]{\textsc{#1}}
%% Because html converters don't know tabularnewline
\providecommand{\tabularnewline}{\\}

%%%%%%%%%%%%%%%%%%%%%%%%%%%%%% Textclass specific LaTeX commands.
\usepackage{enumitem}		% customizable list environments
\newlength{\lyxlabelwidth}      % auxiliary length 

\@ifundefined{date}{}{\date{}}
%%%%%%%%%%%%%%%%%%%%%%%%%%%%%% User specified LaTeX commands.
\usepackage{tikz,pgfplots}
%\usepackage{tikz-3dplot,cancel,polynom}
\usetikzlibrary{matrix,arrows,calc,decorations,intersections}
\usepackage{latexsym,stmaryrd,stackrel,braket,bbm,subfig,framed,esvect,scrhack,calc}
\usepackage [OMLmathrm,OMLmathbf,sfdefault=fav]{isomath}
\usepackage[explicit]{titlesec}
\usepackage[activate]{pdfcprot}

\pgfkeys{/pgf/number format/dec sep={\text{,}}}

% Inhaltsverzeichnis
\usepackage[subfigure]{tocloft}

\tocloftpagestyle{fancy}

\renewcommand{\cftchapindent}{1 em}
\renewcommand{\cftchapnumwidth}{1.5 em}

\renewcommand{\cftsecindent}{2.7 em}
\renewcommand{\cftsecnumwidth}{2.5em}

\renewcommand{\cftsubsecindent}{5.2 em}
\renewcommand{\cftsubsecnumwidth}{3.8 em}

\renewcommand{\cftsubsubsecindent}{9 em}
\renewcommand{\cftsubsubsecnumwidth}{4.5 em}

% Mathe-Operatoren
\DeclareMathOperator*{\exsop}{\exists}
\DeclareMathOperator*{\exsgop}{\exists!}
\DeclareMathOperator*{\fallop}{\forall}
\DeclareMathOperator*{\bcupdop}{\dot{\bigcup}}
\DeclareMathOperator*{\bcapdop}{\dot{\bigcap}}

%Operatornorm
\newcommand{\opnor}[1]{\abs{\hspace*{-1.1pt}\norm{#1}\hspace*{-1.1pt}}}

% nicht-totales Differential
\newcommand{\dBar}{\mathchar'26\mkern-12mu \textnormal{d}}

% Angström
\newcommand{\ang}{\textup{\AA}}

% schöne Vektorpfeile
\renewcommand{\vec}[1]{\vv{#1}}

% Rotieren
\newcommand{\Rotate}[1]{
\begin{tikzpicture}
\node[rotate=90] {\ensuremath{#1}};
\end{tikzpicture}
}

%QED-Zeichen (Box)
\newcommand{\qed}{\ensuremath{\Box}}
\newcommand{\qqed}[1][\arabic{chapter}.\arabic{section}\ifnum\arabic{subsection}>0{.\arabic{subsection}}\fi]{\hfill\qed\ensuremath{_{\text{#1}}}}

% Mengen Modulo
\newcommand{\moduloT}[2]{
\mbox{\raisebox{0.6ex}{\ensuremath{\displaystyle #1}}
{\hspace*{-1.5mm}\Large /}
\raisebox{-0.6ex}{\hspace*{-1.5mm}\ensuremath{\displaystyle #2}}
}}

% Links Modulo
\newcommand{\lmoduloT}[2]{
\mbox{\raisebox{-0.6ex}{\ensuremath{\displaystyle #1}}
{\hspace*{-1.5mm}\Large \ensuremath{\backslash}}
\raisebox{0.6ex}{\hspace*{-1.5mm}\ensuremath{\displaystyle #2}}
}}

% Für Z/2Z, um nicht soviel schreiben zu müssen
\newcommand{\modloT}[2]{\moduloT{ \mathbb{#1}}{#2\mathbb{#1}}}

%Die Modulo-Kommandos in klein, für die Darstellungen unter Quantoren.
\newcommand{\moduloScriptT}[2]{
\mbox{\raisebox{0.4ex}{\scriptsize\ensuremath{\displaystyle #1}}
{\hspace*{-1.5mm}\footnotesize /}
\raisebox{-0.4ex}{\hspace*{-1.5mm}\scriptsize\ensuremath{\displaystyle #2}}
}}

\newcommand{\lmoduloScriptT}[2]{
\mbox{\raisebox{-0.4ex}{\scriptsize\ensuremath{\displaystyle #1}}
{\hspace*{-1.5mm}\footnotesize \ensuremath{\backslash}}
\raisebox{0.4ex}{\hspace*{-1.5mm}\scriptsize\ensuremath{\displaystyle #2}}
}}

\newcommand{\modloScriptT}[2]{\moduloScriptT{ \mathbb{#1}}{#2\mathbb{#1}}}

% Wurzel mit Häkchen
\newcommand{\hsqrt}[2][{}]{\setbox0=\hbox{$\sqrt[#1]{\phantom{|}\!\! #2\hspace*{1pt}}$}\dimen0=\ht0
  \advance\dimen0-0.2\ht0
  \setbox2=\hbox{\vrule height\ht0 depth -\dimen0}
  {\box0\lower0.4pt\box2}}

% Damit nicht immer "Kapitel 1" etc. über der Kapitelüberschrift steht
\titleformat{\chapter}
  {\huge\bfseries}
  {\textrm{\thechapter} }{0pt}
  {\textrm{#1} \thispagestyle{fancy}
  }

% Neudefinition der Abschnittsmarker für die Kopfzeile
\renewcommand\partmark[1]{\markboth{#1}{}}
\renewcommand\chaptermark[1]{\markright{\arabic{chapter} #1}}
\renewcommand\sectionmark[1]{}
\renewcommand\subsectionmark[1]{}

% Schriften auf Serif umstellen
\addtokomafont{descriptionlabel}{\rmfamily}
\addtokomafont{disposition}{\rmfamily}

% Zeilenumbrüche in Gleichungen
 \allowdisplaybreaks

% Damit auf Teil-Seiten keine Seitennummer angezeigt wird
\fancypagestyle{plain}{\fancyhf{}
  \renewcommand{\headrulewidth}{0pt}
  \renewcommand{\footrulewidth}{0pt}}

% Kopf- und Fußzeile
% Höhe der Kopfzeile
\setlength{\headheight}{14pt}
% obere Trennlinie
%\renewcommand{\headrulewidth}{0.4pt}
\fancyhf{} %alle Kopf- und Fußzeilenfelder bereinigen
\fancyhead[L]{\textbf{IK2a - Thermodynamik und Festkörperphysik}} %Kopfzeile links
%\fancyhead[C]{\leftmark} %zentrierte Kopfzeile
\fancyhead[R]{\rightmark} %Kopfzeile rechts
\fancyfoot[C]{\thepage\quad\!\!\!\slash\quad\!\!\!\pageref{END-front}} %Seitenzahl der Front-Matter

\AtBeginDocument{
  \def\labelitemi{\normalfont\bfseries{--}}
  \def\labelitemii{\(\circ\)}
  \def\labelitemiii{\(\triangleright\)}
}

\makeatother

\begin{document}


\global\long\def\norm#1{\left\lVert #1\right\rVert }


\global\long\def\abs#1{\left\lvert #1\right\rvert }


\global\long\def\opnorm#1{\opnor{#1}}


\global\long\def\BRA#1{\Bra{#1}}


\global\long\def\KET#1{\Ket{#1}}


\global\long\def\BraKet#1{\Braket{#1}}


\global\long\def\mins{\textnormal{-}}


\global\long\def\exs{\exsop}


\global\long\def\exsg{\exsgop}


\global\long\def\fall{\fallop}


\global\long\def\bcupd{\bcupdop}


\global\long\def\bcapd{\bcapdop}


\global\long\def\sr#1#2#3{\underset{#3}{\overset{#2}{#1}}}


\global\long\def\dd{\textnormal{d}}


\global\long\def\DD{\textnormal{D}}


\global\long\def\dbar{\dBar}


\global\long\def\angs{\ang}


\global\long\def\TT{\textnormal{T}}


\global\long\def\ii{\textbf{i}}


\global\long\def\modulo#1#2{\moduloT{#1}{#2}}


\global\long\def\lmodulo#1#2{\lmoduloT{#1}{#2}}


\global\long\def\modlo#1#2{\modloT{#1}{#2}}


\global\long\def\moduloScript#1#2{\moduloScriptT{#1}{#2}}


\global\long\def\lmoduloScript#1#2{\lmoduloScriptT{#1}{#2}}


\global\long\def\modloScript#1#2{\modloScriptT{#1}{#2}}


\global\long\def\vek#1{\vectorsym{#1}}


\global\long\def\mat#1{\matrixsym{#1}}


\global\long\def\ten#1{\tensorsym{#1}}


\global\long\def\msd#1{\mathstrut_{#1}}


\global\long\def\msu#1{\mathstrut^{#1}}


\pagenumbering{roman}


\title{{\Huge \hspace*{1mm}\vspace*{-15mm}}\\
{\Huge Integrierter Kurs IIa}\\
{\Huge Thermodynamik und Festkörperphysik}}


\author{\vspace*{-5mm}\\
\textit{\small Vorlesung von}\\
\noun{\small Prof. Dr. Christian Back}\\
\textit{\small im Wintersemester 2012}\\
\textit{\small Überarbeitung und Textsatz in \LyX{} von}\\
\noun{\small Andreas Völklein}\\
\vspace*{5mm}\\
\includegraphics[clip,width=15cm]{unir}\\
\vspace*{3mm}\\
{\normalsize Stand: \today}\\
\vspace*{-30mm}}

\maketitle
\fancyhead[R]{Lizenz}


\subsubsection*{ACHTUNG}

Diese Mitschrift ersetzt \emph{nicht} die Vorlesung.

Es wird daher \emph{dringend} empfohlen, die Vorlesung zu besuchen.

\vfill{}


\selectlanguage{english}%

\subsubsection*{Copyright Notice}

Copyright © 2012-2013 \noun{Andreas Völklein}

Permission is granted to copy, distribute and/or modify this document
under the terms of the GNU Free Documentation License, Version 1.3
or any later version published by the Free Software Foundation;

with no Invariant Sections, no Front-Cover Texts, and no Back-Cover
Texts.

A copy of the license is included in the document entitled “GFDL”.


\subsubsection*{Disclaimer of Warranty}

\noun{Unless otherwise mutually agreed to by the parties in writing
and to the extent not prohibited by applicable law, }\textbf{\noun{the
Copyright Holders and any other party, who may distribute the Document
as permitted above,   provide the Document “as is}}\textbf{”,}\textbf{\noun{
without warranty of any kind}}\noun{, expressed, implied, statutory
or otherwise, including, but not limited to, the implied warranties
of merchantability, fitness for a particular purpose, non-infringement,
the absence of latent or other defects, accuracy, or the absence of
errors, whether or not discoverable.}


\subsubsection*{Limitation of Liability}

\textbf{\noun{In no event}}\noun{ unless required by applicable law
or agreed to in writing }\textbf{\noun{will the Copyright Holders,
or any other party, who may distribute the Document as permitted above,
be liable to you for any damages}}\noun{, including, but not limited
to, any general, special, incidental, consequential, punitive or exemplary
damages, however caused, regardless of the theory of liability, arising
out of or related to this license or any use of or inability to use
the Document, even if they have been advised of the possibility of
such damages.}

\textbf{\noun{In no event will the Copyright Holders'/Distributor's
liability to you}}\noun{, whether in contract, tort (including negligence),
or otherwise, }\textbf{\noun{exceed the amount you paid the Copyright
Holders/Distributor}}\noun{ for the document under this agreement.}

\selectlanguage{ngerman}%

\subsubsection*{Links}

Der Text der „\foreignlanguage{english}{GNU Free Documentation License}“
kann auch auf der Seite
\begin{quote}
\url{https://www.gnu.org/licenses/fdl-1.3.de.html}
\end{quote}
nachgelesen werden.

Eine transparente Kopie der aktuellen Version dieses Dokuments kann
von
\begin{quote}
\url{https://github.com/andiv/IK2a}
\end{quote}
heruntergeladen werden.

\newpage{}

\fancyhead[R]{Literatur}


\subsection*{Literatur}

Thermodynamik:
\begin{itemize}
\item [$\bullet$]\noun{Torsten Fließbach}: \emph{Lehrbuch zur theoretischen
Physik IV:} \emph{Statistische Physik}; Spektrum Akademischer Verlag,
2010\\
ISBN: 978-3-8274-2527-0
\end{itemize}
Festkörperphysik:
\begin{itemize}
\item [$\bullet$]\noun{Wolfgang Demtröder}: \emph{Experimentalphysik III};
Springer, 2010\\
ISBN: 978-3-642-03910-2\\
(elementar)
\item [$\bullet$]\noun{Neil W. Ashcroft, David N. Mermin}: \emph{Festkörperphysik};
Oldenbourg Verlag, 2013\\
ISBN: 978-3-486-71301-5\\
(sehr ausführlich, sehr empfehlenswert)
\item [$\bullet$]\noun{Charles Kittel}: \emph{Einführung in die Festkörperphysik};
Oldenbourg Verlag, 2006\\
ISBN: 978-3-486-57723-5\\
(Klassiker!)
\item [$\bullet$]\noun{Siegfried Hunklinger}: \emph{Festkörperphysik};
Oldenbourg Verlag, 2011\\
ISBN: 978-3-486-70547-8\\
(auch ein sehr schönes Buch)
\end{itemize}
{\small \newpage{}}\fancyhead[R]{Inhaltsverzeichnis}
\fancyhead[C]{}

\tableofcontents{}\label{END-front}\newpage{}\pagenumbering{arabic}
\fancyfoot[C]{\thepage\quad\!\!\!\slash\quad\!\!\!\pageref{END}} % Seitenzahl des Hauptteils
\fancyhead[R]{\rightmark}
%\fancyhead[C]{\leftmark}%DATE: Mi 17.10.2012


\part{Thermodynamik}


\chapter{Einführung in die Thermodynamik}

\noindent \begin{center}
\begin{tabular}{c|c}
Thermodynamik$\vphantom{\Big|}$ & statistische Physik\tabularnewline
\hline 
\hline 
makroskopische Eigenschaften$\vphantom{\Big|}$ & mikroskopische Modelle\tabularnewline
\hline 
„Axiome“: Hauptsätze$\vphantom{\Big|}$ & $H\left(x\right)\to E_{r}\left(x\right)\to\Omega\left(E,x\right)$\tabularnewline
und Zustandsgleichungen & $\Rightarrow S,T,X$\tabularnewline
\end{tabular}
\par\end{center}


\section{Grundbegriffe der Thermodynamik}

Viele Phänomene werden durch wenige physikalische Größen beschrieben.
\begin{itemize}
\item Mechanik: Ort, Impuls, Kraft und Geschwindigkeit ($3N$ Koordinaten)
\item Elektrodynamik: elektrischer Ladung, magnetischer Fluss, Spannung
elektrischer Strom, elektromagnetische Felder
\item Wärmelehre: Entropie, Druck, Temperatur, $\ldots$
\end{itemize}

\subsubsection*{Definitionen}
\begin{enumerate}[label=\alph*)]
\item \emph{thermodynamisches System}: Gegenstand der physikalischen Betrachtung
(Gas, Flüssigkeit, Festkörper, Quantenvakuum) $\to$ Reduktion auf
physikalische Größen!
\item \emph{(thermodynamische) Zustandsgrößen}: messbare makroskopische
Größen eines Systems\\
z.B.: Volumen $V$, Druck $P$, Temperatur $T$, Energie $E$, Teilchenzahl
$N$, Entropie $S$, $\ldots$
\item \emph{thermodynamischer Zustand}: Momentaufnahme des Systems
\item \emph{Gleichgewichtszustand} (Abkürzung: GGW): Die Zustandsgrößen
ändern sich nicht mehr. Dieser Zustand stellt sich nach hinreichend
langer Zeit von selbst ein. Daher lassen sich Zustandsgrößen nur im
Gleichgewichtszustand eindeutig definieren und messen. (Wichtig!)
Wir gehen immer davon aus, dass wir Gleichgewichtszustände betrachten.
\item \emph{Zustandsänderung (,,Prozesse“)}: Änderung äußerer Parameter
($V,P,$ externe Felder etc.)
\begin{align*}
\text{GGW}_{1} & \to\text{GGW}_{2}
\end{align*}
Eine \emph{quasistatische} Zustandsänderung ist so langsam, dass das
System immer im Gleichgewichtszustand ist.\\
Für nicht quasistatische Prozesse keine Aussage über die Zwischenzustände
(Nichtgleichgewichtszustände)!
\item \emph{Zustandsvariablen}: Minimaler Satz von Zustandsgrößen, die einen
Gleichgewichtszustand festlegen.
\begin{align*}
y & \in\left\{ \left(E,V,N\right),\ \left(T,V,N\right),\ \left(T,p,N\right)\right\} 
\end{align*}
Alle Zustandsgrößen sind Funktionen dieser Variablen. Eine \emph{Zustandsgleichung}
ist ein Ausdruck der Form:
\begin{align*}
0 & =f\left(y\right)
\end{align*}

\item \emph{Extensive und intensive Größen}: Größen die proportional zur
Systemgröße sind, sich also verdoppeln, wenn man zwei gleiche Systeme
zusammen betrachtet, heißen \emph{extensiv}. Von der Systemgröße unabhängige
Größen heißen \emph{intensiv}.


\noindent \begin{center}
\begin{tabular}{c|c}
intensive Größe$\vphantom{\Big|}$ & extensive Größe\tabularnewline
\hline 
\hline 
Geschwindigkeit $\vec{v}$$\vphantom{\Big|}$ & Gesamtimpuls $\vec{p}$\tabularnewline
\hline 
Winkelgeschwindigkeit $\vec{\omega}$$\vphantom{\Big|}$ & Gesamtdrehimpuls $\vec{L}$\tabularnewline
\hline 
Kraft $\vec{F}$$\vphantom{\Big|}$ & Verschiebung $\vec{r}$\tabularnewline
\hline 
Druck $P$$\vphantom{\Big|}$ & Volumen $V$\tabularnewline
\hline 
Spannung $U$$\vphantom{\Big|}$ & elektrische Ladung $Q$\tabularnewline
\hline 
Stromstärke $I$$\vphantom{\Big|}$ & magnetischer Fluss $\phi$\tabularnewline
\end{tabular}
\par\end{center}


Dies sind konjugierte Paare.


Welche Variablen verwendet man? $\to$ Je nachdem, welche zweckmäßig
sind!


Verwende die ,,physikalische Kurzschrift``
\begin{align*}
E\left(y_{1}\right) & =E\left(y_{2}\right)=\ldots
\end{align*}
anstelle von:
\begin{align*}
E\left(y_{1}\right) & =\tilde{E}\left(y_{2}\right)
\end{align*}
Zum Beispiel:
\begin{align*}
E\left(\vec{p}\right) & =\frac{\vec{p}^{2}}{2m}=E\left(\vec{k}\right)=\frac{\hbar^{2}\vec{k}^{2}}{2m}
\end{align*}


\end{enumerate}

\section{Die Gibbssche Fundamentalform}

Die zentrale Annahme der Thermodynamik ist, dass die Energie als Funktion,
genannt „Gibbs-Funktion“, aller \emph{unabhängigen extensiven} Variablen
dargestellt werden kann.
\begin{align}
E & =E\left(x_{1},\ldots,x_{n}\right)
\end{align}
Das totale Differential der Gibbs-Funktion liefert die Gibbssche Fundamentalform:
\begin{align}
\fbox{\ensuremath{{\displaystyle \dd E=-\sum_{i=1}^{n}X_{i}\dd x_{i}}}}
\end{align}
Dabei sind
\begin{align}
X_{i}\left(x_{1},\ldots,x_{n}\right) & :=\frac{\partial E\left(x_{1},\ldots,x_{n}\right)}{\partial x_{n}}\label{eq:verallgemeinerte-Kraefte}
\end{align}
die verallgemeinerten Kräfte. Falls $E\left(x_{1},\ldots,x_{n}\right)$
bekannt ist, beschreiben (\ref{eq:verallgemeinerte-Kraefte}) die
Zustandsgleichungen.


\subsubsection*{Beispiele}
\begin{itemize}
\item Kondensator:\\
\textcolor{green}{TODO: Abb1}
\begin{align*}
\dd E & =U\dd Q
\end{align*}

\item Kondensator mit variablem Plattenabstand:
\begin{align*}
\dd E & =U\left(Q,x\right)\dd Q-F\left(Q,x\right)\dd x
\end{align*}

\end{itemize}
Jeder Term in der Gibbsschen Fundamentalform ist ein Kanal, über den
Energie zu oder abgeführt werden kann.
\begin{align*}
\vec{v}\dd\vec{p} & \quad\text{ „Beschleunigungsarbeit“}\\
\vec{\omega}\dd\vec{L} & \quad\text{ „Rotationssarbeit“}\\
-\vec{F}\dd\vec{x} & \quad\text{„Verschiebungsarbeit“}\\
-P\dd V & \quad\text{„Kompressionsarbeit“}\\
U\dd Q & \quad\text{„Aufladungsarbeit (Kondensator)“}\\
I\dd\phi & \quad\text{„Aufladungsarbeit (Spule)“}
\end{align*}



\section{Was ist Wärme?}

Historische Entwicklung:
\begin{itemize}
\item Wärme ist Materialfluss, aber durch Reibung erzeugbar?
\item Vermutung: „Wärme $=$ Energie“, aber die Energieformen sind nicht
äquivalent.
\item Idee (Clausius, 1850): Zufuhr von Wärme ist die gleichzeitige Zufuhr
von Energie $\dbar Q$ und Entropie $\dd S$.
\item Zur Beschreibung von thermischen Phänomenen wird ein neues Paar konjugierter
Größen benötigt:
\begin{align*}
\text{Temperatur }T & \quad\text{„Wärmespannung“}\\
\text{Entropie }S & \quad\text{„Wärmeladung“}
\end{align*}
Dies liefert für die Gibbssche Fundamentalform:
\begin{align}
\dd E & =T\dd S
\end{align}
Temperatur ist die verallgemeinerte Kraft zur Entropie.
\item Beide Größen sind experimentell messbar.
\item Beachte: Die Wärmeenergie ist keine Zustandsgröße, weshalb $\dbar Q$
statt $\dd Q$ geschrieben wird. Hingegen sind die Größen $S$ und
$T$ Zustandsgrößen.
\item Mikroskopisch: Wärme ist die Energie der ungeordneten Bewegung der
Atome.
\begin{align}
S & =k_{B}\ln\left(\Omega\right)\\
\frac{1}{T} & =\frac{\partial S\left(E,x\right)}{\partial E}
\end{align}
$\Omega$ ist die Zustandssumme, also die Zahl der Mikrozustände.\\
Entropie ist ein Maß für die Unordnung des Systems. Temperatur gibt
die Energie pro Freiheitsgrad an.
\end{itemize}
%DATE: Do 18.10.12


\section{Die Hauptsätze der Thermodynamik}

Im Rahmen der Thermodynamik sind die Hauptsätze experimentelle Befunde.
Sie lassen sich aber mittels statistischer Physik herleiten.

Vorbemerkung: Zwei Systeme im thermischen Gleichgewicht haben die
gleiche Temperatur.
\begin{itemize}
\item \emph{0. Hauptsatz}: Alle Systeme, die sich mit einem gegebenen System
im Gleichgewicht befinden, sind auch untereinander im Gleichgewicht.
\item \emph{1. Hauptsatz}: Die Energie eines Systems kann sich nur durch
Zufuhr von Wärme oder Arbeit ändern.
\begin{align}
\dd E & =\dbar Q+\dbar W &  & \text{(Energieerhaltung)}
\end{align}
Arbeit ist eine Energieänderung im thermisch isolierten System (mechanisch,
elektrisch, magnetisch, $\ldots$). Sie ist positiv, $\dbar W>0$,
wenn Arbeit am System geleistet wird.\\
Wärmeenergie ist der Energieübertrag bei konstanten externen Parametern
und $\dbar Q>0$ steht für Wärmezufuhr.
\item \emph{2. Hauptsatz}: 

\begin{enumerate}[label=\alph*)]
\item Für abgeschlossene Systeme sind nur Prozesse möglich, für die die
Entropie nicht abnimmt, also $\Delta S\ge0$ gilt.
\item Für nicht abgeschlossene Systeme gilt:
\begin{align}
\dd S & \ge\frac{\dbar Q}{T}
\end{align}
Bei quasistatischem Wärmeübertrag $\dbar Q_{\text{qs}}$ gilt:
\begin{align}
\dd S & =\frac{\dbar Q_{\text{qs}}}{T}
\end{align}

\end{enumerate}

Beachte: $\Delta S=0$ bedeutet, dass der Prozesse reversibel und
in einem abgeschlossenen System abläuft. Reversible Prozesse lassen
sich durch extrem langsame Zustandsänderungen, man sagt durch \emph{quasistatische
Prozesse}, approximieren.
\begin{itemize}
\item Die Entropiedifferenz zwischen Gleichgewicht \emph{a} und Gleichgewicht
$b$
\begin{align*}
\Delta S & =\int_{a}^{b}\frac{\dbar Q_{\text{qs}}}{T}
\end{align*}
ist experimentell messbar!
\item Es gibt zahlreiche äquivalente Formulierungen des 2. Hauptsatzes:\\
,,Ein perpetuum mobile 2. Art ist unmöglich.``\\
„Entropie kann erzeugt, aber nicht vernichtet werden.“
\item Für quasistatische Prozesse gilt
\begin{align*}
\dbar W_{\text{qs}} & =-\sum_{i}X_{i}\dd x_{i}
\end{align*}
mit den verallgemeinerten Kräften:
\begin{align*}
X_{i} & :=-\frac{\partial E\left(x_{i},\ldots\right)}{\partial x_{i}}
\end{align*}
\emph{Beispiel}: \\
\textcolor{green}{TODO: Abb1}\\
\begin{align*}
\dbar W_{\text{qs}} & =\ldots-P\dd V
\end{align*}
Hier ist der Druck $P$ die verallgemeinerte Kraft die konjugierte
Größe zum Volumen.
\end{itemize}
\item \emph{3. Hauptsatz}: Es gilt das Nernst-Theorem:
\begin{align}
\lim_{T\to0}S & =0
\end{align}


\begin{itemize}
\item Thermodynamische Bedeutung: Festlegung der absoluten Größe der Entropie
$S$.
\item Mikroskopische Bedeutung:
\begin{align*}
S\left(T,x\ldots\right) & =k_{B}\ln\left(\Omega\left(T,x,\ldots\right)\right)
\end{align*}
Hier ist $\Omega$ die Zahl der möglichen Mikrozustände für den Makrozustand.
Es gibt also nur \emph{einen} (oder allgemeiner sehr wenige) Mikrozustände
am absoluten Temperatur-Nullpunkt.
\end{itemize}
\end{itemize}

\section{Messung von Zustandsgrößen}
\begin{enumerate}[label=\alph*)]
\item \emph{Energie}:\\
\textcolor{green}{TODO: Abb2}\\
Die Messung von Energiedifferenzen zwischen Gleichgewichtszuständen
erfolgt mit Hilfe des 1. Hauptsatzes:
\begin{align*}
\Delta E & =E_{b}-E_{a}=A_{\text{mech}}+\underbrace{\Delta Q}_{=A_{\text{elektrisch}}}
\end{align*}

\item \emph{Wärmemenge}: Nach dem 1. Hauptsatz gilt:
\begin{align*}
\Delta Q & =\Delta E-\Delta W & \left[Q\right] & =\text{J}=\text{N}\cdot\text{m}=\text{V}\cdot\text{A}\cdot\text{s}
\end{align*}

\item \emph{Temperatur}: Gehe wie folgt vor: Bringe ein „kleines“ Messsystem
$M$ in thermisches Gleichgewicht mit dem Testsystem $A$, sodass
sich die Temperaturen angleichen, $T_{A}=T_{M}$, und miss $T_{M}$.\\
Nutze zum Beispiel die Zustandsgleichung des idealen Gases:
\begin{align*}
P_{M} & =\frac{N}{V}k_{B}T_{M} & P_{M}' & =\frac{N}{V}k_{B}T_{M}'
\end{align*}
\begin{align*}
\Rightarrow\qquad\frac{T_{M}'}{T_{M}} & =\frac{p_{M}'}{p_{M}}
\end{align*}
Dies liefert eine lineare Temperatur-Skala. Definiere neben $T=0$
einen weiteren Referenzpunkt:\\
\textcolor{green}{TODO: Abb3}\\
T: Tripelpunkt, K: kritischer Punkt\\
Definiton: Der \emph{Tripelpunkt} von Wasser liege bei $T_{t}=273{,}16\text{K}$.
Es folgt:
\begin{align}
1\text{K} & =\frac{1}{273{,}16}T_{\text{T}}
\end{align}
\emph{Anmerkung}: Die Celsius Skala ist keine SI-Einheit! 
\begin{align}
\theta & =\left(\frac{T}{\text{K}}-273{,}16\right)\text{\textdegree}\text{C}\\
\theta_{\text{Gefrierpunkt}} & :=0\text{\textdegree}\text{C}\nonumber \\
\theta_{\text{Siedepunkt}} & :=100\text{\textdegree}\text{C}\nonumber 
\end{align}
Dies gilt nur bei Normaldruck $P=1013\,\text{hPa}$.


Praktische Thermometer:
\begin{itemize}
\item Benutze die Wärmeausdehnung, das heißt bekannte Koeffizienten.
\begin{align*}
\beta & :=\frac{1}{V}\frac{\dd V\left(T\right)}{\dd T} & \beta' & :=\frac{1}{L}\frac{\dd L\left(T\right)}{\dd T}
\end{align*}
Häufig in Verwendung sind Quecksilber oder Metalle.
\item Temperaturabhängigkeit des elektrischen Widerstands
\item Gas bei konstantem Druck
\end{itemize}
\item \emph{Entropie}:
\begin{align}
\Delta S & =\int_{a}^{b}\frac{\dbar Q_{\text{qs}}}{T} &  & \text{nach 2. Hauptsatz}\\
\lim_{T\to0}S\left(T\right) & =0 &  & \text{nach 3. Hauptsatz}
\end{align}

\end{enumerate}

\chapter{Das ideale Gas}


\section{Zustandsgleichung}

Ausgangspunkt ist die empirisch gefundene thermische Zustandsgleichung:
\begin{align}
P & =\frac{\nu RT}{V}=\frac{RT}{v} &  & \text{(ideales Gas)}\label{eq:Zustandsgleichung-ideals_Gas}\\
P & =-\frac{a}{v^{2}}+\frac{RT}{v-b} &  & \text{(van der Waals-Gas)}
\end{align}
$\nu$: Stoffmenge, $v=\frac{V}{\nu}$: molares Volumen, $R$: allgemeine
Gaskonstante, $a$: Binnendruck, $b$: Binnenvolumen

\emph{Energie}: (konstantes $\nu$)

Nach dem 1. Hauptsatz gilt:
\begin{align*}
\dd E & =\dbar Q_{\text{qs}}+\dbar W_{\text{qs}}
\end{align*}
Mit dem 2. Hauptsatz folgt:
\begin{align*}
\dbar Q_{\text{qs}} & =T\dd S\\
\dbar W_{\text{qs}} & =-P\dd V
\end{align*}
Es folgt:
\begin{align}
\dd S & =\frac{1}{T}\dd E+\frac{P}{T}\dd V
\end{align}
Setze (\ref{eq:Zustandsgleichung-ideals_Gas}) und das vollständige
Differential
\begin{align*}
\dd E & =\frac{\partial E\left(T,V\right)}{\partial T}\dd T+\frac{\partial E\left(T,V\right)}{\partial V}\dd V
\end{align*}
 von $E\left(T,V\right)$ ein. Es folgt:
\begin{align*}
\dd S= & \underbrace{\frac{1}{T}\frac{\partial E\left(T,V\right)}{\partial T}}_{\frac{\partial S\left(T,V\right)}{\partial T}}\dd T+\underbrace{\left(\frac{1}{T}\frac{\partial E\left(T,V\right)}{\partial V}+\frac{\nu R}{V}\right)}_{=\frac{\partial S\left(T,V\right)}{\partial V}}\dd V
\end{align*}
\begin{align*}
\frac{\partial^{2}S\left(T,V\right)}{\partial T\partial V} & =\frac{\partial}{\partial V}\left(\frac{\partial S\left(T,V\right)}{\partial T}\right)=\frac{1}{T}\frac{\partial^{2}E\left(T,V\right)}{\partial T\partial V}=\\
 & \stackrel{!}{=}\frac{\partial}{\partial T}\left(\frac{\partial S\left(T,V\right)}{\partial V}\right)=\frac{1}{T}\frac{\partial^{2}E\left(T,V\right)}{\partial T\partial V}-\frac{1}{T^{2}}\frac{\partial E\left(T,V\right)}{\partial V}
\end{align*}
\begin{align}
\Rightarrow\qquad\frac{\partial E\left(T,V\right)}{\partial V} & =0\nonumber \\
\Rightarrow\qquad E\left(T,V\right) & =E\left(T\right)\label{eq:E_E(T)}
\end{align}
Die Energie des idealen Gases hängt also nicht vom Volumen ab. Die
experimentelle Überprüfung erfolgt mittels freier Expansion.\\
\textcolor{green}{TODO: Abb4}\\
Für ideales Gas gilt: $T_{a}=T_{b}$

%DATE: Mo 22.10.12


\subsubsection{Zustandssumme des idealen Gases}

Die Berechnung der Zustandssumme geht vom Hamiltonoperator $H=\hat{H}\left(\hat{q},\hat{p},x\right)$
aus. Hier ist nur die Abhängigkeit von $V$ und $N$ interessant.
\begin{align*}
H & =H\left(V,N\right)
\end{align*}
Das Vorgehen ist wie folgt:
\begin{itemize}
\item Berechne die Energieeigenwerte $E_{\nu}\left(V,N\right)$ der Mikrozustände.
\item Berechne die Zustandssumme:
\begin{align*}
\fbox{\ensuremath{\Omega\left(E,V,N\right)}}
\end{align*}

\item Berechne zunächst die Entropie:
\begin{align*}
S & =S\left(E,V,N\right)=k_{B}\text{ln}\left(\Omega\right)
\end{align*}

\item Berechne alle anderen makroskopischen Größen und Beziehungen, insbesondere
die kalorische Zustandsgleichung $E=E\left(T,V,N\right)$ und die
thermische Zustandsgleichung $P=P\left(T,V,N\right)$.
\end{itemize}
\emph{Beispiel}: ideales Gas

\begin{align*}
H\left(V,N\right) & =\sum_{\nu=1}^{N}\frac{-\hbar^{2}}{2m}\Delta_{\nu}+U\left(\vec{r}_{\nu}\right)=\sum_{\nu=1}^{N}\underbrace{h\left(\vec{r}_{\nu},\vec{p}_{\text{op},\nu},V\right)}_{\text{1-Teilchen-Hamiltonop.}}
\end{align*}
Das Potential $U\left(\vec{r}_{\nu}\right)$ verschwindet im Inneren
des Volumens und ist sonst $\infty$.

Ideales Gas:
\begin{itemize}
\item keine inneren Freiheitsgrade der Kerne
\item keine Vibrationen, Rotationen (z.B. 1-atomig)
\item keine Wechselwirkung der Atome untereinander
\end{itemize}
Es handelt sich also um Teilchen im unendlich hohen Potentialtopf.
Zur Einfachheit sei $V=L^{3}$ ein Würfel. Wir erhalten quantisierte
Impulse:
\begin{align*}
p_{k} & =p_{3\nu+j-3}=\frac{\pi\hbar}{L}n_{k}\quad\text{mit}\quad\begin{cases}
\nu\in\left\{ 1,\ldots,N\right\} \\
j\in\left\{ 1,2,3\right\} \\
k\in\left\{ 1,2,\ldots,3N\right\} 
\end{cases}
\end{align*}
Ein Mikrozustand wird beschrieben durch
\begin{align*}
r & =\left(\ldots,n_{k},\ldots\right)=\left(n_{1},n_{2},\ldots,n_{3N}\right)
\end{align*}
mit $n_{k}\in\left\{ 1,2,\ldots\right\} $. Der Energieeigenwert des
Mikrozustands ist:
\begin{align*}
E_{r}\left(V,N\right) & =\sum_{\nu=1}^{N}\frac{\vec{p}_{\nu}^{2}}{2m}=\sum_{n=1}^{3N}\frac{\pi^{2}\hbar^{2}}{2mL^{2}}n_{k}^{2}
\end{align*}
Das Randpotential führt zur Quantisierung.

\emph{Jetzt}: Finde die mikrokanonische Zustandssumme.

\begin{framed}

Für ein abgeschlossenes System ist \emph{E} eine Erhaltungsgröße.
Daher sind nur Energien $E_{\nu}$ in der Nähe von \emph{E} zugänglich!

\end{framed}

Dies ist eine Aufgabe der statistischen Physik: Betrachte den Phasenraum
und summiere \emph{zugängliche} Zustände.
\begin{align*}
\Phi\left(E,V,N\right) & =\sum_{\nu:\, E_{\nu}\left(V,N\right)\le E}1\\
\Omega\left(E\right) & =\Phi\left(E\right)-\Phi\left(E-\delta E\right)
\end{align*}
Bis auf einen Vorfaktor erhält man zum Beispiel:
\begin{align*}
\Omega\left(E,V\right) & \approx c\cdot V^{N}E^{\frac{3N}{2}}
\end{align*}
Unter Berücksichtigung von $N\gg1$ folgt:
\begin{align*}
\ln\left(\Omega\left(E,V,N\right)\right) & =\frac{3N}{2}\ln\left(\frac{E}{N}\right)+N\ln\left(\frac{V}{N}\right)+N\ln\left(c'\right)
\end{align*}


\emph{Aussage}: Die Anzahl der Zustände pro Atom ist proportional
zu $\frac{V}{N}$ und zu $\left(\frac{E}{N}\right)^{\frac{3}{2}}$.

\emph{Und was gilt für ein reales Gas?}

Beim idealen Gas gilt:
\begin{align*}
Pv & =k_{B}T
\end{align*}


Dabei ist $v=\frac{V}{N}$ das \emph{reduzierte Volumen}.

Van der Waals machte einen klassischen thermodynamischen Ansatz:
\begin{align*}
\left(P+P_{\text{B}}\right)\left(v-b\right) & =k_{B}T
\end{align*}
Dabei ist $P_{B}$ der Binnendruck und $b$ das Ausschlussvolumen.
Für ein Modell harter Kugeln ist das Ausschlussvolumen $b=v_{0}=\text{konst.}$,
aber realistischer ist $b=b\left(T,p\right)$.\\
Für $v\to0$ (hohe Verdünnung) muss $P_{B}v\to0$ gehen, damit sich
im Grenzfall das ideale Gas ergibt. Entwickle also $P_{B}\left(v\right)$
als Potenzreiche in $\frac{1}{v}$.
\begin{align*}
P_{B} & =\frac{c}{v}+\frac{a}{v^{2}}+\frac{d}{v^{3}}+\ldots
\end{align*}
Einsetzen und bilden des Limes $v\gg b$ ergibt:
\begin{align*}
\fbox{\ensuremath{{\displaystyle Pv+c+\frac{a}{v}+\frac{d}{v^{2}}=k_{B}T}}}
\end{align*}
$c$ muss Null sein, damit man für $v\to\infty$ das ideale Gas erhält.
Berücksichtige nur den führenden Term in der Volumenabhängigkeit und
erhalte:
\begin{align*}
\fbox{\ensuremath{{\displaystyle \left(P+\frac{a}{v^{2}}\right)\left(v-b\right)=k_{B}T}}}
\end{align*}



\section{Wärmekapazitäten}
\begin{itemize}
\item Betrachte ein System, dem die Wärmemenge $\dbar Q$ quasistatisch
zugeführt wird, um die Temperatur $T$ auf $T+\dd T$ zu erhöhen.
\begin{align*}
\dd S & =\frac{\dbar Q_{\text{qs}}}{T}
\end{align*}

\item Allgemeine Definition: Die Wärmekapazität bei konstantem Volumen $V$
ist:
\begin{align}
C_{V} & =\frac{\dd Q_{\text{qs}}\left(T,V\right)}{\dd T}\bigg|_{V=\text{konst.}}\stackrel{\text{2. HS}}{=}\frac{T\dd S}{\dd T}\bigg|_{V=\text{konst.}}=:T\frac{\partial S\left(T,V\right)}{\partial T}
\end{align}

\item Analog definiert man die Wärmekapazität bei konstantem Druck $P$:
\begin{align}
C_{P} & =T\frac{\partial S\left(T,P\right)}{\partial T}
\end{align}
Oft verwendet man die Wärmekapazität pro Molmasse $c:=\frac{C}{M}$
auch \emph{spezifische Wärme} genannt.
\item Betrachte zum Beispiel ein Gas mit den Zustandsvariablen Temperatur
$T$ und Volumen $V$ oder Druck $P$, so gilt:
\begin{align}
\dd S & =\frac{1}{T}\dd E+\frac{P}{T}\dd V=\frac{1}{T}\frac{\partial E\left(T,V\right)}{\partial T}\dd T+\frac{1}{T}\frac{\partial E\left(T,V\right)}{\partial V}\dd V+\frac{P}{T}\dd V
\end{align}
Es folgt:
\begin{align}
\fbox{\ensuremath{{\displaystyle C_{V}=T\frac{\partial S\left(T,V\right)}{\partial T}=\frac{\partial E\left(T,V\right)}{\partial T}}}}
\end{align}
(Dies gilt allgemein, auch im realen Gas.)
\item Im idealen Gas gilt nach (\ref{eq:E_E(T)}):
\begin{align*}
\frac{\partial E\left(T,V\right)}{\partial V} & =0
\end{align*}
Damit folgt:
\begin{align}
\dd S & =\frac{C_{V}}{T}\dd T+\frac{\nu R}{V}\dd V\label{eq:dS(Cv)}
\end{align}

\item Bestimmung von ${\displaystyle C_{P}=T\frac{\partial S\left(T,P\right)}{\partial T}}$:\\
Benötige $S\left(T,P\right)$. Der Ausgangspunkt ist die thermische
Zustandsgleichung:
\begin{align*}
V & =\frac{\nu RT}{P}\\
\dd V & =\frac{\nu R}{P}\dd T-\frac{\nu RT}{P^{2}}\dd P
\end{align*}
Einsetzen in (\ref{eq:dS(Cv)}) ergibt:
\begin{align*}
\dd S & =\left(\frac{C_{V}}{T}+\frac{\nu^{2}R^{2}}{VP}\right)\dd T-\frac{\nu^{2}R^{2}T}{VP^{2}}\dd P
\end{align*}
Mit 
\begin{align*}
\frac{\nu R}{V} & =\frac{P}{T}
\end{align*}
folgt:
\begin{align*}
\dd S & =\frac{C_{V}+\nu R}{T}\dd T-\frac{\nu R}{P}\dd P
\end{align*}
Also gilt:
\begin{align}
\fbox{\ensuremath{{\displaystyle C_{P}=C_{V}+\nu R}}} &  & \fbox{\ensuremath{{\displaystyle c_{P}-c_{V}=R}}}
\end{align}
Dies gilt nur für das ideale Gas, \emph{nicht} für das reale.
\item Einige Konstanten:\\
\emph{Allgemeine Gaskonstante}:
\begin{align*}
R & =N_{A}k_{B}=8{,}314\ldots\,\frac{\text{J}}{\text{mol}\cdot\text{K}}
\end{align*}
\emph{Boltzmankonstante}:
\begin{align*}
k_{B} & =1{,}38\cdot10^{-23}\,\frac{\text{J}}{\text{K}}
\end{align*}
\emph{Avogadrokonstante}:
\begin{align*}
N_{A} & =6{,}022\cdot10^{-23}\,\frac{1}{\text{mol}}
\end{align*}

\item Aus der mikroskopischen Betrachtung folgt für das (einatomige) ideale
Gas die kalorische Zustandsgleichung
\begin{align}
E\left(T\right) & =\frac{3}{2}\nu RT=C_{V}\cdot T
\end{align}
sowie:
\begin{align}
C_{V} & =\frac{3}{2}\nu R & c_{V} & =\frac{3}{2}R\\
C_{P} & =\frac{5}{2}\nu R & c_{P} & =\frac{5}{2}R
\end{align}

\item Oft betrachtet man auch den \emph{Adiabaten-/Isentropenkoeffizient}:
\begin{align}
\gamma & =\frac{c_{P}}{c_{V}}=\frac{C_{P}}{C_{V}}=\frac{1}{1-\frac{R}{c_{P}}}
\end{align}

\item Beispiele:


\noindent \begin{center}
\begin{tabular}{c|c|c|c}
Gas$\vphantom{\Big|}$ & $\frac{c_{p}}{R}$ & $\gamma_{\text{exp}}$ & $\gamma_{\text{theo}}$\tabularnewline
\hline 
\hline 
He$\vphantom{\Big|}$ & 2,5 & 1,667 & 1,667\tabularnewline
\hline 
$\text{O}_{2}$$\vphantom{\Big|}$ & $3{,}5$ & 1,387 & 1,397\tabularnewline
\hline 
$\text{CO}_{2}$$\vphantom{\Big|}$ & $4{,}3$ & 1,301 & 1,302\tabularnewline
\end{tabular}\hspace*{3em}%
\begin{tabular}{c|cc}
System$\vphantom{\Bigg|}$ & ${\displaystyle \frac{c_{p}}{R}}$ & \tabularnewline
\cline{1-2} 
ideales Gas$\vphantom{\Big|}$ & $\frac{5}{2}$ & \tabularnewline
\cline{1-2} 
Edelgas$\vphantom{\Big|}$ & $2{,}5$ & \tabularnewline
\cline{1-2} 
$\text{H}_{2}$$\vphantom{\Big|}$ & $3{,}5$ & Vibrationen ,,ausgefroren``\tabularnewline
\cline{1-2} 
Cu$\vphantom{\Big|}$ & $2{,}9$ & \multirow{2}{*}{$\left.\vphantom{\Bigg|}\right\} $siehe Festkörperphysik}\tabularnewline
\cline{1-2} 
Diamant$\vphantom{\Big|}$ & $0{,}7$ & \tabularnewline
\end{tabular}
\par\end{center}

\end{itemize}

\section{Entropie}
\begin{itemize}
\item Aus dem Differential
\begin{align*}
\dd S & =\frac{\nu c_{V}\left(T\right)}{T}\dd T+\frac{\nu R}{V}\dd V
\end{align*}
folgt:
\begin{align}
S\left(T,V\right) & =\nu\int\dd T\frac{c_{V}\left(T\right)}{T}+\nu R\ln\left(V\right)+\text{konst.}\label{eq:Entropie-Integral}
\end{align}

\item Für ein ideales einatomiges Gas gilt $c_{V}=\frac{3}{2}R$ und es
folgt:
\begin{align}
S\left(T,V\right) & =\frac{3}{2}\nu R\ln\left(T\right)+\nu R\ln\left(V\right)+\text{konst.}
\end{align}

\item Zusatzüberlegung: Wie hängt die Entropie von Teilchenzahl $N$ ab?
Mache folgenden Ansatz:
\begin{align}
S\left(T,V,N\right) & =\frac{3}{2}Nk_{B}\ln\left(T\right)+Nk_{B}\ln\left(V\right)+\underbrace{Ng\left(N\right)}_{=?}
\end{align}
Da $S$ eine extensive Größe ist, muss gelten:
\begin{align*}
S\left(T,V,N\right) & =N\cdot s\left(T,\frac{V}{N}\right)
\end{align*}
Also muss $g\left(N\right)$ dergestalt sein, dass gilt:
\begin{align}
S\left(T,V,N\right) & =N\left(\frac{3}{2}k_{B}\ln\left(T\right)+k_{B}\ln\left(\frac{V}{N}\right)+\text{konst.}\right)
\end{align}

\end{itemize}

\section{Adiabatengleichung}

Aus der thermischen Zustandsgleichung $PV=\nu RT$ folgt für konstante
Temperatur (\emph{Isotherme}) das Gesetz von Boyle-Mariotte:
\begin{align*}
PV & =\text{konst.}
\end{align*}


\textcolor{green}{TODO: Plot $pV=\text{konst.}$ bei $T_{1}$ und
$T_{2}>T_{1}$, Beschriftung Isotherme}

\textcolor{green}{TODO: Plot Adiabaten=Isentropen bei $S_{1}$ und
$S_{2}>S_{1}$, Beschriftung Adiabaten=Isentropen}

Analog lassen sich Adiabaten als Kurven für quasistatische, adiabatische
Zustandsänderungen angeben.
\begin{align*}
\dd S & =\frac{C_{V}}{T}\dd T+\left(C_{P}-C_{V}\right)\frac{\dd V}{V}\stackrel{!}{=}0
\end{align*}
Es sei $C_{V}$ unabhängig von der Temperatur, also ${\displaystyle \gamma=\frac{c_{P}}{c_{V}}}$
ebenfalls konstant. Dann gilt
\begin{align*}
\ln T+\left(\gamma-1\right)\ln V & =\text{konst.}
\end{align*}
und es folgt:
\begin{align}
\fbox{\ensuremath{{\displaystyle \ensuremath{TV^{\left(\gamma-1\right)}=\text{konst.}}}}}
\end{align}


Weiter gilt:
\begin{align*}
\dd S & =\frac{C_{P}}{T}\dd T-\left(C_{P}-C_{V}\right)\frac{\dd P}{P}\stackrel{!}{=}0
\end{align*}
Damit ergibt sich für die \emph{Adiabaten}:
\begin{align}
 & \fbox{\ensuremath{TP^{\left(\frac{1}{\gamma}-1\right)}=\text{konst.}}}\\
 & \quad\fbox{\ensuremath{PV^{\gamma}=\text{konst.}}}
\end{align}


%DATE: Mo 29.10.12


\chapter{Thermodynamische Potentiale}


\section{Definitionen}
\begin{itemize}
\item Bislang haben wir die Energie $E$ als Funktion (extensiver) Zustandsvariablen,
zum Beispiel $V$ oder $S$ betrachtet. Nun gehen wir zu verallgemeinerten
(thermodynamischen) Kräften über:
\begin{align}
X_{i} & =\frac{\partial E}{\partial x_{i}}
\end{align}
Ziel des Kapitels ist die Verallgemeinerung auf neue Potentiale.
\item Betrachte einen Gleichgewichtszustand eines thermodynamischen Systems,
der durch zwei Zustandsvariablen definiert ist, zum Beispiel:
\begin{align}
\left(S,V\right);\ \left(T,V\right);\ \left(S,P\right);\ \left(T,P\right)
\end{align}

\item Definiere thermodynamische Potentiale als Zustandsgrößen, deren partielle
Ableitungen thermodynamische Kräfte sind.\\
\emph{Beispiel}: Die Energie $E$ in Fundamentalform ist:
\begin{align*}
\dd E & =T\dd S-P\dd V
\end{align*}
\emph{Analogie zur Mechanik}: Der Ort ist durch $\left(x,y\right)$
festgelegt und das Potential ist $U\left(x,y\right)$. Die (verallgemeinerte)
Kräfte sind hier:
\begin{align*}
K_{x} & =-\frac{\partial U\left(x,y\right)}{\partial x} & K_{y} & =-\frac{\partial U\left(x,y\right)}{\partial y}
\end{align*}
In der \emph{Thermodynamik} haben wir hier die thermodynamische Kräfte:
\begin{align}
T & =\frac{\partial E\left(S,V\right)}{\partial S} & P & =-\frac{\partial E\left(S,V\right)}{\partial V}
\end{align}
Interpretation:

\begin{itemize}
\item Die Temperatur ist die treibende Kraft für den Wärme- oder Entropieaustausch.
\item Der Druck ist die treibende Kraft für den Volumenaustausch.
\end{itemize}
\end{itemize}
\noindent \begin{center}
\begin{tabular}{c|c|c}
Thermodynamisches Potential$\vphantom{\Big|}$ & Differential & natürliche Variable\tabularnewline
\hline 
\hline 
Energie $E$$\vphantom{\Big|}$ & $\dd E=T\dd S-P\dd V$ & $S,V$\tabularnewline
\hline 
Freie Energie $F=E-TS$$\vphantom{\Big|}$ & $\dd F=-S\dd T-P\dd V$ & $T,V$\tabularnewline
\hline 
Enthalpie $H=E+PV$$\vphantom{\Big|}$ & $\dd H=T\dd S+V\dd P$ & $S,P$\tabularnewline
\hline 
Freie Enthalpie $G=E-TS+PV$$\vphantom{\Big|}$ & $\dd G=-S\dd T+V\dd P$ & $T,P$\tabularnewline
\end{tabular}
\par\end{center}

Die Übergänge zwischen den Potentialen nennt man \emph{Legendre-Transformationen}.\\
Nebenbemerkung:
\begin{align*}
\dd\left(TS\right) & =T\dd S+S\dd T\\
\dd\left(pV\right) & =p\dd V+V\dd p
\end{align*}



\section{Thermodynamische Kräfte}

Die thermodynamischen Kräften ergeben sich aus den vollständigen Differentialen
zu:
\begin{alignat}{2}
\frac{\partial E\left(S,V\right)}{\partial S} & =T & \frac{\partial E\left(S,V\right)}{\partial V} & =-P\nonumber \\
\frac{\partial F\left(T,V\right)}{\partial T} & =-S & \frac{\partial F\left(T,V\right)}{\partial V} & =-P\\
\frac{\partial H\left(S,P\right)}{\partial S} & =T & \frac{\partial H\left(S,P\right)}{\partial p} & =V\nonumber \\
\frac{\partial G\left(T,P\right)}{\partial T} & =-S & \frac{\partial G\left(P,T\right)}{\partial p} & =V\nonumber 
\end{alignat}
In allen Fällen sind die partiellen Ableitungen \emph{einfache} Zustandsgrößen.
Es bestehen Beziehung zwischen partiellen Ableitungen, denn für zweifach
stetig differenzierbare Funktionen $f$ gilt:
\begin{align*}
\dd f & =\frac{\partial f\left(x,y\right)}{\partial x}\dd x+\frac{\partial f\left(x,y\right)}{\partial y}\dd y=A\left(x,y\right)\dd x+B\left(x,y\right)\dd y
\end{align*}
\begin{align}
\frac{\partial^{2}f\left(x,y\right)}{\partial x\partial y}=\frac{\partial^{2}f\left(x,y\right)}{\partial y\partial x}\,\\
\Rightarrow\qquad\fbox{\ensuremath{{\displaystyle \frac{\partial A\left(x,y\right)}{\partial y}=\frac{\partial B\left(x,y\right)}{\partial x}}}}\nonumber 
\end{align}



\section{Maxwellrelationen}

Somit ergeben sich die \emph{Maxwellrelationen}:

\begin{align}
\frac{\partial T\left(S,V\right)}{\partial V} & =-\frac{\partial P\left(S,V\right)}{\partial S} &  & \text{(aus \ensuremath{\dd E})}\nonumber \\
\frac{\partial S\left(T,V\right)}{\partial V} & =\phantom{-}\frac{\partial P\left(T,V\right)}{\partial T} &  & \text{(aus \ensuremath{\dd F})}\\
\frac{\partial T\left(S,P\right)}{\partial P} & =\phantom{-}\frac{\partial V\left(S,P\right)}{\partial S} &  & \text{(aus \ensuremath{\dd H})}\nonumber \\
-\frac{\partial S\left(T,P\right)}{\partial P} & =\phantom{-}\frac{\partial V\left(T,P\right)}{\partial T} &  & \text{(aus \ensuremath{\dd G})}\nonumber 
\end{align}

\begin{itemize}
\item Verallgemeinerung auf weitere äußere Parameter $\left(x=\left(x_{1},\ldots,x_{n}\right)\right)$:\\
Für einen quasi-statischer Prozess gilt:
\begin{align*}
\dd E & =T\dd S-\sum_{i=1}^{n}X_{i}\dd x_{i}
\end{align*}
Beispiel:
\begin{align}
\dd E & =T\dd S-P\dd V-VM\dd B+\mu\dd N
\end{align}
$M$ ist die Magnetisierung, $B$ das Magnetfeld, $\mu$ das chemische
Potential und $N$ die Teilchenzahl.\\
Dies liefert die thermodynamischen Kräfte:
\begin{align}
\frac{\partial E\left(S,V,B,N\right)}{\partial S} & =T & \frac{\partial E\left(S,V,B,N\right)}{\partial V} & =-P\\
\frac{\partial E\left(S,V,B,N\right)}{\partial B} & =-VM & \frac{\partial E\left(S,V,B,N\right)}{\partial N} & =\mu
\end{align}
Dies ergibt sechs Maxwellrelationen und neue thermodynamische Potentiale.
\item Beachte: Ein thermodynamisches Potential enthält die vollständige
thermodynamische Information, denn daraus lassen sich alle anderen
Potentiale und die Zustandsgleichungen berechnen.\\
\emph{Beispiel}: Gegeben sei $E\left(S,V\right)$, so erhält man aus
\begin{align*}
T & =\frac{\partial E\left(S,V\right)}{\partial S}=T\left(S,V\right)
\end{align*}
durch Auflösen $S\left(T,V\right)$. Einsetzen in $E\left(S,V\right)$
führt auf die kalorische Zustandsgleichung $E\left(T,V\right)$. Aus
\begin{align*}
P & =-\frac{\partial E\left(S,V\right)}{\partial V}=P\left(S,V\right)
\end{align*}
erhält man durch Einsetzen von $S\left(T,V\right)$ die thermische
Zustandsgleichung $P\left(T,V\right)$. Ebenso erhält man alle weiteren
thermodynamischen Potentiale.
\item Experimentell wird typischerweise eher der umgekehrter Weg beschritten,
zum Beispiel die Zustandsgleichung $P\left(T,V\right)$ empirisch
bestimmt und Wärmekapazität $C_{V}\left(T,V_{0}\right)$ gemessen.
Durch eine geeignete Integration folgen alle thermodynamischen Potentiale.\\
Beispiel: Wende die partielle Ableitung ${\displaystyle \frac{\partial}{\partial T}}$
auf eine Maxwellrelation an:
\begin{align*}
\frac{\partial C_{V}\left(T,V\right)}{\partial V} & =T\frac{\partial^{2}P\left(T,V\right)}{\partial T^{2}}\\
\Rightarrow\qquad C_{V}\left(T,V\right) & =C_{V}\left(T,V_{0}\right)+T\int_{V_{0}}^{V}\dd V'\frac{\partial^{2}P\left(T,V'\right)}{\partial T^{2}}
\end{align*}
Außerdem erhält man aus einer Maxwellrelation:
\begin{align*}
\dd S & =\frac{\partial S\left(T,V\right)}{\partial T}\dd T+\frac{\partial S\left(T,V\right)}{\partial V}\dd V=\frac{C_{V}}{T}\dd T+\frac{\partial P\left(T,V\right)}{\partial T}\dd V
\end{align*}
Damit folgt:
\begin{align*}
\dd E & =T\dd S-P\dd V=C_{V}\dd T+\left[T\frac{\partial P\left(T,V\right)}{\partial T}-P\right]\dd V
\end{align*}
Zusammenfassung: Alle Größen
\begin{align*}
E\left(S,V\right);\ F\left(T,N\right);\ H\left(S,P\right);\ G\left(T,P\right);\ S\left(E,V\right);\ \Omega\left(E,V\right);\ \left(P\left(T,V\right),\, C_{V}\left(T,V_{0}\right)\right)
\end{align*}
enthalten die vollständige thermodynamische Information.
\end{itemize}

\section{Extremalbedingungen}

Es gilt:
\begin{align}
S & =\text{maximal} &  & \text{bei festen }E,V\label{eq:Smax}\\
F & =\text{minimal} &  & \text{bei festen }T,V\label{eq:Fmin}\\
G & =\text{minimal} &  & \text{bei festen }T,P\label{eq:Gmin}
\end{align}
Die Begründung für Gleichung (\ref{eq:Smax}) folgt aus der statistischen
Physik (,,wahrscheinlichster Makrozustand``).

Um die Gleichung (\ref{eq:Fmin}) zu erhalten, betrachte ein makroskopisches,
aber kleines System $A$ im thermischen Kontakt mit einem Wärmebad
$B$ (System $A$ $\ll$ System $B$). Wärme- und Volumenaustausch
sind möglich.
\begin{align*}
E_{A}\ll E & =E_{A}+E_{B}\\
V_{A}\ll V & =V_{A}+V_{B}\\
T_{A} & =T_{B}=:T\\
P_{A} & =P_{B}
\end{align*}
\begin{align*}
S\left(E_{A}\right) & =S_{A}\left(E_{A},V_{A}\right)+S_{B}\left(E-E_{A},V_{B}\right)\stackrel{!}{=}\text{maximal}\\
 & \stackrel{\text{Taylor}}{\approx}S_{A}+S_{B}\left(E,V_{B}\right)-\frac{\partial S_{B}\left(E,V_{B}\right)}{\partial E}E_{A}=\\
 & =\text{konst.}+S_{A}-\frac{E_{A}}{T}=\text{konst.}-\frac{1}{T}\left(E_{A}-TS_{A}\right)
\end{align*}
Aus $F_{A}=E_{A}-TS_{A}$ folgt, dass $F_{A}$ für das Untersystem
$A$ minimal sein muss, wenn $T$ und $V$ gegeben sind. Analog folgt
Gleichung (\ref{eq:Gmin}).

%DATE: Mi 31.10.12


\chapter{Zustandsänderungen}

Betrachte homogene Systeme mit zwei Zustandsvariablen, zum Beispiel
$\left(S,V\right),\left(T,V\right)$, etc..


\section{Wärmezufuhr}
\begin{itemize}
\item Betrachte eine Wärmezufuhr $\dbar Q_{\text{qs}}$ bei konstantem Druck
$P$ oder konstantem Volumen $V$. Es erfolgt eine Temperaturerhöhung
nach Maßgabe der Wärmekapazitäten:
\begin{align}
C_{V} & =\frac{\dbar Q_{\text{qs}}}{\dd T}\bigg|_{V=\text{konst.}}=T\frac{\partial S\left(T,V\right)}{\partial T}\\
C_{P} & =\frac{\dbar Q_{\text{qs}}}{\dd T}\bigg|_{P=\text{konst.}}=T\frac{\partial S\left(T,P\right)}{\partial T}
\end{align}
Für das ideale Gas gilt:
\begin{align*}
C_{P}-C_{V} & =\nu R
\end{align*}

\item Welche Beziehung gilt zwischen $C_{V}$ und $C_{P}$ allgemein? Aus
$S\left(T,V\right)$ erhält man:\stepcounter{equation}
\begin{align}
\dd S & =\frac{\partial S\left(T,V\right)}{\partial T}\dd T+\frac{\partial S\left(T,V\right)}{\partial V}\dd V\stackrel{\text{Maxwell}}{=}\frac{C_{V}}{T}\dd T+\frac{\partial P\left(T,V\right)}{\partial T}\underbrace{\dd V}_{\text{totales Differential}}\tag{\arabic{chapter}.\arabic{equation}a}
\end{align}
Aus $S\left(T,P\right)$ erhält man:
\begin{align}
\dd S & =\frac{\partial S\left(T,P\right)}{\partial T}\dd T+\frac{\partial S\left(T,P\right)}{\partial P}\dd P\stackrel{\text{Maxwell}}{=}\frac{C_{P}}{T}\dd T-\frac{\partial V\left(T,P\right)}{\partial T}\dd P\tag{\arabic{chapter}.\arabic{equation}b}
\end{align}
\begin{align}
\dd S & =\frac{C_{V}}{T}\dd T+\frac{\partial P\left(T,V\right)}{\partial T}\left(\frac{\partial V\left(T,P\right)}{\partial T}\dd T+\frac{\partial V\left(T,P\right)}{\partial P}\dd P\right)\nonumber \\
\dd S & =\underbrace{\left(\frac{C_{V}}{T}+\frac{\partial P\left(T,V\right)}{\partial T}\frac{\partial V\left(T,P\right)}{\partial T}\right)}_{=\frac{\partial S\left(P,T\right)}{\partial T}=\frac{C_{P}}{T}}\dd T+\underbrace{\frac{\partial P\left(T,V\right)}{\partial T}\frac{\partial V\left(T,P\right)}{\partial P}}_{=\frac{\partial S\left(P,T\right)}{\partial P}}\dd P
\end{align}
Also gilt:
\begin{align}
C_{P} & =C_{V}+T\frac{\partial P\left(T,V\right)}{\partial T}\frac{\partial V\left(T,P\right)}{\partial T}\label{eq:Cp-Bestimmung}
\end{align}

\item Definiere den Wärmeausdehnungskoeffizient:
\begin{align}
\alpha & :=\frac{1}{V}\frac{\partial V\left(T,p\right)}{\partial T}
\end{align}
Das Vorzeichen von $\alpha$ ist nicht festgelegt!\\
Zum Beispiel bei Wasser ist $\alpha<0$ für $0^{\circ}\text{C}<\theta<4^{\circ}\text{C}$.
\item Definiere die isotherme Kompressibilität:
\begin{align}
\kappa_{T} & =-\frac{1}{V}\frac{\partial V\left(P,T\right)}{\partial P}\stackrel{!}{>}0
\end{align}
Dies ist ein \emph{Stabilitätskriterium}, da sonst eine Implosion
erfolgte.
\item Außerdem gilt:
\begin{align*}
\dd f & =\frac{\partial f\left(x,y\right)}{\partial x}\dd x+\frac{\partial f\left(x,y\right)}{\partial y}\dd y\stackrel{!}{=}0\\
\Rightarrow\quad\frac{\partial x\left(y,f\right)}{\partial y} & =-\frac{\ \frac{\partial f\left(x,y\right)}{\partial y}\ }{\ \frac{\partial f\left(x,y\right)}{\partial x}\ }
\end{align*}
Hier speziell:
\begin{align}
\frac{\partial P\left(T,V\right)}{\partial T} & =-\frac{\ \frac{\partial V\left(T,P\right)}{\partial T}\ }{\ \frac{\partial V\left(T,P\right)}{\partial P}\ }=\frac{\alpha}{\kappa_{T}}\label{eq:dP_dT}
\end{align}

\item Einsetzen in (\ref{eq:Cp-Bestimmung}) liefert:
\begin{align}
\fbox{\ensuremath{{\displaystyle C_{P}-C_{V}=\frac{V\cdot T\alpha^{2}}{\kappa_{T}}}}}
\end{align}
Dies sind alles Messgrößen! Es folgt allgemein (für Gase, Flüssigkeiten
und Festkörper):
\begin{align*}
C_{P} & \ge C_{V}
\end{align*}

\item \emph{Beispiel}: ideales Gas
\begin{align}
PV & =\nu RT\nonumber \\
\alpha & =\frac{1}{V}\frac{\partial V\left(T,P\right)}{\partial T}=\frac{1}{T}\nonumber \\
\kappa_{T} & =-\frac{1}{V}\frac{\partial V\left(T,P\right)}{\partial P}=\frac{1}{P}\nonumber \\
\Rightarrow\quad C_{P}-C_{V} & =\nu R
\end{align}
Die Ausdehnungsarbeit ist:
\begin{align*}
\left(C_{P}-C_{V}\right)\dd T & =\nu R\dd T=P\dd V
\end{align*}
Diese Ausdehnungsarbeit ist bei Temperaturerhöhung unter konstantem
Druck zu leisten.
\item \emph{Ausdehnungskoeffizienten} (bei Normaldruck und $T=300\,\text{K}$):


\noindent \begin{center}
\begin{tabular}{c|c|c}
$\vphantom{\Big|}$ & ideales Gas (Edelgas) & Kupfer\tabularnewline
\hline 
\hline 
$\alpha$$\vphantom{\Big|}$ & $3{,}4\cdot10^{-3}\,\frac{1}{\text{K}}$ & $5\cdot10^{-5}\,\frac{1}{\text{K}}$\tabularnewline
\hline 
$\kappa_{T}$$\vphantom{\Big|}$ & $1{,}0\,\frac{1}{\text{bar}}$ & $7{,}4\cdot10^{-7}\,\frac{1}{\text{bar}}$\tabularnewline
\hline 
$c_{P}$$\vphantom{\Big|}$ & $21\,\frac{\text{J}}{\text{K}\cdot\text{mol}}$ & $24\,\frac{\text{J}}{\text{K}\cdot\text{mol}}$\tabularnewline
\hline 
$c_{P}-c_{V}$$\vphantom{\Big|}$ & $8{,}3\,\frac{\text{J}}{\text{K}\cdot\text{mol}}$ & $0{,}73\,\frac{\text{J}}{\text{K}\cdot\text{mol}}$\tabularnewline
\hline 
$\gamma=\frac{c_{P}}{c_{V}}$$\vphantom{\Big|}$ & $\frac{5}{3}=1{,}\overline{6}$ & $1{,}03$\tabularnewline
\end{tabular}
\par\end{center}

\end{itemize}

\section{Expansion (Kompression)}

Betrachte drei Fälle:
\begin{enumerate}[label=\alph*)]
\item Drosselversuch nach Guy-Lussac: \emph{Freie Expansion}\\
\textcolor{green}{TODO: Abb5}
\item \emph{Quasistatische Expansion}:\\
\textcolor{green}{TODO: Abb6}
\item \emph{Joule-Thomson-Expansion}:\\
\textcolor{green}{TODO: Abb7}
\end{enumerate}
Zu Berechnen beziehungsweise zu messen ist die Temperaturänderung.
Folgende Größen sind zu berechnen:
\begin{align*}
\frac{\partial T\left(V,E\right)}{\partial V},\ \frac{\partial T\left(V,S\right)}{\partial V},\;\frac{\partial T\left(V,H\right)}{\partial V}
\end{align*}

\begin{enumerate}[label=\alph*)]
\item \emph{freie Expansion}:

\begin{itemize}
\item Das System ist thermisch isoliert, das heißt es gilt $\dbar Q=0$.\\
Das Öffnen der Drosselklappe ist ohne Arbeitsleistung und das Gas
verrichtet keine Arbeit. Es folgt:
\begin{align}
\dd E & =\dbar Q+\dbar W=0
\end{align}
\emph{Beachte}: Die Zustandsänderung geht über Nichtgleichgewichtszustände,
ist also irreversibel.\\
\emph{Aber}: Die Endzustände sind im Gleichgewicht.
\begin{align}
E\left(T_{a},V_{a}\right) & =E\left(T_{b},V_{b}\right)
\end{align}
Diese Bedingung legt die Temperaturänderung $\Delta T$ bei vorgegebener
Volumenänderung $\Delta V$ fest.
\begin{align*}
\dd E & =T\dd S-P\dd V=T\frac{\partial S\left(T,V\right)}{\partial T}\dd T+T\frac{\partial S\left(T,V\right)}{\partial V}\dd V-P\dd V=\\
 & =C_{V}\dd T+\left(T\frac{\partial P\left(T,V\right)}{\partial T}-P\right)\dd V\stackrel{!}{=}0
\end{align*}
Unter Verwendung von (\ref{eq:dP_dT}) folgt:
\begin{align}
\fbox{\ensuremath{{\displaystyle \frac{\partial T\left(V,E\right)}{\partial V}=\frac{1}{C_{V}}\left(P-T\frac{\partial P\left(T,V\right)}{\partial T}\right)}}}
\end{align}

\item In konkreten Beispielen:
\begin{align*}
\frac{\partial T\left(E,v\right)}{\partial v} & =\nu\frac{\partial T\left(E,V\right)}{\partial V}=\begin{cases}
0 & \text{(ideales Gas)}\\
-\frac{a}{C_{V}v^{2}} & \text{(van der Waals)}
\end{cases}
\end{align*}
Im idealen Gas gilt:
\begin{align*}
E & =\sum_{i=1}^{N}\varepsilon_{\text{kin},i}
\end{align*}
Dies ist unabhängig von $V$!\\
Im van-der-Waals-Gas ergibt sich eine \emph{Abkühlung}.\\
Qualitativ ist der Grund, dass die Vergrößerung von $V$ eine Verringerung
der attraktiven Kräfte auf Kosten der kinetischen Energie bewirkt.
\end{itemize}
\item \emph{Quasistatische, adiabatische Expansion}:

\begin{itemize}
\item Das System sei thermisch isoliert, also $\dbar Q=0$. Das Gas verrichtet
Arbeit bei quasistatischem Herausziehen des Stempels.
\begin{align*}
\dbar W_{\text{qs}} & =-P\dd V\not=0
\end{align*}

\item Nach dem 2. Hauptsatz gilt:
\begin{align}
\dd S & =\frac{\dbar Q_{\text{qs}}}{T}=0
\end{align}
\begin{align*}
\Rightarrow\quad S\left(T_{a},V_{a}\right) & =S\left(V_{b},T_{b}\right)
\end{align*}
\begin{align*}
\dd S & =\frac{\partial S\left(T,V\right)}{\partial T}\dd T+\frac{\partial S\left(T,V\right)}{\partial V}\dd V=\frac{C_{V}}{T}\dd T+\frac{\partial P\left(T,V\right)}{\partial T}\dd V\stackrel{!}{=}0
\end{align*}
\begin{align}
\fbox{\ensuremath{{\displaystyle \frac{\partial T\left(V,S\right)}{\partial V}=-\frac{T}{C_{V}}\frac{\partial P\left(T,V\right)}{\partial T}=-\frac{T\alpha}{C_{V}\kappa_{T}}}}}
\end{align}
Für Gase gilt immer $\alpha>0$. Also kühlen sich Gase bei quasistatischer
adiabatischer Expansion ab. Dies ermöglicht das Prinzip (den Kreisprozess)
einer \emph{Kältemaschine}:

\begin{enumerate}[label=\arabic*.]
\item Gas bei Temperatur $T_{1}$ (Umgebungstemperatur)\\
$\rightarrow$ quasistatische, adiabatische Kompression\\
$\quad\rightarrow$ Erwärmung auf $T_{2}$
\item thermischer Kontakt mit Umgebung\\
$\rightarrow$ Abkühlung auf $T_{1}$
\item Quasistatische, adiabatische Expansion\\
$\rightarrow$ Temperatur $T_{3}<T_{1}$
\end{enumerate}
\end{itemize}
\item \emph{Joule-Thomson-Expansion}: 

\begin{itemize}
\item Gas mit Druck $P_{a}$ wird durch einen porösen Stopfen gepresst.\\
Das System sei thermisch isoliert, das heißt $\dbar Q=0$.\\
Hinter dem Stopfen herrscht ein niedrigerer Druck $P_{b}$.
\item Beachte: Das System ist insgesamt nicht im Gleichgewicht.\\
Annahme: Separat links und rechts vom System bestehen ungefähre Gleichgewichtszustände.
\item Betrachte zum Beispiel $1\,\text{mol}$ Gas mit Volumen $V_{a}$ bei
Druck $P_{a}$.\\
Die Arbeit zum Weiterschieben des Gases vor dem Stopfen ist $P_{a}V_{a}$.\\
Die Arbeit zum Weiterschieben des Gases nach dem Stopfen ist $P_{b}V_{b}$.\\
Insgesamt wird die Arbeit
\begin{align}
\Delta W & =P_{a}V_{a}-P_{b}V_{b}(4.17)
\end{align}
dem Gas zugeführt.\\
Aus dem 1. Hauptsatz folgt:
\begin{align*}
\Delta E & =E_{b}-E_{a}=\underbrace{\Delta Q}_{=0}+\Delta W
\end{align*}
\begin{align}
E_{b}+P_{b}V_{b} & =E_{a}+P_{a}V_{a}\nonumber \\
H\left(T_{a},P_{a}\right) & =H\left(T_{b},P_{b}\right)
\end{align}
Die Enthalpie ist konstant!
\item Nun gilt:
\begin{align*}
\dd H & =T\dd S+V\dd P=T\left(\frac{\partial S\left(T,P\right)}{\partial T}\dd T+\frac{\partial S\left(T,P\right)}{\partial P}\dd P\right)+V\dd P=\\
 & =C_{P}\dd T+\left(V-T\frac{\partial V\left(T,P\right)}{\partial T}\right)\dd P=\\
 & =C_{P}\dd T+\left(V-VT\alpha\right)\dd P\stackrel{!}{=}0
\end{align*}

\item Definiere den Joule-Thomson-Koeffizienten:
\begin{align}
\mu_{\text{JT}} & :=\frac{\partial T\left(P,H\right)}{\partial P}=\frac{V}{C_{P}}\left(T\alpha-1\right)
\end{align}

\item Beachte: Für ideales Gas gilt $\alpha=\frac{1}{T}$, also $\mu_{\text{JT}}=0$.\\
Für ein reales Gas gilt im Allgemeinen $\mu_{\text{JT}}\not=0$.\\
Zum Beispiel für Stickstoff gilt bei Raumtemperatur:
\begin{align*}
\mu_{\text{JT}} & >0
\end{align*}
Also erfolgt eine Abkühlung bei einer Entspannung. (Joule-Thomson-Kühlmaschine!)
\end{itemize}
\end{enumerate}
%DATE: Do 8.11.12 (doppelt so lang)


\chapter{Wärmekraftmaschinen}

\emph{Wärmekraftmaschinen} dienen zur Umwandlung von Wärme in Arbeit.
Bei einem \emph{Kreisprozess} kehrt das System zyklisch in den Ausgangszustand
zurück.


\section{Die Unmöglichkeit eines Perpetuum Mobile zweiter Art}

\textcolor{green}{TODO: Abb8}

\emph{Perpetuum Mobile zweiter Art}: Die Maschine nimmt nur die Wärme
$q$ auf und verwandelt diese in Arbeit $w$.
\begin{itemize}
\item Nach dem 1. Hauptsatz gilt:
\begin{align*}
\Delta E_{M} & =q-w & \Rightarrow\qquad w & =q
\end{align*}

\item Unmöglich nach 2. Hauptsatz!\\
Schließe das System mit Hilfe eines Energiespeichers $A$ für mechanisch
verrichtete Arbeit, zum Beispiel einer Feder.\\
Aus dem 2. Hauptsatz folgt $\Delta S\ge0$ für abgeschlossene Systeme.
\begin{align*}
\Delta S & =\Delta S_{R}+\Delta S_{M}+\Delta S_{A}\stackrel{!}{\ge}0
\end{align*}
Nach statistischer Physik gilt:
\begin{align*}
S & =k_{B}\ln\Omega
\end{align*}
\begin{align*}
S_{R} & \approx10^{24}k_{B} & S_{A} & \approx k_{B}
\end{align*}
Also ist $S_{A}$ vernachlässigbar, da die Feder nur einen Freiheitsgrad
hat.
\begin{align*}
\Delta S & =\Delta S_{R}+\Delta S_{M}+\Delta S_{A}=-\frac{q}{T}
\end{align*}
Nach einem Zyklus gilt:
\begin{align*}
\left.\begin{array}{c}
\Delta S_{M}=0\\
\Delta S_{R}=-\frac{q}{T}
\end{array}\right\} \Rightarrow\qquad & \Delta S=-\frac{q}{T}<0
\end{align*}
\emph{Die Entropie nimmt ab!}
\end{itemize}
Gedankenexperiment:
\begin{itemize}
\item Betrachte einen Kasten mit idealem Gas in Kontakt mit einem Reservoir.
\item Der Kasten wird thermisch isoliert. \emph{Warte bis Teilchen zufällig
alle in einer Hälfte des Kastens sind}.
\item Seitenwand einschieben: $V_{2}<V_{1}$
\item Führe adiabatische, quasistatische Expansion durch, die die Arbeit
$W$ abgibt und das Gas abkühlt.
\item Erwärmung des Gases durch Kontakt mit dem Reservoir:
\begin{align*}
\Delta E_{M} & =0
\end{align*}

\end{itemize}
Betrachte $10^{24}$ Teilchen mit $1\%$ mehr links. Das sind $10^{10}$
Standardabweichungen! Tatsächliche statistische Fluktuationen:
\begin{align*}
\frac{\Delta N}{N} & \approx10^{-12}
\end{align*}


Maxwellscher Dämon:

\textcolor{green}{TODO: Abb9}

Voraussetzung: Die Arbeit zur Bedienung der Klappe ist $w\ll k_{B}T$.
\begin{itemize}
\item Beachte:

\begin{enumerate}
\item $w\to q$ ist unbeschränkt möglich.\\
$q\to w$ ist nur nach 2. Hauptsatz möglich.\\
$\rightarrow$ Wärme (ungeordnete Bewegung) ist minderwertig gegen
Arbeit.
\item $\Delta S\ge0$ $\Leftrightarrow$ $\frac{\dd S}{\dd t}>0$ $\rightarrow$
Zeitrichtung !?!
\end{enumerate}
\end{itemize}

\section{Wirkungsgrad einer Wärmekraftmaschine}

\emph{Carnot-Prozess}:

\textcolor{green}{TODO: Abb10}

Hier ist $T_{1}>T_{2}$. Betrachte wieder eine zyklische Maschine
(z. B. Wärmepumpe).

Was ist die ideale Maschine?

Die Maschine nutzt den Wärmefluss vom warmen zum kalten Reservoir
und leistet dabei Arbeit.
\begin{align*}
\Delta E_{M} & =0
\end{align*}

\begin{itemize}
\item 1. Hauptsatz:
\begin{align*}
\Delta E_{M} & =0=q_{1}-q_{2}-w=0\\
w & =q_{1}-q_{2}
\end{align*}

\item 2. Hauptsatz für das Gesamtsystem aus Reservoir 1, Reservoir 2, Maschine
und Speicher:
\begin{align*}
\Delta S & =\underbrace{\Delta S_{R_{1}}}_{=-\frac{q_{1}}{T_{1}}}+\underbrace{\Delta S_{R_{2}}}_{=\frac{q_{2}}{T_{2}}}+\underbrace{\Delta S_{M}}_{=0}+\underbrace{\Delta S_{A}}_{\approx0}\ge0\\
\Delta S & =\frac{q_{2}}{T_{2}}-\frac{q_{1}}{T_{1}}\ge0
\end{align*}
Die untere Grenze ist:
\begin{align*}
q_{2} & \ge\frac{T_{2}}{T_{1}}q_{1}
\end{align*}

\item \emph{Wirkungsgrad}:
\begin{align*}
\eta & =\frac{\text{erzeugte Arbeit}}{\text{aufgewendete Wärme}}=\frac{w}{q_{1}}=1-\frac{q_{2}}{q_{1}}
\end{align*}
\begin{align}
\eta & \le1-\frac{T_{2}}{T_{1}}
\end{align}

\item Wirkungsgrad einer idealen Wärmekraftmaschine:
\begin{align}
\fbox{\ensuremath{{\displaystyle \eta_{\text{ideal}}=1-\frac{T_{2}}{T_{1}}}}}
\end{align}

\end{itemize}

\section{Der Carnotprozess}

Dies ist ein Beispiel für einen Kreisprozess mit idealem Wirkungsgrad.
Voraussetzung sind quasistatische Zustandsänderungen.

\textcolor{green}{TODO: Abb11}
\begin{enumerate}[label=\roman*)]
\item Isotherme Expansion: $\left(T_{a},S_{a}\right)\to\left(T_{a},S_{b}\right)$
\item Adiabatische Expansion: $\left(T_{a},S_{b}\right)\to\left(T_{b},S_{b}\right)$
\item Isotherme Kompression: $\left(T_{b},S_{b}\right)\to\left(T_{b},S_{a}\right)$
\item Adiabatische Kompression: $\left(T_{b},S_{a}\right)\to\left(T_{a},S_{a}\right)$
\end{enumerate}
\begin{align}
\Delta S_{\text{Gas}} & =\frac{q_{a}}{T_{a}}-\frac{q_{b}}{T_{b}}\stackrel{!}{=}0\\
\Delta E_{\text{Gas}} & =w+q_{b}-q_{a}\stackrel{!}{=}0\quad\Rightarrow\quad w=q_{a}-q_{b}\\
w & =\oint P\dd V=\text{„eingeschlossene Fläche“, da q.s.}\\
\eta_{\text{Carnot}} & =\frac{w}{q_{a}}=1-\frac{T_{b}}{T_{a}}=\eta_{\text{ideal}}
\end{align}


Gesamte Entropieänderung:
\begin{align}
\Delta S & =\Delta S_{R_{1}}+\Delta S_{R_{2}}+\Delta S_{\text{Gas}}+\Delta S_{S}=\nonumber \\
 & =-\frac{q_{a}}{T_{a}}+\frac{q_{b}}{T_{b}}+\frac{q_{a}}{T_{a}}-\frac{q_{b}}{T_{b}}+0=0
\end{align}
Der Vorgang ist umkehrbar und man erhält Wärmepumpe oder Kühlschrank.

\textcolor{green}{TODO: Abb12}

1. Hauptsatz:
\begin{align*}
q_{2}+w & =q_{1}
\end{align*}
2. Hauptsatz:
\begin{align*}
\Delta S & =\Delta S_{R_{1}}+\Delta S_{R_{2}}+\Delta S_{M}+\Delta S_{A}=\frac{q_{1}}{T_{1}}-\frac{q_{2}}{T_{2}}\ge0
\end{align*}
Wirkungsgrad:
\begin{align}
\eta & \le\eta_{\text{ideal}}=\begin{cases}
{\displaystyle \frac{q_{1}}{w}=\frac{T_{1}}{T_{1}-T_{2}}} & \qquad\text{Wärmepumpe}\\
{\displaystyle \frac{q_{2}}{w}=\frac{T_{2}}{T_{1}-T_{2}}} & \qquad\text{Kühlschrank}
\end{cases}
\end{align}
Dies ist wichtig für Heizungskonzepte, zum Beispiel:
\begin{align*}
T_{1} & =813\,\text{K} & T_{2} & =313\,\text{K}
\end{align*}
\begin{align*}
\eta_{\text{ideal}} & \approx62\% & \eta_{\text{typisch}} & \approx45\%
\end{align*}
Wärmepumpe:
\begin{align*}
T_{2} & =273\,\text{K} & T_{1} & =293\,\text{K}
\end{align*}
\begin{align*}
\eta_{\text{ideal}} & \approx15
\end{align*}
Mit Brauchwasser $T_{1}=323\,\text{K}$:
\begin{align*}
\eta_{\text{ideal}} & \approx6
\end{align*}
\begin{align*}
\eta_{\text{typisch}} & \approx3
\end{align*}



\chapter[Chemisches Potential]{Austausch von Teilchen und chemisches Potential}


\section{Definition \textmd{(Großkanonisches Potential)}}
\begin{itemize}
\item Bisher war $N$ fix, als $\dd N=0$.\\
Betrachte jetzt Systeme mit Teilchenaustausch.
\item Das chemische Potential $\mu$ ist die verallgemeinerte Kraft zur
Teilchenzahl $N$.
\begin{align*}
\dbar W_{\text{qs}} & =-P\dd V & \dbar Q_{\text{qs}} & =T\dd S
\end{align*}
\begin{align}
\fbox{\ensuremath{{\displaystyle \dd E=T\dd S-P\dd V+\mu\dd N}}}
\end{align}
\begin{align*}
E & =E\left(S,V,N\right)
\end{align*}
\begin{align}
\mu & =-T\frac{\partial S\left(N,E,V\right)}{\partial N}=\frac{\partial E\left(N,S,V\right)}{\partial N}
\end{align}

\item Das heißt $\mu$ ist die Energie, die aufgewendet werden muss, um
dem System ein weiteres Teilchen hinzuzufügen.
\item Beachte: Der Nullpunkt von $\mu$ ist willkürlich.
\end{itemize}
Analog gilt für alle weiteren thermodynamischen Potentiale:
\begin{align}
\left.\begin{array}{c}
\dd E=T\dd S-P\dd V+\mu\dd N\\
\dd H=T\dd S+V\dd P+\mu\dd N\\
\dd F=-S\dd T-P\dd V+\mu\dd N\\
\dd G=-S\dd T+V\dd P+\mu\dd N
\end{array}\right\}  & \text{Nicht zusätzliche natürliche Variable.}
\end{align}
Eine Legendre Transformation liefert das \emph{großkanonische Potential}:
\begin{align}
J & =J\left(T,V,\mu\right)\\
J & =E-TS-\mu N=F-\mu N\\
\dd J & =-S\dd T-P\dd V-N\dd\mu
\end{align}



\section{Duhem-Gibbs-Relation}
\begin{itemize}
\item Die freie Enthalpie $G$ ist eine extensive Größe, womit folgt:
\begin{align}
G\left(T,P,N\right) & =Ng\left(T,P\right)
\end{align}
Wegen
\begin{align*}
\mu & =\frac{\partial G\left(N,T,P\right)}{\partial N}=g\left(T,P\right)=\frac{G\left(T,P,N\right)}{N}
\end{align*}
gilt:
\begin{align}
G\left(T,P,N\right) & =E-TS+PV=N\mu\left(T,P\right)\\
\dd G & =N\dd\mu+\mu\dd N=-S\dd T+V\dd P+\mu\dd N
\end{align}
Es folgt die \emph{Duhem-Gibbs-Relation}:
\begin{align}
\fbox{\ensuremath{{\displaystyle \dd\mu=-\frac{S}{N}\dd T+\frac{V}{N}\dd P}}}\label{eq:Duhem-Gibbs}
\end{align}

\item Die bevorzugten natürlichen Variablen von $\mu$ sind $T$ und $P$.
\begin{align*}
s & =\frac{S}{N} & v & =\frac{V}{N}
\end{align*}

\end{itemize}

\section{Gleichgewichtsbedingungen}
\begin{itemize}
\item Bei Temperatur $T$ und Druck $P$ gilt für das Gleichgewicht:
\begin{align}
G\left(T,P,N\right) & =N\cdot\mu\left(T,P\right)=\text{minimal}
\end{align}

\item Für fixes $N$ folgt:
\begin{align}
\fbox{\ensuremath{{\displaystyle \mu\left(T,P\right)\stackrel{!}{=}\text{minimal im Gleichgewicht}}}}
\end{align}

\item Wichtige Anwendung: Gleichgewicht zwischen verschiedenen Phasen!\\
Beispiel:\\
\textcolor{green}{TODO: Abb13}\\
Gleichgewicht zwischen Eis und Wasser\\
2 homogene Systeme: $A$: Eis, $B$: Wasser\\
Austausch von Wärme und Teilchen (Volumen). Für ein abgeschlossenes
Gesamtsystem gilt:
\begin{align*}
S\left(E_{A},N_{A}\right) & =S_{A}\left(E_{A},V_{A},N_{A}\right)+S_{B}\left(E-E_{A},V_{B},N-N_{A}\right)\stackrel{!}{=}\text{maximal}
\end{align*}
\begin{align*}
\Rightarrow\quad\frac{\partial S}{\partial E_{A}} & =0 & \Leftrightarrow\quad\frac{1}{T_{A}} & =\frac{1}{T_{B}}\\
\frac{\partial S}{\partial N_{A}} & =0 & \Leftrightarrow\quad\frac{\mu_{A}}{T_{A}} & =\frac{\mu_{B}}{T_{B}}
\end{align*}
Es folgt:
\begin{align*}
T_{A} & =T_{B} & \mu_{A} & =\mu_{B}
\end{align*}
Mit Volumenaustausch folgt zusätzlich $P_{A}=P_{B}$. Dies sind die
Gleichgewichtsbedingungen.
\end{itemize}

\section{Phasendiagramm}

\textcolor{green}{TODO: Abb14}

Phasen- oder Zustandsdiagramm von Wasser.

Ohne Beschränkung der Allgemeinheit verwende $T,P,N$ als Zustandsvariablen.
\begin{align*}
\mu\left(T,P\right) & =\frac{G\left(T,P,N\right)}{N}
\end{align*}
Also bestimmen $T$ und $P$ das Gleichgewicht (legen $G$ pro Teilchenzahl
fest).
\begin{itemize}
\item Betrachte das Gleichgewicht zwischen zwei Phasen unter Wärme- und
Teilchenaustausch.
\begin{align}
\mu_{A}\left(T,P\right) & \stackrel{!}{=}\mu_{B}\left(T,P\right)
\end{align}
Dies definiert Phasenkoexistenzkurven:
\begin{align*}
P & =P_{d}\left(T\right) &  & \text{Dampfdruckkurve}\\
T & =T_{s}\left(P\right) &  & \text{Siedetemperaturkurve}
\end{align*}
Diese Kurven im $P$-$T$-Diagramm trennen verschiedene Phasen.
\end{itemize}
\emph{Beachte}:
\begin{itemize}
\item $\mu_{A}\to\mu_{B}$ ändert sich \emph{stetig} bei Phasenübergang
$A\to B$.
\item Für alle Stoffe endet die Übergangskurve zwischen flüssig/gasförmig
in einem kritischen Punkt.
\item Allgemein: Sind drei Phasen im Gleichgewicht, so ist dies ein \emph{Tripelpunkt}.
\begin{align}
\mu_{A}\left(T_{t},P_{t}\right) & =\mu_{B}\left(T_{t},P_{t}\right)=\mu_{C}\left(T_{t},P_{t}\right)
\end{align}

\end{itemize}

\section{Clausius-Clapeyron-Gleichung}
\begin{itemize}
\item Gleichgewicht zwischen 2 Phasen $A$ und $B$. ($P=P_{d}\left(T\right)$)
\begin{align*}
\mu_{A}\left(T,P_{d}\left(T\right)\right) & =\mu_{B}\left(T,P_{d}\left(T\right)\right)\qquad/\frac{\dd}{\dd T}
\end{align*}
„Dampfdruckkurve“:
\begin{align}
\frac{\partial\mu_{A}\left(T,P\right)}{\partial T}+\frac{\partial\mu_{A}\left(T,P\right)}{\partial P}\cdot\frac{\dd P_{d}\left(T\right)}{\dd T} & =\frac{\partial\mu_{B}\left(T,P\right)}{\partial T}+\frac{\partial\mu_{B}\left(T,P\right)}{\partial P}\cdot\frac{\dd P_{d}\left(T\right)}{\dd T}
\end{align}
Mit
\begin{align*}
N\dd\mu & =-S\dd T+V\dd P
\end{align*}
folgt daraus:
\begin{align}
\left(v_{A}-v_{B}\right)\frac{\dd P_{d}}{\dd T} & =s_{A}-s_{B}\label{eq:dPd}
\end{align}

\item Mit $\dd H=T\dd S+V\dd P$ gilt für quasistatische Zustandsänderungen
bei festem Druck $P$:
\begin{align}
\Delta s & =\frac{\Delta h}{T}=\frac{q}{T}\label{eq:Verdampfungsenthalpie}
\end{align}
Dabei ist $q=\Delta h$ die latente Wärme oder (molare) \emph{Verdampfungsenthalpie}.\\
Aus (\ref{eq:dPd}) und (\ref{eq:Verdampfungsenthalpie}) folgt die
\emph{Clausius-Clapeyron-Beziehung}:
\begin{align}
\fbox{\ensuremath{{\displaystyle \frac{\dd P_{d}}{\dd T}=\frac{\Delta h\left(T\right)}{T\left(v_{A}-v_{B}\right)}=\frac{q\left(T\right)}{T\left(v_{A}-v_{B}\right)}}}}
\end{align}

\item Beispiel: Dampfdruck über Festkörpern und Flüssigkeiten.

\begin{itemize}
\item Vernachlässige $v_{\text{fl}},v_{\text{fest}}\ll v_{\text{Gas}}$.
\item Vernachlässige Temperaturabhängigkeit der molaren Verdampfungsenthalpie
$q\left(T\right)=q$.
\item Behandle das Gas wie ein ideales Gas mit $Pv_{\text{Gas}}=k_{B}T$.
$v_{B}=v_{\text{flüssig}}\ll v_{\text{Gas}}=v_{A}$
\begin{align*}
\frac{\dd P}{\dd T} & \approx\frac{q}{T}\cdot\frac{1}{v_{\text{Gas}}}=\frac{q}{T}\frac{P}{k_{B}T}=\frac{q}{k_{B}}\frac{P}{T^{2}}\\
\Rightarrow\qquad\frac{\dd P}{P} & =\frac{q}{k_{B}}\frac{\dd T}{T^{2}}\\
\Rightarrow\qquad\ln\left(\frac{P}{P_{0}}\right) & =-\frac{q}{k_{B}T}
\end{align*}
\begin{align}
\Rightarrow\qquad\fbox{\ensuremath{{\displaystyle P_{d}\left(T\right)=P_{0}e^{-\frac{q}{k_{B}T}}}}}
\end{align}
Dies schränkt die Materialauswahl bei der Erzeugung von Ultrahochvakuum
ein, zum Beispiel sollte man kein Blei oder Zinn verwenden.
\end{itemize}
\end{itemize}
%DATE: Mi 14.11.12


\section{Osmotischer Druck}

Im Lösungsmittel seien $N_{C}$ Teilchen gelöst, $n_{C}=\frac{N_{C}}{V}$.
Für das Lösungsmittel gelte $n=\frac{N}{V}$.

\textcolor{green}{TODO: abb15; rot: semipermeable Membran (undurchlässig
z.B. für Salz); grün: Salz}

\textcolor{green}{TODO: Abb16; $P_{r},P_{l}$ so, dass die Kolben
ruhen}
\begin{itemize}
\item Betrachte eine semipermeable Membran.

\begin{itemize}
\item durchlässig für Lösungsmittelmoleküle
\item undurchlässig für Moleküle des gelösten Stoffs\\
$\rightarrow$ Der Druckunterschied ist der \emph{osmotische} Druck
$P_{\text{osm}}$.
\end{itemize}
\item Konzentration: ${\displaystyle c=\frac{N_{C}}{N}=\frac{n_{C}}{n}}$
\item $\Omega_{\text{LM}}$: Zustandssumme des Lösungsmittels\\
$\Omega_{C}$: Zustandssumme der gelösten Moleküle
\end{itemize}
Annahme: Die gelösten Teilchen bewegen sich im Volumen unabhängig
von einander (keine Wechselwirkung), so wie im idealen Gas. Analog
zum idealen Gas gilt $\Omega_{C}\left(V\right)\sim V^{N_{C}}$.\\
Entropie:
\begin{align*}
S_{L} & =k_{B}\ln\left(\Omega_{LM}\Omega_{C}\right)=S_{LM}+N_{C}k_{B}\ln V
\end{align*}
Druck der Lösung:
\begin{align*}
P_{L} & =T\frac{\partial S_{L}\left(V,E\right)}{\partial V}=T\frac{\partial S_{\text{LM}}\left(V,E\right)}{\partial V}+\frac{N_{C}k_{B}T}{V}=P_{\text{LM}}+P_{\text{osm}}
\end{align*}
Dies liefert das \emph{van't Hoffsche Gesetz}:
\begin{align}
\fbox{\ensuremath{{\displaystyle P_{\text{osm}}=n_{C}k_{B}T}}}
\end{align}
Auf beiden Seiten der semipermeablen Membran muss im Gleichgewicht
derselbe Druck $P_{\text{LM}}$ für das Lösungsmittel herrschen.
\begin{itemize}
\item Unmittelbar über der Membran: $P_{1}=P_{\text{LM}}+P_{\text{osm}}$
\item Unmittelbar unter der Membran: $P_{2}=P_{\text{LM}}$
\item In der Säule:\\
Im Schwerefeld ist $P_{\text{osm}}\sim h$. Dabei ist $h$ die Höhe
der Flüssigkeitssäule.\\
Annahme: Massendichte $\varrho$ der Lösung und des Lösungsmittels
sind ungefähr gleich:
\begin{align*}
P_{\text{osm}} & =P_{1}-P_{2}=\varrho gh
\end{align*}
Dies ist in der Biologie wichtig.
\end{itemize}

\section{Siedepunkterhöhung und Gefrierpunkterniedrigung}
\begin{itemize}
\item Die Lösung eines Stoffs (effektive Druckänderung) führt zur Verschiebung
der Übergangstemperaturen. Der Gesamtdruck ist der Druck der Lösung:
\begin{align*}
P_{L} & =P
\end{align*}
Der osmotische Druck ist:
\begin{align*}
P_{\text{osm}} & =\Delta P=n_{C}k_{B}T
\end{align*}
Der Druck des Lösungsmittels ist:
\begin{align*}
P_{\text{LM}} & =P-\Delta P
\end{align*}
Das Lösungsmittel hat die Konzentration $c$.\\
Das Chemisches Potential des Lösungsmittels für $\Delta P\ll P$ und
$c\ll1$ ist:
\begin{align}
\mu_{C}\left(T,P_{\text{LM}}\right) & =\mu\left(T,P-\Delta P\right)\approx\mu\left(T,P\right)-\frac{\partial\mu\left(P,T\right)}{\partial P}\Delta P=\mu\left(T,P\right)-ck_{B}T
\end{align}

\item Betrachte den Übergang zwischen Phase $A$ (gasförmig oder fest) und
Phase $B$ (flüssig).\\
Für $c=0$ gilt:
\begin{align}
\mu_{A}\left(T_{s},P\right) & =\mu_{B}\left(T_{s},P\right)
\end{align}
Für $c>0$ ist das Gleichgewicht bei einer anderen Temperatur $T_{s}+\Delta T_{s}$
für festen Druck~$P$.
\begin{align}
\mu_{A}\left(T_{s}+\Delta T_{s},P\right) & =\mu_{c,B}\left(T_{s}+\Delta T_{s},P\right)
\end{align}
Entwickle beide Seiten, wobei
\begin{align*}
\dd\mu & =-s\dd T+v\dd P
\end{align*}
gilt:
\begin{align}
\mu_{A}\left(T_{s}+\Delta T_{s},P\right) & \approx\mu_{A}\left(T_{s},P\right)-s_{A}\Delta T_{s}\\
\mu_{c,B}\left(T_{s}+\Delta T_{s},P\right) & \approx\mu_{c,B}\left(T_{s},P\right)-s_{B}\Delta T_{s}-ck_{B}\left(T_{s}+\Delta T_{s}\right)\approx\nonumber \\
 & \stackrel{ck_{B}\Delta T_{s}\approx0}{\approx}\mu_{B}\left(T_{s},P\right)-s_{B}\Delta T_{s}-ck_{B}T_{s}
\end{align}
Hier wurde $ck_{B}\Delta T_{s}$ vernachlässigt, weil es quadratisch
in kleinen Größen ist. Es folgt:
\begin{align}
\fbox{\ensuremath{{\displaystyle \Delta T_{s}=c\cdot\frac{k_{B}T_{s}}{s_{A}-s_{B}}}}}
\end{align}
Der Entropieunterschied ist durch die Umwandlungsenthalpie gegeben:
\begin{align*}
q & =h_{A}-h_{B} &  & \text{(für Übergang \ensuremath{A\to B})}
\end{align*}
Mit $s_{A}-s_{B}=\frac{q}{T_{s}}$ folgt damit die Änderung der Übergangstemperatur:
\begin{align}
\fbox{\ensuremath{{\displaystyle \Delta T_{s}=c\frac{k_{B}T_{s}^{2}}{q}}}}
\end{align}

\end{itemize}
Anmerkungen:
\begin{itemize}
\item $s_{\text{fest}}<s_{\text{flüssig}}$ und somit $\Delta T<0$, also
Gefrierpunkterniedrigung.\\
$s_{\text{Gas}}>s_{\text{flüssig}}$ und somit $\Delta T>0$, also
Siedepunktserhöhung.
\item Aus Messung von $\frac{\Delta T_{s}}{c}$ erhält man die latente Wärme.
(,,Kryoskopie`` $\to$ physikalische Chemie)
\item Ein Beispiel ist das Enteisen von Gehwegen durch Salz.

\begin{itemize}
\item molare Übergangsenthalpie $q_{\text{Wasser}\to\text{Eis}}=-6\,\frac{\text{kJ}}{\text{mol}}$
\item bei $T_{s}=273\,\text{K}$
\item Es sei $c=10\%$, womit folgt:
\begin{align*}
\Delta T_{s} & =c\frac{RT_{s}^{2}}{q}\approx-0,1\cdot\frac{8,3\cdot273^{2}}{6000}\,\text{K}=-10\,\text{K}
\end{align*}

\end{itemize}
\end{itemize}

\section{Das Massenwirkungsgesetz}

Betrachte das Reaktionsgleichgewicht von $m$ verschiedenen Stoffen
$X_{i}$.\\
$N_{i}$ ist die Anzahl der Moleküle der Stoffsorte $X_{i}$ und $\nu_{i}\in\mathbb{Z}$
der \emph{stöchiometrischer Koeffizient}.
\begin{align}
\sum_{i=1}^{m}\nu_{i}N_{i} & =0
\end{align}
Beispiel:
\begin{align*}
\text{N}_{2}+3\text{H}_{2} & \leftrightarrows2\text{NH}_{3}
\end{align*}
Das heißt $\nu_{1}=1$, $\nu_{2}=3$, $\nu_{3}=-2$.

Annahme: Mischung idealer Gase/idealer Lösungen.
\begin{itemize}
\item Sei $\dd k$ die Zahl der molekularen Reaktionsschritte, so ändert
sich die Anzahl der $X_{i}$-Moleküle um $\dd N_{i}$ mit: 
\begin{align}
\dd N_{i} & =\nu_{i}\dd k &  & \left(i\in\left\{ 1,2,\ldots,m\right\} \right)
\end{align}

\item Das Gesamtsystem aus $m$ Stoffen sei abgeschlossen und $T,P$ konstant,
also ist das Gleichgewicht durch ein Minimum von $G$ gegeben.
\begin{align}
G\left(T,P,N_{1},\ldots,N_{m}\right) & =\sum_{i=1}^{m}N_{i}\mu_{i}\left(T,P\right)\label{eq:G_Ni}
\end{align}
Reaktionen laufen solange ab, bis gilt:
\begin{align}
\frac{\dd G}{\dd k} & =\sum_{i=1}^{m}\frac{\partial G\left(T,P,N_{1},\ldots,N_{m}\right)}{\partial N_{i}}\nu_{i}\stackrel{!}{=}0\label{eq:dG_dk}
\end{align}
Aus Gleichungen (\ref{eq:G_Ni}) und (\ref{eq:dG_dk}) folgt:
\begin{align}
\fbox{\ensuremath{{\displaystyle \sum_{i=1}^{m}\mu_{i}\nu_{i}=0}}}\label{eq:Summe_mu_i-nu_i}
\end{align}

\item Mikroskopisches Bild: Behandlung wie ideales Gas $\mu\left(T,\frac{V}{N_{i}}\right)$
mit ${\displaystyle \frac{N_{i}}{V}=c_{i}\frac{N}{V}=c_{i}\frac{P}{k_{B}T}}$.
\end{itemize}

\subsubsection*{Einschub \textmd{(chemisches Potential des idealen Gases)}}
\begin{itemize}
\item Gasgleichung (\ref{eq:Zustandsgleichung-ideals_Gas}):
\begin{align*}
PV & =Nk_{B}T
\end{align*}

\item Volumenabhängigkeit der Energie (\ref{eq:E_E(T)}):
\begin{align*}
E & =E\left(T,N\right)=N\cdot c\left(T\right)
\end{align*}
\begin{align*}
\frac{\partial E\left(V,T\right)}{\partial V} & =0
\end{align*}

\item Entropie (\ref{eq:Entropie-Integral}):
\begin{align*}
S\left(T,V,N\right) & =N\left(\int\frac{c_{V}\left(T'\right)}{T'}\dd T'+k_{B}\ln\left(\frac{V}{N}\right)+\text{konst.}\right)
\end{align*}

\end{itemize}
Aus
\begin{align*}
G\left(T,P,N\right) & =E-TS+PV=N\mu\left(T,P\right)
\end{align*}
folgt:
\begin{align*}
\mu & =\frac{E}{N}-\frac{TS}{N}+\frac{PV}{N}=\underbrace{c_{V}\left(T\right)}_{=\frac{3}{2}k_{B}T}-T\left(\int\frac{c_{V}\left(T'\right)}{T'}\dd T'+k_{B}\ln\left(\frac{V}{N}\right)+\text{konst.}\right)+k_{B}T
\end{align*}
Mit $c_{V}=\frac{3}{2}k_{B}$ ergibt sich:
\begin{align*}
\frac{\mu}{k_{B}T} & =-f\left(T\right)-\ln\left(\frac{V}{N}\right)+\text{konst.}=-f\left(T\right)+\ln\left(\frac{P}{k_{B}T}\right)+\text{konst.}
\end{align*}
$f\left(T\right)$ enthält die $T$-Abhängigkeiten beschrieben durch
das Integral. Jetzt folgt:
\begin{align*}
\frac{\mu_{i}}{k_{B}T} & =-f_{i}\left(T\right)+\ln\left(C_{i}\right)+\ln\left(\frac{P}{k_{B}T}\right)+\text{konst.}
\end{align*}
In Gleichung (\ref{eq:Summe_mu_i-nu_i}):
\begin{align*}
0 & \stackrel{!}{=}\sum_{i=1}^{m}\frac{\mu_{i}}{k_{B}T}\nu_{i}=-\sum_{i=1}^{m}f_{i}\nu_{i}+\sum_{i=1}^{m}\nu_{i}\ln\left(C_{i}\right)+\sum_{i=1}^{m}\nu_{i}\ln\left(\frac{P}{k_{B}T}\right)+\text{konst.}=\\
 & =\ln\left(\prod_{i=1}^{m}C_{i}^{\nu_{i}}\right)+\ln\left(P^{\sum_{i=1}^{m}\nu_{i}}\right)-\ln\left(k\left(T\right)\right)
\end{align*}
Dies ergibt das \emph{Massenwirkungsgesetz}:
\begin{align}
\fbox{\ensuremath{{\displaystyle \prod_{i=1}^{m}C_{i}^{\nu_{i}}=\frac{k\left(T\right)}{P^{\sum_{i=1}^{m}\nu_{i}}}}}}
\end{align}
Beispiel:
\begin{enumerate}
\item $\text{NH}_{3}$-Synthese:
\begin{align*}
\frac{C_{\text{N}_{2}}\cdot\left(C_{\text{H}_{2}}\right)^{3}}{C_{\text{NH}_{3}}} & =\frac{k\left(T\right)}{P^{2}}
\end{align*}
Zum Beispiel begünstigt eine Druckerhöhung die $\text{NH}_{3}$-Synthese
\item Wasser:
\begin{align*}
\text{H}_{3}\text{O}^{+}+\text{OH}^{-} & \leftrightarrows2\text{H}_{2}\text{O}
\end{align*}
$\nu_{1}=1,\nu_{2}=1,\nu_{3}=-2$
\begin{align*}
\frac{C_{\text{H}_{3}\text{O}^{+}}C_{\text{OH}^{-}}}{\left(C_{\text{H}_{2}\text{O}}\right)^{2}} & =k\left(T\right)\\
k\left(300\,\text{K}\right) & \approx10^{-14}\\
\Rightarrow\quad C_{\text{H}_{2}\text{O}} & \approx C_{\text{H}_{3}\text{O}^{+}}=C_{\text{OH}^{-}}\approx10^{-7}
\end{align*}
pH-Wert: 
\begin{align*}
-\log C_{\text{H}_{3}\text{O}^{+}} & \approx7
\end{align*}

\end{enumerate}
%DATE: Do 15.11.12


\chapter{Phasenübergänge}


\section{Klassifizierung}

\textcolor{green}{TODO: Abb17}
\begin{itemize}
\item Falls $P$ und $T$ vorgegeben sind, liegt im Gleichgewicht die Phase
mit minimalem $\mu\left(T,P\right)$ vor!
\item Die Einstellung des Gleichgewichts kann beliebig lange dauern. Metastabile
Phasen sind möglich (Quasigleichgewicht, zum Beispiel Diamant).\\
\textcolor{green}{TODO: Abb18}
\item Typisches Phasendiagramm:\\
\textcolor{green}{TODO: Abb19}\\
Der Übergang zwischen zwei Phasen kann \emph{diskret} (blau) oder
\emph{kontinuierlich} (grün) erfolgen.\\
Betrachte 2 mögliche Phasen $A$ und $B$. In den Phasen $A$ und
$B$ herrscht Gleichgewicht gegenüber Volumen- und Wärmeaustausch.
Dies führt zu $\mu_{A}$ und $\mu_{B}$. Nur an der Phasengrenze gilt
$\mu_{A}=\mu_{B}$. (Das muss nicht für die partiellen Ableitungen
gelten.)
\item Überquere die Phasengrenze bei konstantem Druck:
\begin{align*}
\frac{\partial\mu_{A}\left(T,P\right)}{\partial T} & \not=\frac{\partial\mu_{B}\left(T,P\right)}{\partial T}
\end{align*}
Dies führt zu einem Sprung $\Delta s$ der Entropie.
\item Überquere die Phasengrenze bei konstanter Temperatur:
\begin{align*}
\frac{\partial\mu_{A}\left(T,P\right)}{\partial P} & \not=\frac{\partial\mu_{B}\left(T,P\right)}{\partial P}
\end{align*}
Dies führt zu einem Sprung $\Delta v$ im Volumen.\\
\textcolor{green}{TODO: Abb20,21,22,23}\\
Wenn die Übergänge in Richtung kritischer Punkt verschoben werden,
gehen die Sprünge in der Entropie beziehungsweise Volumen gegen Null.
Hinter dem kritischen Punkt ist der Übergang kontinuierlich.\\
Am kritischen Punkt gilt:
\begin{align*}
\mu_{A} & =\mu_{B}\\
\frac{\partial}{\partial T}\left(\mu_{A}-\mu_{B}\right)\left(T,P\right) & =0\\
\frac{\partial^{2}}{\partial T^{2}}\left(\mu_{A}-\mu_{B}\right)\left(T,P\right) & \not=0
\end{align*}
Analog für $P$.
\item Ehrenfestsche Klassifizierung: Falls eine $n$-te (partielle) Ableitung
von $\mu\left(T,P\right)$ einen Sprung aufweist, dann spricht man
von einem Phasenübergang $n$-ter Ordnung. In diesem Fall sind alle
$n$-ten partiellen Ableitungen von $\mu\left(T,P\right)$ unstetig.\\
Der Phasenübergang fest-flüssig, flüssig-gasförmig ist 1-ter Ordnung\\
Der Phasenübergang am kritischen Punkt flüssig-gasförmig ist 2-ter
Ordnung.
\item Andere Klassifizierung: Stetigkeit eines Ordnungsparameters, zum Beispiel
der Dichte.\\
Falls $\psi$ unstetig ist: Übergang 1-ter Ordnung\\
Falls $\psi$ stetig ist: Übergang 2-ter Ordnung\\
\textcolor{green}{TODO: Abb}\\
Bei $T=T_{c}$ kann man nach dem Ordnungsparameter entwickeln, da
dieser klein ist. (Landau)
\end{itemize}
Ferromagnet (FM)-Paramagnet (PM) Übergang $\leadsto$ Magnetismus

Flüssiges Helium $\leadsto$ Tieftemperatur


\section{Das van der Waals-Gas (gasförmig-flüssig)}
\begin{itemize}
\item Die thermische Zustandsgleichung mit Eigenvolumen $b>0$ und Druckabsenkung
$\frac{a}{v^{2}}>0$ durch attraktive Wechselwirkung zwischen den
Molekülen ist:
\begin{align}
\fbox{\ensuremath{{\displaystyle P=P\left(T,v\right)=\frac{k_{B}T}{v-b}-\frac{a}{v^{2}}}}}\label{eq:P_van-der-Waals}
\end{align}
\textcolor{green}{TODO: Abb25; ideales Gas blau}\\
Isothermen des van der Waals-Gases:
\begin{align*}
\fbox{\ensuremath{{\displaystyle E=E\left(T,V,N\right)=Ne\left(T\right)-N\frac{a}{v}}}}
\end{align*}
Ideales Gas:
\begin{align*}
P & =\frac{k_{B}T}{V}
\end{align*}

\end{itemize}

\subsubsection*{Maxwell-Konstruktion}

Experimentell sind typischerweise $P$ und $T$ vorgegeben. Wie erhält
man $v$?
\begin{itemize}
\item Ein System ist nur dann stabil, wenn für die Kompressibilität gilt:
\begin{align}
\kappa_{T} & =-\frac{1}{V}\frac{\partial V\left(P,T\right)}{\partial P}>0 & \frac{\partial P\left(P,T\right)}{\partial V} & <0
\end{align}
Bereiche positiver Steigung von $P\left(V\right)$ sind mechanisch
instabil.\\
\textcolor{green}{TODO: Abb26}\\
Überhitzte und unterkühlte Phase sind nicht im Gleichgeweicht, aber
realisierbar.
\item Welcher Verlauf der Isothermen tritt tatsächlich ein?
\item Für kleine Drücke nähert sich die Isotherme dem idealen Gas.
\begin{align*}
\Rightarrow\quad\mu_{A} & \,\widehat{=}\,\text{gasförmige Phase}
\end{align*}
Für große Drücke nähert sich das Volumen dem Eigenvolumen:
\begin{align*}
\Rightarrow\quad\mu_{B} & \,\widehat{=}\,\text{flüssige Phase}
\end{align*}

\item Tatsächlich hat $\mu_{A}\left(T,P_{d}\right)\stackrel{!}{=}\mu_{B}\left(T,P_{d}\right)$
für nicht monotone Isothermen genau eine Lösung $P_{d}\left(T\right)$.
Es existiert also ein Druck $P=P_{d}\left(T\right)$, bei dem die
Phasen im Gleichgewicht sind.
\begin{align*}
\mu & =\frac{G}{N}=\frac{F}{N}+\frac{PV}{N}=f+Pv
\end{align*}
\begin{align}
0 & \stackrel{!}{=}\mu_{A}-\mu_{B}=f_{A}-f_{B}+P_{d}\left(v_{A}-v_{B}\right)\nonumber \\
\Rightarrow\quad f_{B}-f_{A} & =P_{d}\left(v_{A}-v_{B}\right)\label{eq:fB-fA}
\end{align}
Unabhängig davon gilt:
\begin{align}
f_{B}-f_{A} & =\int_{v_{B}}^{v_{A}}\frac{\partial f\left(T,v\right)}{\partial v}\dd v=\int_{v_{B}}^{v_{A}}\dd vP\left(T,v\right)\label{eq:fB-fA-Integral}
\end{align}
Aus (\ref{eq:fB-fA}) und (\ref{eq:fB-fA-Integral}) folgt:
\begin{align}
P_{d}\left(v_{A}-v_{B}\right) & =\int_{v_{A}}^{v_{B}}\dd vP\left(T,v\right)
\end{align}
Das heißt $P_{d}$ ist so zu wählen, dass ,,Flächen unter und über
der Horizontalen $P_{d}$ gleich sind.``
\begin{align*}
P & =P_{d}\left(T\right)
\end{align*}
ist eindeutig festgelegt. Bei $P=P_{d}$ gilt $\mu_{A}=\mu_{B}$.\\
Das Verhalten des chemischen Potentials folgt aus der Gibbs-Duhem-Gleichung
(\ref{eq:Duhem-Gibbs}).
\begin{align*}
\dd\mu & =-\frac{S}{N}\dd T+\underbrace{\frac{V}{N}}_{=\frac{\partial\mu_{A,B}}{\partial P}=v_{A,B}}\dd P
\end{align*}
Das chemische Potential hat unterschiedliche Steigung.\\
\textcolor{green}{TODO: Abb V(p), $\mu\left(P\right)$. Siehe oben.
$\mu\left(T,P\right)$ minimal.}
\item Verdampfungsenthalpie:
\begin{align*}
\dd H & =T\dd S+V\dd P\stackrel{\dd P=0}{=}T\dd S
\end{align*}
\begin{align*}
Q & =\int_{S_{B}}^{S_{A}}T\dd S=\int_{H_{B}}^{H_{A}}\dd H=H_{A}-H_{B}
\end{align*}
\begin{align}
h & =\frac{H}{N}=\frac{E+PV}{N}\stackrel{\text{van der Waals}}{=}\frac{3}{2}k_{B}T-\frac{a}{V}+Pv
\end{align}
\begin{align}
q & =\frac{Q}{N}=h_{A}-h_{B}=\frac{q}{v_{B}}-\frac{q}{v_{A}}+P\left(v_{A}-v_{B}\right)\approx\\
 & \approx\underbrace{\frac{a}{v_{\text{flüssig}}}}_{\sr{}{\text{Mittlere Wechselwirkung}}{\text{zwischen den Molekülen}}}+\underbrace{Pv_{\text{gas}}}_{\text{Ausdehnungsarbeit}}=\text{Übergangsenthalpie}\not=0\nonumber 
\end{align}
\textcolor{green}{TODO: ABB $S\left(T\right)$ und $\mu\left(T\right)$,
siehe oben.}
\item Am kritischen Punkt gilt:
\begin{align*}
v_{\text{Gas}}-v_{\text{flüssig}} & =v_{A}-v_{B}\to0
\end{align*}
\begin{align}
q & =T\left(s_{A}-s_{B}\right)\xrightarrow{T\to T_{\text{krit.}}}0
\end{align}
\begin{align*}
\frac{\partial\mu\left(T,P\right)}{\partial T}\bigg|_{T_{\text{krit.}}} & =0
\end{align*}
Also ist es ein Phasenübergang 2-ter Ordnung.\\
Man kann für die Lage des kritischen Punktes zeigen:
\begin{align}
v_{\text{krit.}} & =3b & k_{B}T_{\text{krit.}} & =\frac{8}{27}\frac{a}{b} & P_{\text{krit.}} & =\frac{a}{27b^{2}}
\end{align}
\begin{align*}
\frac{P_{\text{krit}}\cdot v_{\text{krit.}}}{k_{B}T_{\text{krit.}}} & =\frac{3}{8}=0,375
\end{align*}
Experimentelle Werte sind etwas kleiner, zum Beispiel für $^{4}\text{He}$
ist es $\approx0,308$. Viele Gase sind innerhalb von $10\%$ des
van der Waals-Gases.
\end{itemize}
%DATE: Mi 21.11.12


\part{Festkörperphysik}


\chapter{Die Struktur fester Körper}

Es gibt vier Klassen von Festkörpern:
\begin{enumerate}
\item Einkristalle: Orte der Atome werden durch ein Gitter von Raumpunkten
beschrieben.
\item Polykristalline Festkörper: Aus vielen Einkristallen zusammengesetzt,
deren Größe und Orientierung variiert.
\item Amorphe Festkörper: Atome oder Moleküle sind unregelmäßig angeordnet;
keine strenge Periodizität, aber es gibt oft eine Nahordnung (glasartig).
\item Flüssigkristalle: Zwischenzustand zwischen kristallinem Festkörper
und isotroper Flüssigkeit.
\end{enumerate}
Wir betrachten im Folgenden hauptsächlich Einkristalle.


\section{Einkristalle}
\begin{itemize}
\item Ein Kristall ist eine dreidimensionale periodische Anordnung von Atomen.
\item \emph{Bravaisgitter:} unendliche Anordnung diskreter Punkte, die von
jedem Gitterpunkt aus betrachtet identisch erscheint.
\item Für jeden beliebigen Gitterpunkt gilt:
\begin{align*}
\vec{r} & =n_{1}\vec{a}+n_{2}\vec{b}+n_{3}\vec{c} & n_{1},n_{2},n_{3} & \in\mathbb{Z}
\end{align*}
$\vec{a},\vec{b}$ und $\vec{c}$ sind die primitiven Gittervektoren,
die das Bravaisgitter aufspannen.
\begin{align*}
\fbox{Kristall \ensuremath{=}Gitter \ensuremath{+}Basis}
\end{align*}

\end{itemize}
Beispiel: NaCl

\textcolor{green}{TODO: Abb28, Abb29}

Basis: Relativ zu jedem Gitterpunkt $j$ liegt das Zentrum des Basisatoms
bei:
\begin{align*}
\vec{R}_{j} & =x_{j}\vec{a}+y_{j}\vec{b}+z_{j}\vec{c}
\end{align*}



\section{Die primitive Einheitszelle}
\begin{itemize}
\item Volumenbereich, der durch Translation mit allen Gittervektoren den
gesamten Kristall lückenlos ausfüllt.
\item Primitive Einheitszelle enthält genau einen Gitterpunkt.\\
\textcolor{green}{TODO: Abb30}
\item Wigner-Seitz-Zelle:\\
\textcolor{green}{TODO: siehe Folie}
\end{itemize}

\section{Symmetrieoperationen}
\begin{enumerate}[label=\alph*)]
\item Gittertranslation:
\begin{align*}
\vec{T} & =m_{1}\vec{a}+m_{2}\vec{b}+m_{3}\vec{c}
\end{align*}

\end{enumerate}
Punktsymmetrien:
\begin{enumerate}[label=\alph*)]
\item [b)]\addtocounter{enumi}{2}Spiegelung an einer Ebene:
\begin{align*}
\left(\begin{array}{c}
x\\
y\\
z
\end{array}\right) & \mapsto\left(\begin{array}{c}
x'\\
y'\\
z'
\end{array}\right)=\left(\begin{array}{ccc}
1 & 0 & 0\\
0 & 1 & 0\\
0 & 0 & -1
\end{array}\right)\left(\begin{array}{c}
x\\
y\\
z
\end{array}\right)
\end{align*}

\item Inversion:
\begin{align*}
\left(\begin{array}{c}
x\\
y\\
z
\end{array}\right) & \mapsto\left(\begin{array}{c}
x'\\
y'\\
z'
\end{array}\right)=\left(\begin{array}{ccc}
-1 & 0 & 0\\
0 & -1 & 0\\
0 & 0 & -1
\end{array}\right)\left(\begin{array}{c}
x\\
y\\
z
\end{array}\right)
\end{align*}

\item Drehachsen: 2-, 3-, 4- und 6-zählig\\
Bezeichnung: 2, 3, 4 und 6.
\item Drehinversionsachsen: Drehung und Inversion\\
Bezeichung: $\overline{2}$, $\overline{3}$, $\overline{4}$ und
$\overline{6}$\\
(„fünfzählige Symmetrie“: Damit kann man den Raum nicht füllen. Dies
kommt aber zum Beispiel auf der Oberfläche einer Kugel vor, wie bei
$\text{C}_{60}$-Fullerenen.)
\end{enumerate}

\section{Fundamentale Gitter in drei Dimensionen}
\begin{enumerate}[label=\alph*)]
\item Gruppierung der Bravaisgitter nach Punktsymmetrien in 7 kristallographische
Gittersysteme (Punktgruppen):

\begin{itemize}
\item kubisch, tetragonal, orthorombisch, monoklin, triklin
\item rhomboedrisch
\item hexagonal
\end{itemize}
\item Unter Hinzunahme der Translationssymmetrien ergeben sich 14 Bravaisgitter
(Raumgruppen):\\
Beispiel: kubisches Gittersystem

\begin{itemize}
\item einfach kubisch, \foreignlanguage{english}{simple cubic} (sc)
\item kubisch flächenzentriert, \foreignlanguage{english}{face centered
cubic} (fcc)
\item kubisch raumzentriert, \foreignlanguage{english}{body centered cubic}
(bcc)
\end{itemize}
\item Punktgruppen unter Hinzunahme einer Basis.\\
Aus der Gruppentheorie: 32 Kristallklassen (Punktgruppen)\\
Bezeichnung der Punktgruppen mit Schönflies-Symbolen:\\
Symmetriegruppe eines Würfels: $\text{O}_{n}$;\\
$n$ bedeutet Spiegelebene senkrecht zur Drehachse höchster Zähligkeit\\
O bedeutet 4 mal 3-zählige und 3 mal 4-zählige Drehachsen.
\item Punktsymmetrie und Translation ergeben 230 Raumgruppen.
\end{enumerate}

\section{Die Bedeutung der Symmetrie}
\begin{itemize}
\item Die Symmetrie des Gitters bestimmt die Symmetrie des effektiven Hamiltonoperatos.
\item Lösungen, zum Beispiel der Schrödingergleichung, lassen sich nach
ihrer Symmetrie klassifizieren.\\
Wellenfunktion:\\
\textcolor{green}{TODO: Abb31}
\end{itemize}

\section{Bezeichnung von Ebenen und Richtungen}

\emph{Ebenen} werden mittels Millerindizes charakterisiert.

\textcolor{green}{TODO: Abb32}
\begin{enumerate}
\item Bestimme die Achsenabschnitte $\left(m,n,o\right)$
\item Reziproke Werte $\left(\frac{1}{m},\frac{1}{n},\frac{1}{o}\right)$
\item Multiplikation mit ganzer Zahl $p$, sodass gilt:
\begin{align*}
\left(\frac{1}{m},\frac{1}{n},\frac{1}{o}\right)p & =\left(h,k,l\right)\in\mathbb{Z}^{3}
\end{align*}

\end{enumerate}
Beispiel: Achsenabschnitt $\left(1,2,2\right)$ $\leadsto$ $\left(1,\frac{1}{2},\frac{1}{2}\right)\cdot2=\left(2,1,1\right)$
\begin{itemize}
\item Schreibweisen: $\left(-h,-k,-l\right)=\left(\overline{h},\overline{k},\overline{l}\right)$
\item Menge äquivalenter Ebenen (unter Drehungen, sodass die Kristallstruktur
sich nicht ändert):
\begin{align*}
\left\{ 001\right\}  & =\left(001\right),\left(010\right),\left(100\right),\left(00\overline{1}\right),\left(0\overline{1}0\right),\left(\overline{1}00\right)
\end{align*}

\end{itemize}
Weitere Beispiele:

\noindent \begin{center}
\begin{tabular}{c|c}
Millerindizes$\vphantom{\Big|}$ & Normalenvektor\tabularnewline
\hline 
\hline 
$\left(001\right)$$\vphantom{\Big|}$ & $\vec{a}_{3}$\tabularnewline
\hline 
$\left(100\right)$$\vphantom{\Big|}$ & $\vec{a}_{1}$\tabularnewline
\hline 
$\left(111\right)$$\vphantom{\Big|}$ & $\vec{a}_{1}+\vec{a}_{2}+\vec{a}_{3}$\tabularnewline
\hline 
$\left(110\right)$$\vphantom{\Big|}$ & $\vec{a}_{1}+\vec{a}_{2}$\tabularnewline
\end{tabular}
\par\end{center}

\emph{Richtungen} werden durch Richtungsvektor charakterisiert:
\begin{itemize}
\item Schreibweise zum Beispiel $\left[001\right]$, also parallel zu $\vec{a}_{3}$.
\item Gesamtheit äquivalenter Richtungen: $\left\langle 001\right\rangle =\left[001\right],\left[010\right],\ldots,\left[\overline{1}00\right]$
\end{itemize}

\section{Einfache Kristallstrukturen}
\begin{itemize}
\item Kubisch flächenzentrierte Struktur (fcc):\\
\textcolor{green}{TODO: Abb33}

\begin{itemize}
\item dichteste Kugelpackung
\item häufig bei Metallen
\item Punktgruppe $\text{O}_{n}$
\end{itemize}
\item Hexagonal dichteste Kugelpackung (hcp)\\
12 nächste Nachbarn (Koordinationszahl 12) wie fcc, aber andere Stapelfolge
\item Kubisch raumzentrierte Struktur (bcc)\\
\textcolor{green}{TODO: Abb34}

\begin{itemize}
\item Koordinationszahl 8
\item ebenfalls häufig bei Metallen
\end{itemize}
\item Diamantstruktur: fcc + 2-atomige Basis\\
\textcolor{green}{TODO: Abb35}
\begin{align*}
\vec{a} & =\frac{a}{2}\left(\begin{array}{c}
1\\
0\\
1
\end{array}\right) & \vec{b} & =\frac{a}{2}\left(\begin{array}{c}
1\\
1\\
0
\end{array}\right) & \vec{c} & =\frac{a}{2}\left(\begin{array}{c}
0\\
1\\
1
\end{array}\right)
\end{align*}
2-atomige Basis:
\begin{align*}
\vec{v}_{1} & =\left(\begin{array}{c}
0\\
0\\
0
\end{array}\right) & \vec{v}_{2} & =\frac{a}{4}\left(\begin{array}{c}
1\\
1\\
1
\end{array}\right)
\end{align*}
äquivalente Basis:
\begin{align*}
\vec{v}_{1}' & =-\frac{a}{8}\left(\begin{array}{c}
1\\
1\\
1
\end{array}\right) & \vec{v}_{2}' & =\frac{a}{8}\left(\begin{array}{c}
1\\
1\\
1
\end{array}\right)
\end{align*}
Diamant: gleiche Atome an $\vec{v}_{1}$ und $\vec{v}_{2}$ , zum
Beispiel C, Si, Ge, $\alpha$-Sn
\item Zinkblendestruktur: wie Diamant, aber zwei unterschiedliche Basisatome,
zum Beispiel ZnS, ZnTc, GaAs.
\item NaCl: fcc + 2-atomig Basis:
\begin{align*}
\vec{v}_{\text{Cl}} & =\left(\begin{array}{c}
0\\
0\\
0
\end{array}\right) & \vec{v}_{\text{Na}} & =\left(\begin{array}{c}
\frac{a}{2}\\
0\\
0
\end{array}\right)
\end{align*}

\item CsCl: einfach kubisch (sc) + 2-atomige Basis:
\begin{align*}
\vec{v}_{\text{Cs}} & =\left(\begin{array}{c}
0\\
0\\
0
\end{array}\right) & \vec{v}_{\text{Cl}} & =\frac{a}{2}\left(\begin{array}{c}
1\\
1\\
1
\end{array}\right)
\end{align*}
Dies ist wie bcc mit wechselnden Atomsorten.
\end{itemize}
%DATE: Do 22.11.12


\chapter{Die chemische Bindung in Festkörpern}

Bindungen mit abnehmender Stärke:
\begin{itemize}
\item Kovalente Bindung
\item Ionische Bindung
\item Metallische Bindung
\item Wasserstoffbrücken-Bindung
\item van der Waals-Bindung
\end{itemize}

\section{Kovalente Bindung}
\begin{itemize}
\item Bindung zwischen nächsten Nachbarn gleicher Atomesorte.\\
Modell: $\text{H}_{2}^{+}$-Ion\\
\textcolor{green}{TODO: Abb36}\\
Schrödingergleichung:
\begin{align*}
H\psi & =E\psi
\end{align*}
\begin{align}
H & =-\frac{\hbar^{2}}{2m}-\frac{e^{2}}{4\pi\varepsilon_{0}r_{B}^{2}}-\frac{e^{2}}{4\pi\varepsilon_{0}r_{A}^{2}}+\frac{e^{2}}{4\pi\varepsilon_{0}R^{2}}
\end{align}

\item Lösungsansatz: Linearkombination der Wellenfunktionen der Atomorbitale
(LCAO):
\begin{align}
\psi & =c_{A}\psi_{A}+c_{B}\psi_{B}
\end{align}
Hier sind $\psi_{A,B}$ die Wellenfunktionen um den Kern $A$ beziehungsweise
$B$.
\begin{align*}
\psi_{+} & =\frac{1}{\sqrt{2}}\left(\psi_{A}+\psi_{B}\right) &  & \text{bindender Zustand}\\
\psi_{-} & =\frac{1}{\sqrt{2}}\left(\psi_{A}-\psi_{B}\right) &  & \text{antibindender Zustand}
\end{align*}
\textcolor{green}{TODO: Abb37}\\
rot: erhöhte Wahrscheinlichkeit zwischen den Kernen.\\
blau: Nullstelle in der Aufenthaltswahrscheinlichkeit zwischen den
Kernen!\\
\textcolor{green}{TODO: Abb38}
\item Die kovalente Bindung ist im Allgemeinen gerichtet, da atomare Orbitale
aus der Hybridisierung resultieren, zum Beispiel C in Diamant:\\
\textcolor{green}{TODO: Abb39}
\begin{align}
\KET{\psi_{1}} & =\frac{1}{2}\left(\KET s+\KET{p_{x}}+\KET{p_{y}}+\KET{p_{z}}\right)\nonumber \\
\KET{\psi_{2}} & =\frac{1}{2}\left(\KET s+\KET{p_{x}}-\KET{p_{y}}-\KET{p_{z}}\right)\\
\KET{\psi_{3}} & =\frac{1}{2}\left(\KET s-\KET{p_{x}}+\KET{p_{y}}-\KET{p_{z}}\right)\nonumber \\
\KET{\psi_{4}} & =\frac{1}{2}\left(\KET s-\KET{p_{x}}-\KET{p_{y}}-\KET{p_{z}}\right)\nonumber 
\end{align}
Durch Hybridisierung werden gerichtete Orbitale erzeugt.\\
\textcolor{green}{TODO: Abb40; sp$^{3}$-Orbital}\\
Durch Überlapp von Hybridorbitalen benachbarter Atome entstehen kovalente
Bindungen.\\
Die potentielle Energie als Funktion des Kernabstands wird durch Morsepotential
genähert:
\begin{align}
V\left(R\right) & =-E_{B}\left(2\cdot e^{-\alpha\left(R-R_{0}\right)}-e^{-2\alpha\left(R-R_{0}\right)}\right)
\end{align}
\textcolor{green}{TODO: Plot Morsepotential; $E_{0}$ ist Bindungsenergie;
$R_{0}$ ist Gleichgewichtsabstand; $\alpha$ ist Abklinglänge}\\
Typische Bindungsenergien sind:
\begin{align*}
\text{Diamant: } & 7{,}3\,\frac{\text{eV}}{\text{Atom}}\\
\text{Silizium: } & 4{,}6\,\frac{\text{eV}}{\text{Atom}}
\end{align*}
Grundannahme ist die Born-Oppenheimer Näherung: Elektronen folgen
der Kernbewegung!
\end{itemize}

\section{Ionenbindung}
\begin{itemize}
\item Elektrostatische Anziehung zwischen Ionen
\item Beispiel: NaCl $\to$ $\text{Na}^{+}$, $\text{Cl}^{-}$\\
\textcolor{green}{TODO: Abb41}\\
Modell starrer Kugeln für Ionen (oft) gut, da Ionen abgeschlossene
Schalen besitzen, zum Beispiel $\left[\text{Na}^{+}\right]=1s^{2}2s^{2}2p^{6}$,
$\left[\text{Cl}^{-}\right]=1s^{2}2s^{2}2p^{6}3s^{2}3p^{6}$.
\item Potentielle Energie zwischen Ion $i$ und Ion $j$ zusammengesetzt
aus elektrostatischer Anziehung und Pauli-Abstoßung; $n\ge8$; $i\not=j$:
\begin{align}
\varphi_{ij} & =\frac{q_{i}q_{j}}{4\pi\varepsilon_{0}}\cdot\frac{1}{r_{ij}}+\begin{cases}
ce^{-\frac{r_{ij}}{\varrho}}\\
c\frac{1}{r_{ij}^{n}}
\end{cases}
\end{align}
\textcolor{green}{TODO: Plot $\varphi_{ij}$}
\item Gesamte potentielle Energie eines Kristalls aus $N$ Ionenpaaren
\begin{align}
\Phi & =N\sum_{j\not=i}\varphi_{ij}=N\sum_{i\not=j}\left(\frac{1}{4\pi\varepsilon_{0}}\frac{q_{i}q_{j}}{r_{ij}}+\frac{c}{r_{ij}^{n}}\right)
\end{align}

\item Berücksichtige nur $z_{\text{nn}}$ nächste Nachbarn im Abstand $R_{\text{nn}}$
für abstoßende Wechselwirkung und parametrisiere:
\begin{align*}
r_{ij} & =R_{\text{nn}}\cdot p_{ij}
\end{align*}
\begin{align}
\Phi & =N\cdot z_{\text{nn}}\cdot\frac{c}{R_{\text{nn}}^{n}}+N\frac{q^{2}}{4\pi\varepsilon_{0}R_{\text{nn}}}\underbrace{\sum_{i\not=j}\frac{\pm1}{p_{ij}}}_{=:\alpha}
\end{align}
$\alpha$ heißt Madelung-Konstante. Sie ist durch die Struktur definiert.
\item Beispiele:\\
NaCl: $\alpha\approx1,748$ (fcc, 2-atomige Basis)\\
CsCl: $\alpha\approx1,763$ (sc, 2-atomige Basis)\\
ZnS: $\alpha\approx1,641$ (Zinkblende)
\item Man zeigt leicht, dass die gesamte Bindungsenergie im Gleichgewicht
ist:
\begin{align}
E_{B} & =\frac{N\alpha q^{2}}{4\pi\varepsilon_{0}R_{0}}\left(1-\frac{1}{n}\right)
\end{align}
Dabei ist $R_{0}$ der Gleichgewichtsabstand nächster Nachbarn.
\item Beispiele:\\
NaCl: $E_{B}=7{,}95\,\frac{\text{eV}}{\text{Molekül}}$, NaJ: $E_{B}=7{,}10\,\frac{\text{eV}}{\text{Molekül}}$
\item Zwischen ionischer und kovalenter Bindung gibt es einen fließenden
Übergang.
\end{itemize}

\section{Metallische Bindung}

\textcolor{green}{TODO: Abb42}
\begin{itemize}
\item Frei bewegliche $s$-Elektronen/Atom sind stark delokalisiert.\\
Reduktion der kinetischen Energie und Abschirmung der Kerne
\item Die Wechselwirkung ist nicht gerichtet, was oft zur dichtesten Kugelpackung
führt.
\item Bei Übergangsmetallen gibt es zusätzliche Bindungskräfte aus der Wechselwirkung
von $d$-Elektronen, also ein richtungsabhängiges kovalentes Gerüst.\\
\textcolor{green}{TODO: Abb43}
\item Bindungsenergien:\\
Alkalimetalle ca. $1\,\frac{\text{eV}}{\text{Atom}}$\\
Eisen ca. $4,3\,\frac{\text{eV}}{\text{Atom}}$\\
Wolfram ca. $8,7\,\frac{\text{eV}}{\text{Atom}}$\\
Bei Eisen und Wolfram tragen die inneren Schalen bei.
\end{itemize}

\section{Wasserstoffbrücken-Bindung}
\begin{itemize}
\item Wasserstoffatome H zwischen zwei stark elektronegativen Atomen bildet
eine „Brücke“.\\
\textcolor{green}{TODO: Abb44}

\begin{itemize}
\item Sauerstoff bindet mit H.
\item Die positive Partialladung an H erlaubt das „Andocken“ von O.
\end{itemize}
\item Wichtig für $\text{H}_{2}\text{O}$ und organische Molekülkristalle
mit Bindungsenergie $\approx$ $0,1\,\text{eV}$.
\end{itemize}

\section{Van der Waals-Bindung}
\begin{itemize}
\item Die van der Waals-Wechselwirkung ist „omnipräsent“, aber nur relevant,
falls keine Bindung möglich ist, zum Beispiel in Edelgasen oder gesättigten
Molekülen mit abgeschlossener Schale.\\
\textcolor{green}{TODO: Abb45}\\
Spontane Fluktuationen in Atom $A$ erzeugen ein Dipolmoment, das
in Atom $B$ ein Dipolmoment induziert.
\item Die elektrischen Felder eines Dipols sind proportional zu $\frac{1}{r^{3}}$
und das induzierte Dipolmoment ist proportional zu $E$, also ist
das Wechselwirkungspotential proportional zu $\frac{1}{r^{6}}$.\\
\textcolor{green}{TODO: Lennard-Jones-Potential; $\sigma=\sqrt[6]{2}\sigma'$
Minimumsabstand, $\sigma'$ Nulldurchgang}
\begin{align*}
V_{\text{LJ}} & =-E_{B}\left(\left(\frac{\sigma^{'}}{r}\right)^{12}-\left(\frac{\sigma'}{r}\right)^{6}\right)
\end{align*}
Dies hat sich empirisch als sinnvoll erwiesen.
\end{itemize}
\noindent \begin{center}
\begin{tabular}{c||c|c}
Element$\vphantom{\Big|}$ & Bindungsenergie pro Atom (eV) & Schmelztemperatur (K)\tabularnewline
\hline 
\hline 
Ne$\vphantom{\Big|}$ & $0{,}02$ & $24$\tabularnewline
\hline 
Ar$\vphantom{\Big|}$ & $0{,}08$ & $84$\tabularnewline
\hline 
Kr$\vphantom{\Big|}$ & $0{,}116$ & $117$\tabularnewline
\hline 
Xe$\vphantom{\Big|}$ & $0{,}17$ & $161$\tabularnewline
\end{tabular}
\par\end{center}

%DATE: Mi 28.11.12


\chapter[Beugung am Kristall]{Beugung am Kristall und reziprokes Gitter}

Zur Untersuchung der Kristallstruktur verwendet man die Beugung von
Wellen einer Wellenlänge von ca. $1\ang$, was in etwa dem interatomaren
Abstand entspricht, am Kristallgitter. Verschiedene Arten von Strahlung
wir verwendet:
\begin{itemize}
\item Röntgenstrahlung
\item Atomstrahlen
\item Elektronen
\item Neutronen
\end{itemize}
Die Kopplungsmechanismen an die Materie sind:
\begin{itemize}
\item Röntgenstrahlung koppelt an Elektronen.
\item Atome (z.B. He) koppeln durch die Wechselwirkung ihrer Hüllen (extrem
oberflächensensitiv).
\item Elektronen wechselwirken via das Coulombpotential und magnetische
Dipolmoment mit den Elektronen (oberflächensensitiv).
\item Neutronen wechselwirken mit den Atomkernen und mit dem magnetischen
Dipolmoment.
\end{itemize}
\textcolor{green}{TODO: Abb: Wellenlängenabhängigkeit der Energie
von Photonen (keV), Neutronen ($10\,\text{meV}$), Elektronen ($10\,\text{eV}$)}
\begin{itemize}
\item Wellenlänge für Photonen: ${\displaystyle \lambda=\frac{hc}{E_{\text{ph}}}}$
\item Wellenlänge für massive Teilchen: ${\displaystyle \lambda_{\text{de Broglie}}=\frac{h}{\sqrt{2mE}}}$
\end{itemize}

\section{Beugung am Kristallgitter}

Gebeugte Strahlung werden nur in den Richtungen detektiert, in denen
an parallelen Atomebenen reflektierte Wellen konstruktiv interferieren.

\textcolor{green}{TODO: Abb Bragg-Bedingung, Gangunterschied $2d\sin\left(\theta\right)$}

Es ergibt sich eine konstruktive Interferenz, wenn die \emph{Bragg-Bedingung}
erfüllt ist:
\begin{align}
\fbox{\ensuremath{{\displaystyle 2d\sin\left(\theta\right)=n\cdot\lambda}}}
\end{align}


Beispiel: Netzebenen des sc-Gitters:

\textcolor{green}{TODO: Abb Netzebenen}

Reflexionen bei hohen Indizes erfordern kürzere Wellenlängen.


\section{Experimentelle Bestimmung von Kristallstrukturen}
\begin{itemize}
\item Laue-Verfahren: Verwende zum Beispiel Röntgenstrahl mit mit kontinuierlichem
Spektrum. Der Kristall beugt nur diskrete $\lambda$, für welche die
Bragg-Bedingung erfüllt ist. Das Beugungsbild zeigt die gleiche Symmetrie
wie der Kristall. Die Orientierung von Kristallen ist so bestimmbar.
\item Bragg-Verfahren (Drehkristallverfahren): Der Kristall wird im monochromatischen
Röntgenstrahl gedreht. In Abhängigkeit von $\theta$ tritt nacheinander
Reflexion an verschiedenen Atomebenen auf.
\item Debye-Scherer-Verfahren (Pulvermethode): Als Probe wird feinkörniges,
polykristallines Pulver verwendet. Kristallite, die so orientiert
sind, dass sie die Bragg-Bedingung erfüllen, erzeugen Reflexe. Gebeugte
Strahlen, die zu einer Ebenenschar gehören, liegen auf einem Kegelmantel.
\end{itemize}

\section{Allgemeine Beugungstheorie}

Für die Elektronendichte $n$ in einem translationsinvarianten Kristall
gilt:
\begin{align*}
n\left(\vec{r}+\vec{T}\right) & =n\left(\vec{r}\right)
\end{align*}
Die meisten Eigenschaften des Kristalls können im Fourierraum ausgedrückt
werden.

Betrachte die \emph{elastische} Streuung an einer beliebig geformten
Probe.

\textcolor{green}{TODO: Abb Streuung}

Welle am Punkt P:
\begin{align*}
A_{P} & =A_{0}e^{\ii\vec{k}_{0}\left(\vec{R}+\vec{r}\right)-\ii\omega_{0}t}
\end{align*}
Welle am Punkt B:
\begin{align*}
A_{B} & =A_{P}\left(\vec{r},t\right)\varrho\left(\vec{r}\right)\frac{e^{\ii\vec{k}\left(\vec{R}'-\vec{r}\right)}}{\norm{\vec{R}'-\vec{r}}}
\end{align*}
$\varrho\left(\vec{r}\right)$ ist die Elektronendichte.
\begin{itemize}
\item Elastische Streuung bedeutet keinen Energieübertrag, das heißt $\omega_{0}'=\omega_{0}$.
\item Die Probe sei klein im Verhältnis zu den Abständen $\overline{\text{QP}},\overline{\text{PB}}$.
Daher ist der Streuvektor $\vec{K}=\vec{k}_{0}-\vec{k}$ näherungsweise
konstant bei gegebenem $\vec{r}$ in der Probe.
\item Die gesamte Streuamplitude ist:
\begin{align*}
A_{B} & \sim e^{-\ii\omega_{0}t}\int_{\text{Probenvolumen}}\dd^{3}r\varrho\left(\vec{r}\right)e^{\ii\vec{k}\cdot\vec{r}}
\end{align*}

\item Experimentell zugänglich ist die Intensität:
\begin{align}
\fbox{\ensuremath{{\displaystyle I\sim\abs{A_{B}}^{2}\sim\abs{\int\dd^{3}r\varrho\left(\vec{r}\right)e^{\ii\vec{k}\cdot\vec{r}}}^{2}}}}
\end{align}
Die Strahlungsintensität ist proportional zur Fouriertransformation
der Streudichteverteilung!
\end{itemize}

\section{Periodische Strukturen und das reziproke Gitter}

Periodische Funktion im Ortsraum in einer Dimension:

\textcolor{green}{TODO: Abb Gitterabstand $a$}

\begin{align*}
\varrho\left(x\right) & =\varrho\left(x+na\right)\\
\varrho\left(x\right) & =\sum_{p=-\infty}^{+\infty}\varrho_{p}e^{\ii\frac{2\pi}{a}px}
\end{align*}
Man benötigt Entwicklungskoeffizienten $\varrho_{p}$ nur für diskrete
Wellenvektoren $k_{p}=\frac{2\pi}{a}p$. Diese bezeichnet man als
\emph{reziprokes Gitter} (rG).

In drei Dimensionen:-
\begin{align*}
\varrho\left(\vec{r}\right) & =\sum_{\vec{G}\in\text{rG}}\varrho_{\vec{G}}e^{\ii\vec{G}\cdot\vec{r}}
\end{align*}
Ein Vektor $\vec{G}$ ist genau dann ein Gittervektor, wenn $e^{\ii\vec{G}\cdot\vec{r}}$
gitterperiodisch ist. Mit einem beliebigem Vektor $\vec{R}$ des Bravaisgitters
und Ortsvektor $\vec{r}$ gilt:
\begin{align*}
e^{\ii\vec{G}\cdot\vec{r}} & =e^{\ii\vec{G}\left(\vec{r}+\vec{R}\right)}\\
e^{\ii\vec{G}\cdot\vec{R}} & =1
\end{align*}
Es folgt mit einem $m\in\mathbb{Z}$:
\begin{align}
\fbox{\ensuremath{{\displaystyle \vec{G}\cdot\vec{R}=2\pi m}}}
\end{align}
Das reziproke Gitter ist ein vollständiger Satz aller Wellenvektoren
$\vec{G}$ für ebene Wellen mit der Periodizität des Bravaisgitters,
das heißt die Fouriertransformation des Bravaisgitters.


\section{Konstruktion des reziproken Gitters}

Es seien $\vec{a}_{1},\vec{a}_{2}$ und $\vec{a}_{3}$ primitive Basisvektoren
des Bravaisgitters. Dann sind
\begin{align}
\vec{b}_{1} & =2\pi\frac{\vec{a}_{2}\times\vec{a}_{3}}{\vec{a}_{1}\cdot\left(\vec{a}_{2}\times\vec{a}_{3}\right)} & \vec{b}_{2} & =2\pi\frac{\vec{a}_{3}\times\vec{a}_{1}}{\vec{a}_{1}\cdot\left(\vec{a}_{2}\times\vec{a}_{3}\right)} & \vec{b}_{3} & =2\pi\frac{\vec{a}_{1}\times\vec{a}_{2}}{\vec{a}_{1}\cdot\left(\vec{a}_{2}\times\vec{a}_{3}\right)}
\end{align}
primitive Vektoren des reziproken Gitters. Dabei ist $\vec{a}_{1}\cdot\left(\vec{a}_{2}\times\vec{a}_{3}\right)$
das Volumen der Einheitszelle. Das gesamte reziproke Gitter ergibt
sich mit $k_{i}\in\mathbb{Z}$ als:
\begin{align*}
\vec{G} & =k_{1}\vec{b}_{1}+k_{2}\vec{b}_{2}+k_{3}\vec{b}_{3}
\end{align*}
Man zeigt leicht $\vec{a}_{i}\cdot\vec{b}_{j}=2\pi\delta_{ij}$. Mit
beliebigem Gittervektor $\vec{R}=n_{1}\vec{a}_{1}+n_{2}\vec{a}_{2}+n_{3}\vec{a}_{3}$
gilt damit:
\begin{align*}
\vec{G}\cdot\vec{R} & =2\pi\left(\sum_{i=1}^{3}n_{i}k_{i}\right)=2\pi\underbrace{m}_{\in\mathbb{Z}}
\end{align*}
Einfachstes Beispiel ist das sc-Gitter:

\textcolor{green}{TODO: Abb sc-Gitters}

\begin{align*}
\norm{\vec{a}_{1}} & =\norm{\vec{a}_{2}}=\norm{\vec{a}_{3}}=a
\end{align*}
\begin{align*}
\vec{b}_{1} & =\frac{2\pi}{a^{3}}\left(\vec{a}_{2}\times\vec{a}_{3}\right)=\frac{2\pi}{a^{3}}\vec{a}_{1}\\
\vec{b}_{2} & =\frac{2\pi}{a^{3}}\left(\vec{a}_{1}\times\vec{a}_{3}\right)=\frac{2\pi}{a^{3}}\vec{a}_{2}\\
\vec{b}_{3} & =\frac{2\pi}{a^{3}}\left(\vec{a}_{1}\times\vec{a}_{2}\right)=\frac{2\pi}{a^{3}}\vec{a}_{3}
\end{align*}
Das reziproke Gitter eines fcc-Gitters ist ein bcc-Gitter und umgekehrt.

%DATE: Do 29.11.12


\section{Netzebenen und reziprokes Gitter}

Man zeigt leicht:
\begin{enumerate}
\item Der reziproke Gittervektor $\vec{G}_{hkl}=h\vec{b}_{1}+k\vec{b}_{2}+l\vec{b}_{3}$
steht senkrecht auf den Netzebenen mit Millerindizes $\left(hkl\right)$.
\item Für den Abstand $d_{\left(hkl\right)}$ dieser Netzebenen gilt $d_{\left(hkl\right)}=\frac{2\pi}{\norm{\vec{G}_{hkl}}}$.
\end{enumerate}
Also hat eine ebene Welle $\exp\left(-\ii\vec{G}_{hkl}\cdot\vec{r}\right)$
die selbe Periodizität wie das direkte Gitter in Richtung $\left[hkl\right]$,
denn es gilt für $r=nd_{hkl}$ und $n\in\mathbb{Z}$:
\begin{align}
\exp\left(\ii\vec{G}_{hkl}\vec{r}\right) & =\exp\left(\ii\frac{2\pi}{d_{hkl}}r\right)=1
\end{align}



\section{Die Streubedingung bei periodischen Strukturen}

Behauptung: Der Streuvektor $\vec{K}=\vec{k}-\vec{k}_{0}$ muss ein
reziproker Gittervektor sein, das heißt $\vec{K}=\vec{G}$ (Laue-Bedingung).
Der Grund dafür ist:
\begin{itemize}
\item Für die Intensität gilt:
\begin{align*}
I\left(\vec{K}\right) & \sim\frac{A_{0}^{2}}{\left(R'\right)^{2}}\abs{\sum_{\vec{G}}\varrho_{G}\int_{V}\dd^{3}re^{\ii\left(\vec{G}-\vec{k}\right)\vec{r}}}^{2}
\end{align*}
\begin{align*}
\int_{V}\dd^{3}re^{\ii\left(\vec{G}-\vec{k}\right)\vec{r}} & \begin{cases}
=V & \text{für }\vec{G}=\vec{K}\\
\approx0 & \text{sonst}
\end{cases}
\end{align*}

\item Beugungsbild:\\
\textcolor{green}{TODO: Abb47; kleines Probenvolumen; Abb48}\\
\emph{Ewald-Kugel}:\\
\textcolor{green}{TODO: Abb49; $\vec{k}_{0}$: einlaufender Wellenvektor,
$\vec{k}$: gebeugter Wellenvektor, $\vec{G}$: reziproker Gittervektor}\\
Bezeichnung der auftretenden Reflexe nach den Indizes $\left(hkl\right)$
des zugehörigen Gittervektors $\vec{G}=h\vec{b}_{1}+k\vec{b}_{2}+l\vec{b}_{3}$.
\end{itemize}

\section{Die Braggsche Deutung der Beugungsbedingung}
\begin{itemize}
\item Laue-Bedingung $\vec{K}=\vec{G}$ ist äquivalent zur Bragg-Gleichung.\\
\textcolor{green}{TODO: Abb50}
\begin{align*}
k & =2k_{0}\sin\theta=G_{hkl}=\frac{2\pi}{d_{hkl}}
\end{align*}
Mit $k_{0}=\frac{2\pi}{\lambda}$ folgt:
\begin{align*}
\lambda & =2d_{hkl}\cdot\sin\theta
\end{align*}

\item Anmerkung: Wegen
\begin{align*}
k_{0} & =\norm{\vec{k}_{0}}=\norm{\vec{k}}
\end{align*}
folgt aus der Laue-Bedingung $\vec{G}=\vec{k}_{0}-\vec{k}$:
\begin{align*}
k_{0}^{2} & =\norm{\vec{k}-\vec{G}}^{2}
\end{align*}
\begin{align}
\fbox{\ensuremath{{\displaystyle \vec{k}_{0}\cdot\hat{G}=\frac{1}{2}G}}}
\end{align}
Dabei ist $\hat{G}$ die Richtung und $G$ der Betrag von $\vec{G}$.\\
Die Streuung erfolgt also nur dann, wenn der Endpunkt von $\vec{k}_{0}$
auf einer Mittelsenkrechten zwischen zwei reziproken Gitterpunkten
liegt. (Bragg-Ebene)\\
\textcolor{green}{TODO: Abb51}
\end{itemize}

\section{Die Brillouinzone}
\begin{itemize}
\item Die erste Brillouinzone ist die Wigner-Seitz-Zelle des reziproken
Gitters.
\item Beachte die Bezeichnung von Punkten hoher Symmetrie, zum Beispiel
wird der Ursprung ($k=0$) als $\Gamma$-Punkt bezeichnet. Weitere
Punkte heißen $X,L,K,W,\ldots$
\item Die erste Brillouinzone eines fcc-Gitters ist die Wigner-Seitz-Zelle
eines bcc-Gitters und umgekehrt.
\end{itemize}

\section{Fourier-Analyse der Basis, Strukturfaktor und Atomformfaktor (Röntgenstrahlung)}
\begin{itemize}
\item Frage: Was ist die Intensität der erlaubten Reflexe bei Beugungsexperimenten?
\item Falls $\vec{k}=\vec{G}$ erfüllt ist, so ist die Streuamplitude:
\begin{align}
A & =\int_{\text{Kristall}}\dd^{3}r\varrho\left(\vec{r}\right)e^{-\ii\vec{G}\vec{r}}\sr ={N\text{ Zellen}}{}N\cdot\int_{\text{Zelle}}\dd^{3}r\varrho\left(\vec{r}\right)e^{-\ii\vec{G}\vec{r}}=N\cdot S_{\vec{G}}
\end{align}
$S_{\vec{G}}=\varrho_{hkl}$ ist die Fourier-Transformierte der Elektronenverteilung
einer Einheitszelle ausgewertet für den Wellenvektor $\vec{G}$.
\item Betrachte $\varrho\left(\vec{r}\right)$ einer Einheitszelle als Superposition
$\varrho_{j}\left(\vec{r}\right)$ aller $s$ Basisatome.\\
\textcolor{green}{TODO: Abb52}
\begin{align}
\vec{r}_{n} & =n_{1}\vec{a}+n_{2}\vec{b}+n_{3}\vec{c}\nonumber \\
\varrho\left(\vec{r}\right) & =\sum_{j=1}^{s}\varrho\left(\vec{r}-\vec{d}_{j}\right)
\end{align}
\begin{align}
S_{\vec{G}} & =\sum_{j=1}^{s}e^{-\ii\vec{G}\vec{d}_{j}}\int\dd^{3}r\varrho_{j}\left(\vec{r}'\right)e^{-\ii\vec{G}\vec{r}'}
\end{align}
\begin{align}
\fbox{\ensuremath{{\displaystyle f_{j}\left(\vec{G}\right)=}\int\dd^{3}\vec{r}\varrho_{j}\left(\vec{r}'\right)e^{-\ii\vec{G}\vec{r}}}}
\end{align}
Der Atomformfaktor $f_{j}$ ist ein Maß für die Streukraft des $j$-ten
Atoms in der Einheitszelle.
\item Der Strukturfaktor wird damit zu:
\begin{align*}
S_{\vec{G}} & =\sum_{j=1}^{s}f_{j}\left(\vec{G}\right)e^{-\ii\vec{G}\vec{d}_{j}}
\end{align*}

\item Mit $\vec{d}_{j}=x_{j}\vec{a}+y_{j}\vec{b}+z_{j}\vec{c}$ ($x_{i},y_{i},z_{i}\in\mathbb{R}$)
gilt für $hkl$-Reflex:
\begin{align}
S_{hkl} & =\sum_{j=1}^{s}f_{j}\left(hkl\right)\exp\left(-2\pi\ii\left(x_{j}h+y_{j}k+z_{j}l\right)\right)
\end{align}
Falls $S_{hkl}=0$ ist, wird keine Strahlung in den vom Raumgitter
erlaubten Reflex $\vec{G}_{hkl}$ gestreut.\\
Beispiel: Strukturfaktor des bcc-Gitters (z.B. metallisches Na)\\
Das bcc-Gitter wird als sc-Gitter mit zweiatomiger Basis beschrieben:
ein Atom bei $\left(0,0,0\right)$ und ein Atom bei $\frac{a}{2}\left(1,1,1\right)$.\\
Da die Atome identisch sind, folgt $f_{1}=f_{2}=f$.
\begin{align*}
S_{hkl} & =f\left(1+\exp\left(-\ii\pi\left(h+k+l\right)\right)\right)=\begin{cases}
0 & \text{falls }h+k+l\text{ ungerade}\\
2f & \text{falls }h+k+l\text{ gerade}
\end{cases}
\end{align*}
Es gibt also keine Reflexe zu den Ebenen $\left(100\right)$, $\left(300\right)$,
$\left(111\right)$ und so weiter.\\
Achtung: $hkl$ beziehen sich hier auf die sc-Einheitszelle.\\
\textcolor{green}{TODO: Abb53}
\item Der \emph{Debye-Waller Faktor} (nach \noun{Peter Debye} und \noun{Ivar
Waller}) beschreibt die Temperaturabhängigkeit der Intensität der
kohärent elastisch gestreuten Strahlung an einem Kristallgitter.
\end{itemize}
%DATE; Mi 5.12.12


\chapter{Dynamik der Kristallgitter}

Bisher haben wir ein statisches Kristallgitter betrachtet.

Jetzt gehen wir davon aus, dass die Atome um ihre Gleichgewichtslage
schwingen können. So entstehen Gitterwellen, deren Quanten \emph{Phononen}
genannt werden. Hierzu betrachten wir die adiabatische (Born-Oppenheimer)
und die harmonische Näherung.


\section{Schwingungen in Gitter mit einatomiger Basis}

Ziel ist die Bestimmung der Frequenz $\omega\left(\vec{k}\right)$
der Gitterwelle in Abhängigkeit vom Wellenvektor $\vec{k}$. (\emph{Dispersionsrelation})

Für jeden Wellenvektor gibt es 3 Schwingungszustände, einen mit longitudinaler
Polarisation und zwei mit transversaler.


\subsubsection*{Longitudinale Gitterwelle}

\textcolor{green}{TODO: Abb54}


\subsubsection*{Transversale Gitterwelle}

\textcolor{green}{TODO: Abb55}

Bei Gittern mit einatomiger Basis können die einzelnen Gitterebenen
zu einem „Atom“ zusammen gefasst werden, was das Problem auf eine
lineare Kette reduziert.

\textcolor{green}{TODO: Abb56}

Die Gesamtkraft auf Atom $s$ ist (Hooksches Gesetz):
\begin{align}
F_{s}\left(t\right) & =\sum_{p}c_{p}\left(u_{s+p}\left(t\right)-u_{s}\left(t\right)\right)
\end{align}
Die Bewegungsgleichung der Ebene $s$ lautet (Atommasse $M$):
\begin{align}
M\frac{\dd^{2}u_{s}\left(t\right)}{\dd t^{2}} & =\sum_{p\in\mathbb{Z}}c_{p}\left(u_{s+p}\left(t\right)-u_{s}\left(t\right)\right)
\end{align}
Der Ansatz
\begin{align*}
u_{s}\left(t\right) & =u_{s}e^{-\ii\omega t}
\end{align*}
führt zu:
\begin{align}
\frac{\dd^{2}u_{s}}{\dd t^{2}} & =-\omega^{2}u_{s}\nonumber \\
\Rightarrow\quad-M\omega^{2}u_{s} & =\sum_{p}c_{p}\left(u_{s+p}-u_{s}\right)\label{eq:Bewegungsgleichung}
\end{align}
Die Lösungen sind ebene Wellen:
\begin{align}
u_{s+p} & =ue^{\ii\left(s+p\right)ka}\label{eq:us+p}
\end{align}
\begin{framed}

Bei gegebenem Gitter hängt der Abstand $a$ zwischen einzelnen Netzebenen
von der Richtung von $\vec{k}$ ab.

\end{framed}

\textcolor{green}{TODO: Abb57}

Gleichung (\ref{eq:us+p}) in (\ref{eq:Bewegungsgleichung}) eingesetzt
liefert:
\begin{align*}
-M\omega^{2}ue^{\ii ska} & =\sum_{p}c_{p}\left(e^{\ii\left(s+p\right)ka}-e^{\ii ska}\right)u\\
M\omega^{2} & =-\sum_{p}c_{p}\left(e^{\ii pka}-1\right)
\end{align*}
Wegen der Translationssymmetrie gilt $c_{p}=c_{-p}$ und es folgt:
\begin{align*}
M\omega^{2} & =-\sum_{p>0}c_{p}\left(e^{\ii pka}+e^{-\ii pka}-2\right)
\end{align*}
\begin{align}
\fbox{\ensuremath{{\displaystyle \omega^{2}=\frac{2}{M}\sum_{p>0}c_{p}\left(1-\cos\left(pka\right)\right)}}}
\end{align}
\begin{align*}
\frac{\dd\omega^{2}}{\dd k} & =\frac{2}{M}\sum_{p>0}c_{p}pa\sin\left(pka\right)\stackrel{\text{für }k=\pm\frac{\pi}{a}}{=}0
\end{align*}
Die Dispersion an der Zonengrenze ist also immer flach.

Falls nur die Wechselwirkung zwischen den nächsten Nachbarebenen berücksichtigt
wird, gilt die Dispersionsrelation:
\begin{align}
\fbox{\ensuremath{{\displaystyle \omega=\sqrt{\frac{4c_{1}}{M}}\cdot\sin\left(\frac{1}{2}ka\right)}}}
\end{align}
\textcolor{green}{TODO: Abb58, Plot $\omega\left(k\right)$}

Beachte: Bei transversaler Auslenkung kann $c_{1}$ verschieden sein.


\section{Die Bedeutung der 1. Brillouin-Zone}

Reales Gitter (einatomige lineare Kette):

\textcolor{green}{TODO: abb59, $\vec{a}=a\vec{e}_{x}$ ist primitiver
Vektor; $\vec{G}=\frac{2\pi}{a}\vec{e}_{x}$; $\vec{G}\cdot\vec{a}=2\pi$}

\begin{framed}

Nur solche $k$, die in der 1. Brillouin-Zone liegen, haben für elastische
Wellen eine Bedeutung.

\end{framed}

\textcolor{green}{TODO: Abb60}

Eine Welle mit Wellenvektor $k=-\frac{3\pi}{2a}$ ergibt die gleiche
Gitterbewegung wie eine Welle mit $k=\frac{\pi}{2a}$.
\begin{itemize}
\item Durch Subtraktion eines geeigneten reziproken Gittervektors von einem
beliebigen Wellenvektor $k$ erhält man immer einen äquivalenten Wellenvektor
$k'$ aus der 1. Brillouin-Zone.
\begin{align}
k' & =k-\frac{2\pi}{a}\cdot n
\end{align}

\item Im Beispiel von oben:\\
\textcolor{green}{TODO: Abb61}
\item Die Gruppengeschwindigkeit ist die Geschwindigkeit eines Wellenpakets.
\begin{align}
v_{g} & =\frac{\dd\omega}{\dd k}=\sqrt{\frac{C_{1}}{M}}\cdot a\cos\left(\frac{1}{2}ka\right)
\end{align}
\textcolor{green}{TODO: Abb62}\\
Am Zonenrand $\left(k=\pm\frac{\pi}{a}\right)$ verschwindet die Gruppengeschwindigkeit.
Grund dafür ist, dass die Bragg-Reflektion eine stehende Welle erzeugt!
\end{itemize}

\section{Gitter mit zwei Atomen in der primitiven Einheitszelle}

Zum Beispiel NaCl, Diamant, etc.. Es gibt zwei Äste:
\begin{itemize}
\item \emph{akustischer Ast} (1 Zweig longitudinal akustisch (LA), 2 Zweige
transversal akustisch (TA))
\item \emph{optischer Ast} (1 Zweig longitudinal optisch (LO), 2 Zweige
transversal optisch (TO))
\end{itemize}
\textcolor{green}{TODO: Abb63}

Für $n$ Atome in der primitven Einheitszelle gibt es $3n$ Äste:
\begin{itemize}
\item 3 akustische
\item $3n-3$ optische
\end{itemize}
Betrachte nur solche Wellen, bei denen die Netzebenen nur jeweils
eine Atomsorte enthalten. Betrachte nur Wechselwirkungen zwischen
nächsten Nachbarn.

\textcolor{green}{TODO: Abb64}

Die Bewegungsgleichungen lauten:
\begin{align}
M_{1}\frac{\dd^{2}u_{s}}{\dd t^{2}} & =c\left(v_{s}+v_{s-1}-2u_{s}\right) & M_{2}\frac{\dd^{2}v_{s}}{\dd t^{2}} & =c\left(u_{s+1}+u_{s}-2v_{s}\right)\label{eq:Bewegungsgleichungen}
\end{align}
 Lösung in Form von laufenden Wellen mit verschiedenen Amplituden
$u$ und $v$ der aufeinander folgenden Netzebenen sind:
\begin{align}
u_{s} & =ue^{\ii\left(ska-\omega t\right)} & v_{s} & =v\cdot e^{\ii\left(ska-\omega t\right)}
\end{align}
Eingesetzt in (\ref{eq:Bewegungsgleichungen}) erhalte:
\begin{align*}
-\omega^{2}M_{1}u & =cv\left(1+e^{\ii ka}\right)-2cu\\
-\omega^{2}M_{2}v & =cu\left(1+e^{\ii ka}\right)-2cv
\end{align*}
\begin{align}
M_{1}\omega^{2}\left(\begin{array}{c}
u\\
v
\end{array}\right) & =\underbrace{\left(\begin{array}{cc}
2c & -c\left(1+e^{\ii ka}\right)\\
-\frac{M_{1}}{M_{2}}c\left(1+e^{\ii ka}\right) & 2\frac{M_{1}}{M_{2}}c
\end{array}\right)}_{\text{dynamische Matrix}}\left(\begin{array}{c}
u\\
v
\end{array}\right)
\end{align}
Die Koeffizientendeterminante muss verschwinden, damit es eine Lösung
gibt:
\begin{align*}
\abs{\begin{array}{cc}
2c-M_{1}\omega^{2} & -c\left(1+e^{\ii ka}\right)\\
-c\left(1+e^{\ii ka}\right) & 2c-M_{2}\omega^{2}
\end{array}} & \stackrel{!}{=}0\\
M_{1}M_{2}\omega^{4}-2c\left(M_{1}+M_{2}\right)\omega^{2}+2c^{2}\left(1-\cos\left(ka\right)\right) & =0
\end{align*}
\begin{align}
\fbox{\ensuremath{{\displaystyle \omega_{\pm}^{2}=\frac{C}{M_{1}M_{2}}\left(M_{1}+M_{2}\pm\sqrt{M_{1}^{2}+M_{2}^{2}+2M_{1}M_{2}\cos\left(ka\right)}\right)}}}
\end{align}
Dies gibt den optischen und den akustischen Ast.

\textcolor{green}{TODO: Abb65; Plot $\omega$ ($M_{2}<M_{1}$)}

\textcolor{green}{TODO: Abb66}

\textcolor{green}{TODO: Abb67}

%DATE: Do 6.12.12


\subsubsection*{Gitter mit Diamantstruktur (z.B. C, Si, Ge)}

\textcolor{green}{TODO: Abb68; LO und TO sind bei $\vec{k}=0$ entartet
(wegen Symmetrie des Gitters)}


\subsubsection*{Gitter mit NaCl-Struktur}

\textcolor{green}{TODO: Abb69; LO-TO-Aufspaltung bei $\vec{k}=0$
(Symmetrie)}


\subsubsection*{Gitter mit einatomiger Basis (Na, bcc)}

\textcolor{green}{TODO: Abb70}

Nur akustische Zweige!


\section{Quantisierung der Gitterschwingungen}

\begin{framed}

Die Energie der Gitterschwingungen ist gequantelt. Energiequant ist
das Quasi-Teilchen \emph{Phonon}.

\end{framed}

Die Energie eines Phonons der Frequenz $\omega$ ist:
\begin{align*}
\fbox{\ensuremath{{\displaystyle E=\hbar\omega}}}
\end{align*}
Ein Phonon mit Wellenvektor $\vec{k}$ tritt mit elektromagnetischer
Strahlung, Neutron oder Elektronen so in Wechselwirkung, als hätte
es einen Impuls.
\begin{align*}
\fbox{\ensuremath{{\displaystyle \vec{p}=\hbar\vec{k}}}}\\
\fbox{\ensuremath{{\displaystyle p=\frac{h}{\lambda}}}}\ \,
\end{align*}



\section{Experimentelle Bestimmung von Gitterschwingungen}
\begin{itemize}
\item Die Energie von Phononen liegt im fern-infraroten Spektralbereich
(Wärmestrahlung) $\mu\text{eV}\ldots100\,\text{meV}$. Die Wellenlänge
ist:
\begin{align*}
\lambda & \approx\mu\text{m}\ldots\text{mm}
\end{align*}

\item Die inelastische Streuung von Neutronen bzw. Licht heißt \emph{Ramanstreuung}
(optische Phononen) bzw. \emph{Brillouinstreuung} (akustische Phononen).
\item Beachte Impulserhaltung (Neutronen und Licht):
\begin{align*}
\vec{k}-\vec{k}' & =\pm\vec{q}
\end{align*}
\textcolor{green}{TODO: Abb71; $\vec{q}$: Wellenvektor des im Prozess
erzeugten ($+$) oder vernichteten ($-$) Phonons.}
\end{itemize}

\subsubsection*{Neutronen}

\begin{align*}
E & =\frac{h^{2}}{2m\lambda^{2}}
\end{align*}
Für $E=0,1\,\text{eV}$ ist $\lambda\approx3\,\angs$, also ist:
\begin{align*}
\norm{\vec{k}} & =\frac{2\pi}{\lambda}\approx2\cdot10^{10}\frac{1}{\text{m}}
\end{align*}
Dies ist die gleiche Größenordnung wie $\frac{\pi}{a}$. Also ist
die Messung in der gesamten Brillouin-Zone möglich, zum Beispiel bei
Si.


\subsubsection*{Photonen \textmd{(Ramanstreuung)}}

\begin{align*}
E & =h\frac{c}{\lambda}
\end{align*}
Für $1\,\text{eV}$, also $\lambda=1,2\,\mu\text{m}$ und somit:
\begin{align*}
\norm{\vec{k}} & =\frac{2\pi}{\lambda}\approx3\cdot10^{6}\frac{1}{\text{m}}\ll\frac{\pi}{a}
\end{align*}


\textcolor{green}{TODO: Abb72}

Typische Ramanspektren sehen wie folgt aus:

\textcolor{green}{TODO: Abb73}

\emph{Stokesprozess}: Ein Phonon wird im inelastischen Streuprozess
erzeugt.

\emph{Anti-Stokesprozess}: Ein Phonon wird vernichtet.


\subsection*{Einschub: Landau-Theorie}

Bei einem Phasenübergang zweiter Ordnung verschwindet der Ordnungsparameter,
bzw. die Ableitung divergiert.

\textcolor{green}{TODO: Abb74}

\begin{align*}
F & =F_{0}+a_{1}q+\frac{1}{2}a_{2}q^{2}+\frac{1}{3}a_{3}q^{3}+\frac{1}{4}a_{4}q^{4}+\ldots
\end{align*}
Wegen der Zeitumkehrinvarianz gilt $a_{2\nu+1}=0$. Mit der Eichung
$F_{0}=0$ erhalte:
\begin{align*}
F & \approx\frac{1}{2}a_{2}q^{2}+\frac{1}{4}a_{4}q^{4}
\end{align*}
\begin{align*}
\mathcal{F} & =\frac{\partial F}{\partial q}=a_{2}q+a_{4}q^{3}
\end{align*}


\textcolor{green}{TODO: Abb75}
\begin{align*}
a_{2} & =\frac{1}{c}\left(T-T_{c}\right)
\end{align*}
\begin{align*}
q_{s}^{2} & =\begin{cases}
-\frac{a_{2}}{a_{4}} & T<T_{c}\\
0 & T>T_{c}
\end{cases}
\end{align*}


%DATE: Mi 12.12.12


\chapter{Thermische Eigenschaften}

Spezifische Wärme des Gitters:
\begin{align}
\fbox{\ensuremath{{\displaystyle c_{v}=\frac{1}{v}\frac{\partial E\left(T,V\right)}{\partial T}}}}
\end{align}
Experimentelle Fakten:
\begin{enumerate}
\item Bei Raumtemperatur gilt für fast alle Festkörper das \emph{Dulong-Petit-Gesetz}:
\begin{align*}
c_{V} & =3k_{B}\approx25\frac{\text{g}}{\text{mol}\cdot\text{K}}
\end{align*}

\item Bei tiefen Temperaturen geht $c_{V}$ bei \emph{Isolatoren} wie $T^{3}$,
bei \emph{Metallen} wie $T$ gegen Null.\\
\textcolor{green}{TODO: Abb76}
\item Bei Phasenübergängen findet man große Beiträge zu $c_{V}$.\\
Isolatoren: Phononen liefern den Hauptbeitrag zu $c_{V}$.\\
Metalle: Hier liefern die freien Elektronen einen zusätzlichen Beitrag.
\end{enumerate}
Die Besetzungszahl im thermischen Gleichegewicht bei Temperatur $T$
für Phononen (harmonische Oszillatoren) verhält sich gemäß der \emph{Planck-Verteilung}:
\begin{align}
\left\langle n\right\rangle  & =\frac{\sum_{s}se^{-\frac{s\hbar\omega}{k_{B}T}}}{\sum_{s}e^{-\frac{s\hbar\omega}{k_{B}T}}}=\frac{1}{e^{\frac{\hbar\omega}{k_{B}T}}-1}\label{eq:Planck-Verteilung}
\end{align}
$\left\langle n\right\rangle $ ist die mittlere Quantenzahl eines
angeregten Oszillators.

\textcolor{green}{TODO: Abb77}

Dies ist ein Ensemble harmonischer Oszillatoren mit Energien $n\hbar\omega$
für $n\in\mathbb{N}_{0}$.

Beachte: Durch Berücksichtigung der Nullpunktsenergie
\begin{align*}
E & =\hbar\omega\left(n+\frac{1}{2}\right)
\end{align*}
des harmonischen Oszillators ergibt sich als thermisch gemittelte
Quantenzahl $\left\langle n\right\rangle +\frac{1}{2}$.


\section{Einstein-Modell}
\begin{itemize}
\item Betrachte Einkristalle aus $N_{A}$ Atomen, also $3N_{A}$ Oszillatoren
(Phononenzustände).
\item Annahme: Alle Phononen haben dieselbe Frequenz $\omega$!\\
\textcolor{green}{TODO: Abb78}
\item Dieses Modell approximiert den Beitrag des optischen Astes!
\item Gesamtenergie:
\begin{align}
E & =N_{a}\left\langle n\right\rangle \hbar\omega=\frac{N_{A}\hbar\omega}{e^{\frac{\hbar\omega}{k_{B}T}}-1}
\end{align}
\begin{align}
\mbox{\fbox{\ensuremath{{\displaystyle c_{V}=\frac{\partial E\left(T,V\right)}{\partial T}=3k_{B}\left(\frac{\hbar\omega}{k_{B}T}\right)^{2}\frac{e^{\frac{\hbar\omega}{k_{B}T}}}{\left(e^{\frac{\hbar\omega}{k_{B}T}}-1\right)^{2}}}}}}
\end{align}
Für hohe Temperaturen konvergiert $c_{V}\to3k_{B}$, was das Dulong-Petit
Gesetz ergibt.
\end{itemize}

\section{Debye-Modell}
\begin{itemize}
\item Betrachtung von Oszillatoren \emph{verschiedener} Frequenz:
\begin{align}
E & =\sum_{i}n_{i}\hbar\omega_{i}
\end{align}

\item Liegen die Frequenzen dicht beisammen, so gilt:
\begin{align}
E & =\int D\left(\omega\right)n\left(\omega,T\right)\hbar\omega\dd\omega
\end{align}
Dabei ist $D\left(\omega\right)\dd\omega$ die Anzahl der Schwingungszustände
im Intervall $\left(\omega,\dd\omega\right)$. $D\left(\omega\right)$
ist die Zustandsdichte.\\
Betrachte zunächst periodische Randbedingungen in einer Dimension.\\
\textcolor{green}{TODO: Abb79; $N=8$, $L=N\cdot a=8a$}
\begin{align*}
u_{s} & \sim e^{\ii ska}
\end{align*}
(Es gibt sowohl longitudinale als auch transversale Schwingungen.)\\
Es muss gelten:
\begin{align*}
u_{s+N} & \stackrel{!}{=}u_{s}
\end{align*}
\begin{align*}
\Rightarrow\quad e^{\ii\left(s+N\right)ka} & =e^{\ii ska}
\end{align*}
Für $n\in\mathbb{Z}$ folgt somit:
\begin{align*}
\Rightarrow\quad Nka & =n\cdot2\pi
\end{align*}
\textcolor{green}{TODO: Abb80}


\begin{framed}Es gibt genau einen erlaubten wert von $k$ pro Länge
$\frac{2\pi}{L}$ im reziproken Raum.\end{framed}

\end{itemize}
Zustandsdichte in drei Dimensionen:

Betrachte einen Würfel mit Kantenlänge $L$ und $N^{3}$ primitiven
Einheitszellen mit einatomiger Basis. Nehme periodische Randbedingungen
an:
\begin{align}
\exp\left(\ii\left(k_{x}x+k_{y}y+k_{z}z\right)\right) & =\exp\left(\ii\left(k_{x}\left(x+L\right)+k_{y}\left(y+L\right)+k_{z}\left(z+L\right)\right)\right)
\end{align}
Also gilt für $N\in2\mathbb{N}$:
\begin{align}
k_{x},k_{y},k_{z} & \in\left\{ 0,\pm\frac{2\pi}{L},\pm\frac{4\pi}{L},\ldots,\pm\frac{N\pi}{L}\right\} 
\end{align}


\begin{framed}

Es gibt im reziproken Raum genau einen erlaubten Wert von $\vec{k}$
pro Volumen ${\displaystyle \left(\frac{2\pi}{L}\right)^{3}}$.

\end{framed}

\textcolor{green}{TODO: Abb81}


\subsubsection*{Debye-Näherung}
\begin{itemize}
\item Nehme an, dass die Schallgeschwindigkeiten konstant seien, das heißt
$\omega=v_{s}\cdot k$ gilt, was den akustischen Zweig für $k\to0$
approximiert. Dies gilt nur für Kristalle mit einatomiger Basis.
\item Die Gesamtzahl von Zuständen mit Wellenvektor $\norm{\vec{k}'}<\norm{\vec{k}}$
(pro Polarisation) ist:
\begin{align}
N & =\frac{\frac{4\pi}{3}k^{3}}{\left(\frac{2\pi}{L}\right)^{3}}=\left(\frac{L}{2\pi}\right)^{3}\cdot\frac{4\pi}{3}\frac{\omega^{3}}{v_{s}^{3}}
\end{align}
Für die Zustandsdichte ergibt sich:
\begin{align}
D\left(\omega\right) & =\frac{\dd N}{\dd\omega}=\frac{V\omega^{2}}{2\pi^{2}v_{s}^{3}}\sim\omega^{2}\label{eq:Zustandsdichte}
\end{align}
Dabei ist $V=L^{3}$.\\
\textcolor{green}{TODO: Abb82}
\item \emph{Definition einer Abschneidefrequenz $\omega_{D}$}: Besteht
ein Kristall aus $N$ primitiven Einheitszellen ($N$-Atome, einatomige
Basis), so gibt es $N$ akustische Schwingungszustände ($3N$ für
alle 3 Polarisationen).
\begin{align*}
N & \stackrel{!}{=}\int_{0}^{\omega_{D}}D\left(\omega\right)\dd\omega=\int_{0}^{\omega_{D}}\frac{V\omega^{2}}{2\pi^{2}v_{s}^{3}}\dd\omega=\frac{V\omega_{D}^{3}}{6\pi^{2}v_{s}^{3}}
\end{align*}
So erhält man die \emph{Debye-Frequenz} 
\begin{align}
\omega_{D}^{3} & :=\frac{6\pi^{2}v_{s}^{3}}{V}N
\end{align}
und den \emph{Debye-Wellenvektor}:
\begin{align}
k_{D} & =\frac{\omega_{D}}{v_{s}}=\left(\frac{6\pi^{2}N}{V}\right)^{\frac{1}{3}}
\end{align}
\textcolor{green}{TODO: Abb83}
\begin{align*}
\omega & =v_{s}k
\end{align*}
\begin{framed}Im Debye-Modell sind keine Schwingungszustände zugelassen,
für die $k>k_{D}$ ist.\end{framed}
\item Beispiel: NaCl\\
\textcolor{green}{TODO: Abb84}
\end{itemize}

\section{Spezifische Wärme im Debye-Modell}

Thermische Energie pro Polarisation (siehe (\ref{eq:Planck-Verteilung})
und (\ref{eq:Zustandsdichte})):
\begin{align}
E & =\int D\left(\omega\right)n\left(\omega,T\right)\hbar\omega\dd\omega=\int_{0}^{\omega_{D}}\frac{V\omega^{2}}{2\pi^{2}v_{s}^{3}}\cdot\frac{\hbar\omega}{e^{\frac{\hbar\omega}{k_{B}T}}-1}\dd\omega
\end{align}
Falls $v_{s}$ für jede Polarisation gleich ist, folgt:
\begin{align*}
E & =\frac{3V\hbar}{2\pi^{2}v_{s}^{3}}\int_{0}^{\omega_{D}}\frac{\omega^{3}}{e^{\frac{\hbar\omega}{k_{B}T}}-1}\dd\omega=\\
 & \sr ={x=\frac{\hbar\omega}{k_{B}T}}{\dd\omega=\frac{k_{B}T}{\hbar}\dd x}\frac{3V\hbar}{2\pi^{2}v_{s}^{3}}\left(\frac{k_{B}T}{\hbar}\right)^{4}\int_{0}^{\frac{\hbar\omega_{D}}{k_{B}T}}\frac{x^{3}}{e^{x}-1}\dd x=\\
 & =\frac{3V\omega_{D}^{3}}{2\pi^{2}v_{s}^{3}}k_{B}T\left(\frac{k_{B}T}{\hbar\omega_{D}}\right)^{3}\int_{0}^{\frac{\hbar\omega_{D}}{k_{B}T}}\frac{x^{3}}{e^{x}-1}\dd x\\
 & \sr ={\omega_{D}=\frac{6\pi^{2}v_{s}^{3}}{V}N}{}9Nk_{B}T\left(\frac{k_{B}T}{\hbar\omega_{D}}\right)^{3}\int_{0}^{\frac{\hbar\omega_{D}}{k_{B}T}}\frac{x^{3}}{e^{x}-1}\dd x
\end{align*}
\begin{align}
\fbox{\ensuremath{{\displaystyle E=9Nk_{B}T\left(\frac{T}{\theta_{D}}\right)^{3}\int_{0}^{x_{D}}\dd x\frac{x^{3}}{e^{x}-1}}}}\label{eq:E-Debye}
\end{align}
Dabei ist $\theta_{D}=\frac{\hbar\omega_{D}}{k_{B}}$ die Debye-Temperatur
und $x_{D}=\frac{\theta_{D}}{T}$.
\begin{align*}
C_{V} & =9Nk_{B}\frac{\partial}{\partial T}\left(T\left(\frac{T}{\theta_{D}}\right)^{3}\int_{0}^{x_{D}}\dd x\frac{x^{3}}{e^{x}-1}\right)=\\
 & =9Nk_{B}\left(4\left(\frac{T}{\theta_{D}}\right)^{3}\int_{0}^{x_{D}}\dd x\frac{x^{3}}{e^{x}-1}+T\left(\frac{T}{\theta_{D}}\right)^{3}\frac{\partial}{\partial T}\int_{0}^{x_{D}}\dd x\frac{x^{3}}{e^{x}-1}\right)=\\
 & =9Nk_{B}\left(4\left(\frac{T}{\theta_{D}}\right)^{3}\int_{0}^{x_{D}}\dd x\frac{x^{3}}{e^{x}-1}+T\left(\frac{T}{\theta_{D}}\right)^{3}\frac{x_{D}^{3}}{e^{x_{D}}-1}\cdot\frac{-\Theta_{D}}{T^{2}}\right)=\\
 & =9Nk_{B}\left(\frac{T}{\theta_{D}}\right)^{3}\left(4\int_{0}^{x_{D}}\dd x\frac{x^{3}}{e^{x}-1}-\frac{x_{D}^{4}}{e^{x_{D}}-1}\right)=\\
 & =9Nk_{B}\left(\frac{T}{\theta_{D}}\right)^{3}\left(4\int_{0}^{x_{D}}\dd x\frac{x^{3}}{e^{x}-1}-\int_{0}^{x_{D}}\dd x\frac{\left(e^{x}-1\right)\cdot4x^{3}-x^{4}e^{x}}{\left(e^{x}-1\right)^{2}}\right)=\\
 & =9Nk_{B}\left(\frac{T}{\theta_{D}}\right)^{3}\int_{0}^{x_{D}}\dd x\frac{x^{4}e^{x}}{\left(e^{x}-1\right)^{2}}
\end{align*}
\begin{align}
\fbox{\ensuremath{{\displaystyle C_{V}=\frac{\partial E\left(T,V\right)}{\partial T}=9Nk_{B}\left(\frac{T}{\theta_{D}}\right)^{3}\int_{0}^{x_{D}}\dd x\frac{x^{4}e^{x}}{\left(e^{x}-1\right)^{2}}}}}
\end{align}



\subsubsection*{Debyesches $T^{3}$-Gesetz}

Für tiefe Temperaturen gilt $x_{D}=\frac{\theta_{D}}{T}\xrightarrow{T\to0}\infty$
und:
\begin{align*}
\int_{0}^{\infty}\dd x\frac{x^{3}}{e^{x}-1} & =\int_{0}^{\infty}\dd x\cdot x^{3}\left(\sum_{s=1}^{\infty}e^{-sx}\right)=\frac{\pi^{4}}{15}
\end{align*}
Mit (\ref{eq:E-Debye}) folgt für $T\ll\theta_{D}$:
\begin{align*}
E & =\frac{3\pi^{4}}{5}\frac{Nk_{B}T^{4}}{\theta_{D}^{3}}
\end{align*}
Damit ergibt sich das Debyesche $T^{3}$-Gesetz:
\begin{align}
\fbox{\ensuremath{{\displaystyle C_{V}=\frac{12\pi^{4}}{5}N_{A}k_{B}\left(\frac{T}{\theta_{D}}\right)^{3}\approx234N_{A}k_{B}\left(\frac{T}{\theta_{D}}\right)^{3}}}}
\end{align}
Beispiele:

\noindent \begin{center}
\begin{tabular}{c|c}
Element$\vphantom{\Big|}$ & $\theta_{D}\text{ in K}$\tabularnewline
\hline 
\hline 
Cs$\vphantom{\Big|}$ & 38\tabularnewline
\hline 
Pb$\vphantom{\Big|}$ & 90\tabularnewline
\hline 
As$\vphantom{\Big|}$ & 226\tabularnewline
\hline 
Cu$\vphantom{\Big|}$ & 343\tabularnewline
\hline 
Al$\vphantom{\Big|}$ & 390\tabularnewline
\hline 
Fe$\vphantom{\Big|}$ & 467\tabularnewline
\hline 
Be$\vphantom{\Big|}$ & 1440\tabularnewline
\hline 
Diamant$\vphantom{\Big|}$ & 2230\tabularnewline
\end{tabular}
\par\end{center}

Anmerkung: Die Spezifische Wärme „freier Elektronen“ ergibt sich wie
folgt:

\begin{framed}

Elektronen sind Fermionen, weswegen die Fermi-Dirac-Statistik zu verwenden
ist!

\end{framed}
\begin{align}
E & =\int D\left(\varepsilon\right)f\left(\varepsilon,T\right)\varepsilon\dd\varepsilon\sim T^{2}
\end{align}
$f\left(\varepsilon,T\right)$ ist die Fermi-Dirac-Statistik.
\begin{align}
\fbox{\ensuremath{{\displaystyle c_{V}=\gamma T}}}
\end{align}
\begin{align}
c_{V} & =c_{V}^{\text{lattice}}+c_{V}^{\text{electr.}}=\alpha T^{3}+\gamma T
\end{align}


%DATE: Do 13.12.12


\section{Anharmonische Wechselwirkungen in Kristallen}

In harmonischer Näherung gilt:
\begin{enumerate}
\item Es gibt keine Wärmeausdehnung:
\begin{align*}
l & =l_{0}\left(1+\alpha T\right) & \alpha & =0
\end{align*}

\item $c_{V}$ wird bei hohen Temperaturen konstant.
\item Elastische Konstanten sind von Temperatur und Druck unabhängig.
\item Phononen beeinflussen sich nicht gegenseitig, das heißt es gibt keine
Dämpfung und keine Phonon-Phonon-Streuung.
\end{enumerate}
Beispiel thermische Ausdehnung: Betrachte das Morsepotential für das
$\text{H}_{2}^{+}$-Ion.

\textcolor{green}{TODO: Abb85}

Analog betrachte anharmonische Terme im Potential $V\left(x\right)$
des Festkörpers:
\begin{align*}
V\left(x\right) & =cx^{2}-gx^{3}-fx^{4}
\end{align*}
$x$ ist die Auslenkung aus der Ruhelage.
\begin{align}
\left\langle x\right\rangle  & =\frac{\int xe^{-\frac{V\left(x\right)}{k_{B}T}}\dd x}{\int e^{-\frac{V\left(x\right)}{k_{B}T}}\dd x}
\end{align}
Für hohe Temperaturen gilt:
\begin{align}
\int xe^{-\frac{V\left(x\right)}{k_{B}T}}\dd x & =\int e^{-\frac{cx^{2}}{k_{B}T}}xe^{\frac{gx^{3}+fx^{4}}{k_{B}T}}\dd x\approx\int e^{-\frac{cx^{2}}{k_{B}T}}x\left(1+\frac{gx^{3}}{k_{B}T}+\frac{fx^{4}}{k_{B}T}\right)\dd x=\nonumber \\
 & =\int e^{-\frac{cx^{2}}{k_{B}T}}\left(x+g\frac{x^{4}}{k_{B}T}+f\frac{x^{5}}{k_{B}T}\right)\dd x=\nonumber \\
 & =\frac{g}{k_{B}T}\left(\frac{k_{B}T}{c}\right)^{\frac{5}{2}}\cdot\frac{3\sqrt{\pi}}{4}\label{eq:Integral1}
\end{align}
\begin{align}
\int e^{-\frac{V\left(x\right)}{k_{B}T}}\dd x & \approx\int e^{-\frac{cx^{2}}{k_{B}T}}\left(1+\frac{gx^{3}}{k_{B}T}+\frac{fx^{4}}{k_{B}T}\right)\dd x\approx\int e^{-\frac{cx^{2}}{k_{B}T}}\dd x=\sqrt{\frac{\pi k_{B}T}{c}}\label{eq:Integral2}
\end{align}
Aus (\ref{eq:Integral1}) und (\ref{eq:Integral2}) folgt:
\begin{align}
\left\langle x\right\rangle  & =\frac{3g}{4c^{2}}k_{B}T\sim T
\end{align}
Makroskopisch ist die Längenausdehnung linear in $T$:
\begin{align}
\fbox{\ensuremath{{\displaystyle l\left(T\right)=l_{0}\left(1+\alpha T\right)}}}
\end{align}
$l_{0}$ ist die Länge bei $T=0$ und $\alpha$ ist der Wärmeausdehnungskoeffizient.


\subsubsection*{Wärmeleitung im Festkörper}
\begin{itemize}
\item Das anharmonisches Potential erlaubt Stöße zwischen den Phononen.
Also gibt es eine \emph{mittlere freie Weglänge} $\lambda$ für die
Phononen.
\item Für ein stationäre Temperaturverteilung $T\left(x\right)$ ergibt
sich ein Wärmestrom:
\begin{align*}
\vec{j} & =-\kappa\vec{\nabla}T
\end{align*}
$\kappa$ ist die Wärmeleitfähigkeit. Vergleiche dies mit dem Ohmschen
Gesetz:
\begin{align*}
\vec{j} & =\sigma\vec{E}=-\sigma\vec{\nabla}\phi
\end{align*}
$\sigma$ ist die elektrische Leitfähigkeit.
\item Aus der kinetischen Gastheorie erhält man für die Wärmeleitfähigkeit
eines Gases:
\begin{align*}
\kappa & =\frac{1}{3}c\overline{v}\lambda
\end{align*}
Dabei ist $c$ die spezifische Wärmekapazität, $\overline{v}$ die
mittlere Teilchengeschwindigkeit und $\lambda$ die mittlere freie
Weglänge.
\item Für einen Festkörper nach dem Debye-Modell ist $\overline{v}$ die
Phononengeschwindigkeit und $\lambda$ die mittlere freie Weglänge.
\end{itemize}

\chapter{Elektronen im Festkörper}


\section{Das freie Elektronengas}

Kräfte zwischen Leitungselektronen und Ionenrümpfen sowie zwischen
den Elektronen werden vernachlässigt. Ein Elektron breitet sich als
Materiewelle im Gitter ungehindert aus.
\begin{itemize}
\item Das \emph{klassisches} freie Elektronengas dient zur Erklärung:

\begin{itemize}
\item des Ohmschen Gesetzes.
\item des Zusammenhangs zwischen elektrischer und thermischer Leitfähigkeit.
\end{itemize}
\item Das \emph{Fermi-Gas} freier Elektronen verwendet man zur Erklärung:

\begin{itemize}
\item der spezifischen Wärme der Elektronen.
\item der magnetischen Suszeptibilität der Elektronen.
\end{itemize}
\item Einfache Metalle: Alkalimetalle (1 Valenzelektron), Erdalkalimetalle
(2 Valenzelektronen), Aluminium (3 Valenzelektronen), Gallium (3 Valenzelektronen),
Indium (3 Valenzelektronen), Blei (4 Valenzelektronen)
\item „Edelmetalle“: Kupfer, Silber, Gold
\item Übergangsmetalle (z.B. Eisen, Kobalt, Nickel), Lanthanide, Aktinide
\end{itemize}

\subsubsection*{Energieniveaus und Zustandsdichte des Fermigases}

Ein Elektron der Masse $m$ befinde sich im Würfel mit Kantenlänge
$L$. Die Schrödingergleichung lautet:
\begin{align}
-\frac{\hbar^{2}}{2m}\left(\frac{\partial^{2}}{\partial x^{2}}+\frac{\partial^{2}}{\partial y^{2}}+\frac{\partial^{2}}{\partial z^{2}}\right)\psi_{\vec{k}}\left(\vec{r}\right) & =E_{\vec{k}}\psi_{\vec{k}}\left(\vec{r}\right)\label{eq:Schroedinger-Gleichung}
\end{align}
Randbedingungen:
\begin{enumerate}[label=\alph*)]
\item Entweder verschwindet die Wellenfunktion außerhalb des Würfels, was
zu stehenden Wellen führt,
\item oder man hat periodische Randbedingungen: Das Elektronengas ist unendlich
ausgedehnt, aber die Wellenfunktion wiederholt sich nach $L$ in Richtung
$x,y$ und $z$.
\begin{align}
\psi_{\vec{k}}\left(x+L,y,z\right) & =\psi_{\vec{k}}\left(x,y,z\right)\quad\text{etc.}
\end{align}
Die laufenden Wellen sind:
\begin{align}
\fbox{\ensuremath{{\displaystyle \psi_{\vec{k}}\left(\vec{r}\right)\sim e^{\ii\vec{k}\vec{r}}}}}
\end{align}
Dabei sind $k_{x},k_{y},k_{z}\in\left\{ 0,\pm\frac{2\pi}{L},\pm\frac{4\pi}{L},\ldots\right\} $.\end{enumerate}
\begin{itemize}
\item Quantenzahlen sind die Komponenten von $\vec{k}$ und der Spin $m_{s}=\pm\frac{1}{2}$.\\
Einsetzen in (\ref{eq:Schroedinger-Gleichung}) liefert:
\begin{align}
\fbox{\ensuremath{{\displaystyle E_{\vec{k}}=\frac{\hbar^{2}}{2m}\vec{k}^{2}=\frac{\hbar^{2}}{2m}\left(k_{x}^{2}+k_{y}^{2}+k_{z}^{2}\right)}}}
\end{align}
Beachte:
\begin{align}
\vec{p}\psi_{\vec{k}}\left(\vec{r}\right) & =-\ii\hbar\vec{\nabla}\psi_{\vec{k}}\left(\vec{r}\right)=\hbar\vec{k}\psi_{\vec{k}}\left(\vec{r}\right)
\end{align}
Also ist die ebene Welle $\psi_{\vec{k}}$ eine Eigenfunktion des
Impulses $\vec{p}$ mit Eigenwert $\hbar\vec{k}$.
\item Teilchengeschwindigkeit im Zustand $\vec{k}$:
\begin{align}
\vec{v} & =\frac{\hbar\vec{k}}{m}
\end{align}

\item Bei $N$ Elektronen erfolgt die Besetzung der erlaubten Zustände nach
der Fermi-Dirac-Verteilung (Pauli-Prinzip!).
\begin{align}
f\left(E,T\right) & =\frac{1}{e^{\frac{E-\mu}{k_{B}T}}+1}
\end{align}
Dabei ist $\mu$ das chemische Potential.
\item Das chemische Potential $\mu$ wird so gewählt, dass die Gesamtzahl
der Teilchen gleich $N$ ist.\\
\textcolor{green}{TODO: Abb86}
\item Konvention: Für $T=0$ heißt $E_{F}:=\mu\left(T=0\right)$ die Fermi-Energie\\
Also liegen für $T=0$ die besetzten Zustände in der Fermi-Kugel mit
Radius $k_{F}$. $k_{F}$ heißt \emph{Fermi-Wellenvektor}:
\begin{align}
E_{F} & =\frac{\hbar^{2}}{2m}k_{F}^{2}
\end{align}

\end{itemize}
%DATE: Mi 19.12.12
\begin{itemize}
\item Mit dem Volumen $\left(\frac{2\pi}{L}\right)^{3}$ eines Zustands
im $k$-Raum folgt:
\begin{align}
N & =2\cdot\frac{\frac{4\pi}{3}k_{F}^{3}}{\left(\frac{2\pi}{L}\right)^{3}}=\frac{V}{3\pi^{2}}k_{F}^{3}
\end{align}
\begin{align}
k_{F} & =\left(\frac{3\pi^{2}N}{V}\right)^{\frac{1}{3}}\nonumber \\
E_{F} & =\frac{\hbar^{2}}{2m}\left(\frac{3\pi^{2}N}{V}\right)^{\frac{2}{3}}\\
v_{F} & =\frac{\hbar}{m}\left(\frac{3\pi^{2}N}{V}\right)^{\frac{1}{3}}=\frac{\hbar k_{F}}{m}\nonumber 
\end{align}
$v_{F}$ ist die Geschwindigkeit der Elektronen auf der Fermioberfläche.
\begin{align}
N & =\frac{V}{3\pi^{2}}\left(\frac{2mE_{F}}{\hbar^{2}}\right)^{\frac{3}{2}}\label{eq:N(E)}
\end{align}
\begin{align}
\fbox{\ensuremath{{\displaystyle D\left(E\right)=\frac{\dd N}{\dd E}=\frac{V}{2\pi^{2}}\cdot\left(\frac{2m}{\hbar^{2}}\right)^{\frac{3}{2}}\sqrt{E}}}}\label{eq:D(E)}
\end{align}
\textcolor{green}{TODO: Abb87; Zustandsdichte eines Fermi-Gases}
\item Aus (\ref{eq:N(E)}) und (\ref{eq:D(E)}) folgt:
\begin{align}
\fbox{\ensuremath{{\displaystyle D\left(E_{F}\right)=\frac{3}{2}\frac{N}{E_{F}}}}}
\end{align}

\end{itemize}

\section{Spezifische Wärme der Elektronen}
\begin{itemize}
\item Gesamtenergie (innere Energie):
\begin{align}
U & =N\left\langle E\right\rangle =\int_{0}^{\infty}D\left(E\right)\cdot f\left(E,T\right)E\dd E
\end{align}
Dabei gilt:
\begin{align*}
N & =\int_{0}^{\infty}D\left(E\right)\cdot f\left(E,T\right)\dd E=\text{konst.}
\end{align*}

\item Für tiefe Temperaturen ist $\mu\approx E_{F}$.
\begin{align}
U & =\frac{V}{2\pi^{2}}\left(\frac{2m}{\hbar^{2}}\right)^{\frac{3}{2}}\int_{0}^{\infty}\frac{E^{\frac{3}{2}}}{e^{\frac{E-E_{F}}{k_{B}T}+1}}\dd E
\end{align}
Dies ist analytisch nicht lösbar!
\item Nähere:
\begin{align}
U & =\int_{0}^{\infty}Ef\left(E,T\right)D\left(E\right)\dd E=\nonumber \\
 & =\int_{0}^{\infty}\left(E-E_{F}\right)f\left(E,T\right)D\left(E\right)\dd E+\underbrace{\int_{0}^{\infty}E_{F}f\left(E,T\right)D\left(E\right)\dd E}_{=N\cdot E_{F}}
\end{align}
\begin{align}
C_{V} & =\frac{\partial U\left(T,V\right)}{\partial T}=\int_{0}^{\infty}\left(E-E_{F}\right)D\left(E\right)\frac{\partial f\left(E,T\right)}{\partial T}\dd E
\end{align}

\item Für $k_{B}T\ll E_{F}$ ist $\frac{\partial f}{\partial T}\not=0$
nur für $E\approx E_{F}$.
\begin{align}
C_{V} & \sr{\approx}{k_{B}T\ll E_{F}}{\mu\approx E_{F}}D\left(E_{F}\right)\int_{-\infty}^{\infty}\frac{\left(E-E_{F}\right)^{2}}{k_{B}T^{2}}\cdot\frac{e^{\frac{E-E_{F}}{k_{B}T}}}{\left(e^{\frac{E-E_{F}}{k_{B}T}}+1\right)^{2}}\dd E
\end{align}
Substitution:
\begin{align*}
x & :=\frac{E-E_{F}}{k_{B}T} & \dd E & :=k_{B}T\dd x
\end{align*}
\begin{align*}
C_{V} & \approx D\left(E_{F}\right)k_{B}^{2}T\underbrace{\int_{-\infty}^{\infty}x^{2}\frac{e^{x}}{\left(e^{x}+1\right)^{2}}\dd x}_{=\frac{\pi^{2}}{3}}=D\left(E_{F}\right)k_{B}^{2}T\frac{\pi^{2}}{3}=\\
 & =\frac{V}{6}\cdot k_{B}^{2}T\left(\frac{2m}{\hbar^{2}}\right)^{\frac{3}{2}}\sqrt{E_{F}}=\frac{V}{6}\cdot k_{B}^{2}T\left(\frac{2m}{\hbar^{2}}\right)^{\frac{3}{2}}\left(\frac{\hbar^{2}}{2m}\left(\frac{3\pi^{2}N}{V}\right)^{\frac{2}{3}}\right)^{\frac{3}{2}}\frac{1}{E_{F}}=\\
 & =\frac{V}{6}\cdot k_{B}^{2}T\cdot\left(\frac{3\pi^{2}N}{V}\right)\cdot\frac{1}{E_{F}}\stackrel{E_{F}=k_{B}T_{F}}{=}\frac{\pi^{2}}{2}\cdot k_{B}N\cdot\frac{T}{T_{F}}
\end{align*}
\begin{align}
\fbox{\ensuremath{{\displaystyle c_{V}^{\text{elektrisch}}=\frac{1}{\nu}\frac{\partial U\left(T,V\right)}{\partial T}=\frac{\pi^{2}}{2}N_{A}k_{B}\frac{T}{T_{F}}=\gamma T}}}
\end{align}

\item Insgesamt gilt:
\begin{align*}
c_{V} & =c_{V}^{\text{Gitter}}+c_{V}^{\text{Elektronen}}
\end{align*}

\end{itemize}

\section{Elektrische Leitfähigkeit und Ohmsches Gesetz}
\begin{itemize}
\item Betrachte Elektronen wie freie, klassische Teilchen (Drude-Modell):
\begin{align}
\vec{p} & =m\vec{v}=\hbar\vec{k}\\
\vec{F} & =m\dot{\vec{v}}=\hbar\dot{\vec{k}}=-e\left(\vec{E}+\vec{v}\times\vec{B}\right)
\end{align}

\item $\vec{E}$-Feld bewirkt gleichförmige Verschiebung der Fermi-Kugel.
($\vec{B}=0$)\\
\textcolor{green}{TODO: Abb88}
\begin{align}
\vec{k}\left(t\right) & =\vec{k}\left(0\right)-\frac{e\vec{E}\cdot t}{\hbar}
\end{align}

\item Kurze Zeit $\tau$ nach dem Anlegen des Feldes bei $t>0$ gilt
\begin{align}
\delta\vec{k} & =-\frac{e\vec{E}\tau}{\hbar}
\end{align}
und es erfolgen Stöße der Elektronen untereinander, mit Verunreinigungen,
Gitterfehlern und Phononen. Dies führt zu einer Verschiebung der Fermikugel
als stationärem Zustand!
\item Es gilt:
\begin{align}
\vec{v} & =\frac{\vec{p}}{m}=\frac{\hbar\delta\vec{k}}{m}=-\frac{e\vec{E}\tau}{m}\label{eq:Driftgeschwindigkeit}
\end{align}
Stromdichte:
\begin{align}
\vec{j} & =-ne\vec{v}=\frac{ne^{2}\tau}{m}\cdot\vec{E}=\sigma\vec{E}
\end{align}
Elektrische Leitfähigkeit:
\begin{align}
\fbox{\ensuremath{{\displaystyle \sigma=\frac{ne^{2}\tau}{m}}}}\label{eq:spezifische-leitfaehigkeit}
\end{align}

\item Beachte: Gleichung (\ref{eq:spezifische-leitfaehigkeit}) folgt aus
der klassischen Theorie (P. Drude) unter der Annahme, dass alle Elektronen
an Streuprozessen teilnehmen und $\tau$ die mittlere Stoßzeit ist.
\item Eine Beschreibung durch ein Fermigas mit Streuprozessen ergibt:
\begin{align}
\fbox{\ensuremath{{\displaystyle \sigma=\frac{ne^{2}\tau\left(E_{F}\right)}{m}}}}\label{eq:elektrische-Leitfaehigkeit}
\end{align}
Das heißt nur die Stoßzeit $\tau\left(E_{F}\right)$ der Elektronen
an der Fermioberfläche spielen eine Rolle.
\item Die mittlere freie Weglänge $l$ eines Leitungselektrons ist:
\begin{align}
l & =v_{F}\cdot\tau
\end{align}
(Nur Elektronen am Ferminiveau tragen bei!)
\item Beispiel Kupfer:
\begin{align*}
l_{\text{Cu}}\left(300\,\text{K}\right) & =3\cdot10^{-6}\,\text{cm}\\
l_{\text{Cu}}\left(4\,\text{K}\right) & =3\,\text{mm}
\end{align*}
In reinen Metallen können die mittleren freien Weglängen bei tiefen
Temperaturen bis zu $10\,\text{cm}$ betragen!
\end{itemize}

\section{Bewegungen in Magnetfeldern, Hall-Effekt}
\begin{itemize}
\item Drude-Ansatz: Newtonsche Bewegungsgleichung mit Reibungsterm
\begin{align}
\fbox{\ensuremath{{\displaystyle m\dot{\vec{v}}+\underbrace{\frac{m\vec{v}}{\tau}}_{\sr{}{\text{Reibungsterm}}{\text{modelliert Stöße}}}=-e\left(\vec{E}+\vec{v}\times\vec{B}\right)}}}
\end{align}

\item Nehme ein statisches $\vec{B}$-Feld entlang der $z$-Achse an, also
$\vec{B}=\left(0,0,B_{z}\right)$. Bei einem stationären Zustand gilt
$\dot{\vec{v}}=0$.
\begin{align}
v_{x} & =-\frac{e\tau}{m}E_{x}-\frac{eB\tau}{m}v_{y}\nonumber \\
v_{y} & =-\frac{e\tau}{m}E_{y}+\frac{eB\tau}{m}v_{x}\label{eq:DGL-Hall}\\
v_{z} & =-\frac{e\tau}{m}E_{x}\nonumber 
\end{align}

\item Hall-Effekt: Das elektrische Feld $\vec{E}_{H}$, \emph{Hall-Feld}
genannt, zeigt in Richtung $\vec{j}\times\vec{B}$.\\
\textcolor{green}{TODO: Abb89}\\
Annahme: Der Strom fließe in $x$-Richtung, d.h. $v_{y}=0$. Aus (\ref{eq:DGL-Hall})
folgt:
\begin{align}
v_{x} & =-\frac{e\tau}{m}E_{x}
\end{align}
Mit (\ref{eq:elektrische-Leitfaehigkeit}) erhält man:
\begin{align}
j_{x} & =-env_{x}=\frac{ne^{2}\tau}{m}E_{x}
\end{align}
Die Hall-Konstante ist definiert als:
\begin{align}
R_{H} & =\frac{E_{y}}{j_{x}\cdot B_{z}}=\frac{v_{x}B}{-env_{x}B}
\end{align}
\begin{align}
\fbox{\ensuremath{{\displaystyle R_{H}=-\frac{1}{ne}}}}
\end{align}
Dies ist $T$ unabhängig. $R_{H}$ ist negativ für Elektronen. Die
Messung von $R_{H}$ ist eine wichtige Methode zur Bestimmung der
Ladungsträgerkonzentration $n$.
\end{itemize}

\section{Wärmeleitfähigkeit von Metallen}
\begin{itemize}
\item Neben Phononen tragen auch freie Elektronen zur Wärmeleitfähigkeit
$\kappa$ bei.
\begin{align*}
\kappa & =\frac{1}{3}c_{V}\overline{v}\Lambda
\end{align*}
$\overline{v}$: mittlere Teilchengeschwindigkeit\\
$\Lambda$: mittlere freie Weglänge
\item Für ein Fermigas ist $\overline{v}=v_{F}$ und $E_{F}=\frac{m}{2}v_{F}^{2}$,
sowie:
\begin{align*}
c_{V} & =c_{V}^{\text{elektrisch}}=\frac{\pi^{2}}{2}\frac{N_{A}k_{B}^{2}}{E_{F}}\cdot T
\end{align*}
\begin{align}
\kappa^{\text{elektrisch}} & =\frac{1}{3}c_{V}^{\text{elektrisch}}\cdot v_{F}\cdot\Lambda=\frac{\pi^{2}N_{A}k_{B}^{2}\tau}{3m}\cdot T
\end{align}

\item Bei Raumtemperatur ist $\frac{\kappa_{\text{Metall}}}{\kappa_{\text{Isolator}}}\approx10\ldots100$.
\end{itemize}
%DATE: Do 20.12.12


\section{Elektronen im periodischen Potential}
\begin{itemize}
\item Berücksichtige das periodische Potentials der Gitterionen.
\item Es entstehen \emph{Energielücken}, diese trennen \emph{Bänder}.
\end{itemize}

\subsection{Modell des fast freien Elektrons}

Füge dem Modell des freien Elektrons die Bragg-Reflexion hinzu, die
die schwache Wechselwirkung mit dem Gitterpotential berücksichtigt.

Für einen linearen eindimensionalen Kristall mit Gitterkonstante $a$
führt die Bragg-Bedingung
\begin{align*}
\left(\vec{k}+\vec{G}\right)^{2} & =\vec{k}^{2}
\end{align*}
wegen $G_{n}=n\cdot\frac{2\pi}{a}$ mit $n\in\mathbb{Z}\setminus\left\{ 0\right\} $
auf:
\begin{align}
k & =\pm\frac{G}{2}=\pm n\frac{\pi}{a}
\end{align}
Die erste Reflexion, das heißt die erste Energielücke, ist bei der
Grenze der 1. Brillouin-Zone:
\begin{align*}
k & =\pm\frac{\pi}{a}
\end{align*}
\begin{framed}

Reflexion bei $k=\pm\frac{\pi}{a}$ tritt auf, weil dann die von einem
Atom des Gitters reflektierte Welle konstruktiv (Phasenunterschied
$2\pi$) mit der vom nächsten Nachbaratom reflektierten Welle interferiert.

\end{framed}

Die Wellenfunktion für freie Elektronen ist proportional zu $e^{\ii kx}$
und hier ist $k=\pm\frac{\pi}{a}$. Dies erbibt stehende Wellen:
\begin{align}
\psi_{\left(+\right)} & =e^{\ii\frac{\pi}{a}x}+e^{-\ii\frac{\pi}{a}x} & \psi_{\left(y\right)} & =e^{\ii\frac{\pi}{a}x}-e^{-\ii\frac{\pi}{a}x}
\end{align}
Aufenthaltswahrscheinlichkeitsdichte:
\begin{align}
\varrho_{\left(+\right)} & =\abs{\psi_{\left(+\right)}}^{2}\sim\cos^{2}\left(\frac{\pi}{a}x\right) & \varrho_{\left(-\right)} & =\abs{\psi_{\left(-\right)}}^{2}\sim\sin^{2}\left(\frac{\pi}{a}x\right)
\end{align}
\textcolor{green}{TODO: Bild Plot $\varrho_{\left(+\right)}$, $\varrho_{\left(-\right)}$
über periodischem Potential}

$\varrho_{\left(+\right)}$ ($\varrho_{\left(-\right)}$) führt zu
Erniedrigung (Erhöhung) der potentiellen Energie des Elektrons (im
Vergleich zur laufenden Welle).

\textcolor{green}{TODO: Abb90}
\begin{itemize}
\item Die Energielücke ist eine Folge der Gitterperiodizität.
\item Je kleiner $a$ ist, desto größer ist die Energielücke.
\item Die Energielücke kann für verschiedene Richtungen im Kristall unterschiedlich
sein!
\end{itemize}

\subsection{Ausgedehntes und reduziertes Zonenschema}

\textcolor{green}{TODO: Bilder von Folie}

\begin{framed}

Bei $N$ Atomen im Kristall erhält man $N$ erlaubte Energiewerte
pro Band aufgrund der periodischen Randbedingungen. Maximal $2N$
Elektronen passen in ein Band (Spin!).

\end{framed}


\subsubsection*{Bemerkungen}
\begin{itemize}
\item Einfache Metalle mit einem Valenzelektron: Es gibt $N$ Elektronen
und daher ist das Leitungsband zur Hälfte besetzt. Die Elektronen
können kinetische Energie aufnehmen und ein Stromfluss kann entstehen.\\
\textcolor{green}{TODO: Abb91}
\item Einfache Metalle mit 2 Valenzelektronen: Das Valenzband ist mit $2N$
Elektronen vollständig gefüllt, \emph{aber} das Valenzband überlappt
mit dem Leitungsband.\\
\textcolor{green}{TODO: Abb92}
\item Ist das Valenzband vollständig besetzt und trennt eine Energielücke
das Valenzband vom Leitungsband, so erhält man einen Isolator oder
einen Halbleiter.\\
\textcolor{green}{TODO: Abb93}\\
Isolatoren oder Halbleiter müssen eine \emph{gerade} Anzahl an Valenzelektronen
haben.
\end{itemize}

\subsubsection*{Fermifläche}

\begin{framed}

Die Fermifläche ist Fläche konstanter Energie $E_{F}$ (Fermienergie)
im $\vec{k}$-Raum.

\end{framed}

Im Modell freier Elektronen ist die Fermifläche eine Kugel im $\vec{k}$-Raum.
Bei nicht einfachen Metallen kann die Fermifläche erheblich von der
Kugelform abweichen (Bandstruktur).


\subsection{Bloch-Funktionen, Bloch-Theorem}

Felix Bloch: Die Lösung der Schrödinger-Gleichung für ein periodisches
Potential haben spezielle Form:
\begin{align}
\fbox{\ensuremath{{\displaystyle \psi_{\vec{k}}\left(\vec{r}\right)=u_{\vec{k}}\left(\vec{r}\right)e^{\ii\vec{k}\vec{r}}}}}
\end{align}
\begin{align}
u_{\vec{k}}\left(\vec{r}\right) & =u_{\vec{k}}\left(\vec{r}+\vec{T}\right)
\end{align}
Das heißt eine ebene Welle ist mit einer gitterperiodischen Funktion
moduliert.

\textcolor{green}{TODO: Bild Bloch-Funktionen}


\subsection{Wellengleichung eines Elektrons in einem periodischen Potential}

Gitterperiodisches Potential in einer Dimension:
\begin{align}
U\left(x\right) & =\sum_{G}U_{G}e^{\ii Gx}
\end{align}
$U\left(x\right)$ soll eine reelle Funktion sein und der konstante
Term ist egal, man kann also schreiben:
\begin{align}
U\left(x\right) & =\sum_{G>0}U_{G}\left(e^{\ii Gx}+e^{-\ii Gx}\right)=2\sum_{G>0}U_{G}\cos\left(Gx\right)
\end{align}



\subsubsection*{Schrödinger-Gleichung in Ein-Elektron-Näherung}

\begin{align*}
H\psi & =E\psi
\end{align*}
Für \emph{ein} Elektron gilt:
\begin{align}
\left(\frac{p^{2}}{2m}+\sum_{G}U_{G}e^{\ii Gx}\right)\psi\left(x\right) & =E\psi\left(x\right)
\end{align}
$U\left(x\right)$ repräsentiert das Potential der Ionenrümpfe und
das Potential der übrigen Valenzelektronen. Die Wellenfunktion sei:
\begin{align}
\psi\left(x\right) & =\sum_{k}c\left(k\right)e^{\ii kx}
\end{align}
Die durch die periodischen Randbedingungen erlaubte Werte für $k$
mit $n\in\mathbb{Z}$ sind:
\begin{align*}
k & =n\cdot\frac{2\pi}{L}
\end{align*}
Einsetzen und Vergleichen der Koeffizienten liefert für alle $k$
die \emph{Hauptgleichung}:
\begin{align}
\fbox{\ensuremath{{\displaystyle \left(\frac{\hbar^{2}k^{2}}{2m}-E\right)c\left(k\right)+\sum_{G}U_{G}c\left(k-G\right)=0}}}\label{eq:Hauptgleichung}
\end{align}
Dies ist ein System von algebraischen Gleichungen.

Finde eine Näherungslösung nahe der Zonengrenze:
\begin{align*}
k & =\pm\frac{\pi}{a}=\frac{G}{2}
\end{align*}
Betrachte nur $G=\pm\frac{2\pi}{a}$.
\begin{align*}
U_{\pm\frac{2\pi}{a}} & =:U
\end{align*}
Berücksichtige nur folgende Koeffizienten:
\begin{align*}
c\left(\frac{G}{2}\right) & \to c\left(\frac{\pi}{a}\right),\ c\left(-\frac{\pi}{a}\right)
\end{align*}
Die Gleichung (\ref{eq:Hauptgleichung}) reduziert sich zu zwei Gleichungen:
\begin{align}
\left(\lambda-E\right)c\left(\frac{\pi}{a}\right)+Uc\left(-\frac{\pi}{a}\right) & =0\nonumber \\
\left(\lambda-E\right)c\left(-\frac{\pi}{a}\right)+Uc\left(\frac{\pi}{a}\right) & =0
\end{align}
Dabei ist:
\begin{align*}
\lambda & =\frac{\hbar^{2}}{2m}\left(\pm\frac{\pi}{a}\right)^{2}
\end{align*}
Dies führt auf:
\begin{align}
\fbox{\ensuremath{{\displaystyle E=\lambda\pm U=\frac{\hbar^{2}}{2m}\frac{\pi^{2}}{a^{2}}\pm U}}}
\end{align}
Für ein freies Elektron ist $E=\lambda$. Dies ergibt die Energielücke
$2U$. Die Wellenfunktion ist:
\begin{align*}
\psi\left(x\right) & =e^{\ii\frac{\pi}{a}x}\pm e^{-\ii\frac{\pi}{a}x}
\end{align*}
\textcolor{green}{TODO: Abb94}


\subsection{Modell des leeren Gitters}

Modelliere die Gitterperiodizität, aber mit $U\to0$. Es bleibt die
Bragg-Bedingung:
\begin{align*}
\left(\vec{k}+\vec{G}\right)^{2} & =\vec{k}^{2}
\end{align*}
Für ein freies Elektron mit Bragg-Bedingung und $\vec{k}$ aus der
1. Brillouin-Zone gilt:
\begin{align}
\fbox{\ensuremath{{\displaystyle E_{hkl}\left(\vec{k}\right)=\frac{\hbar^{2}k^{2}}{2m}=\frac{\hbar^{2}}{2m}\left(\vec{k}+\vec{G}_{hkl}\right)^{2}}}}
\end{align}



\subsubsection*{Beispiel: Elektron im fcc Gitter (z.B. Cu)}

Die reziproken Basisvektoren sind:
\begin{align*}
\vec{A} & =\frac{2\pi}{a}\left(\begin{array}{c}
1\\
1\\
-1
\end{array}\right) & \vec{B} & =\frac{2\pi}{a}\left(\begin{array}{c}
-1\\
1\\
1
\end{array}\right) & \vec{C} & =\frac{2\pi}{a}\left(\begin{array}{c}
1\\
-1\\
1
\end{array}\right)
\end{align*}
\begin{align*}
\vec{G}_{hkl} & =h\vec{A}+k\vec{B}+l\vec{C}=\frac{2\pi}{a}\left(\begin{array}{c}
h-k+l\\
h+k-l\\
-h+k+l
\end{array}\right)
\end{align*}
$\left[100\right]$ ist die $\Gamma X$-Richtung, also mit $x\in\left[0,1\right]$
gilt:
\begin{align*}
\vec{k} & =\frac{2\pi}{a}\left(\begin{array}{c}
x\\
0\\
0
\end{array}\right)
\end{align*}
Verwende folgende Abkürzung:
\begin{align*}
E_{0} & =\frac{\hbar^{2}}{2m}\left(\frac{2\pi}{a}\right)^{2}
\end{align*}
Die Energie ist allgemein
\begin{align*}
E_{hkl} & =\frac{\hbar^{2}}{2m}\left(\left(k_{x}+G_{x}\right)^{2}+\left(k_{y}+G_{y}\right)^{2}+\left(k_{z}+G_{z}\right)^{2}\right)
\end{align*}
und speziell in $\Gamma X$-Richtung mit $x\in\left[0,1\right]$:
\begin{align*}
E_{hkl}\left(\Gamma X\right) & =E_{0}\left(\left(x+h-k+l\right)^{2}+\left(h+k-l\right)^{2}+\left(-h+k+l\right)^{2}\right)
\end{align*}
$x=0$ ist $\Gamma$-Punkt und $x=1$ ist $X$-Punkt.

\textcolor{green}{TODO: Abb95}

%DATE: Mo 7.1.13


\chapter{Halbleiter}

\textcolor{green}{TODO: Abb Folie 7. Halbleiter}

Bei Halbleitern ist die Gap-Energie typischerweise kleiner als $6\text{eV}$.
Ist es größer, so handelt es sich um einen Isolator.
\begin{itemize}
\item $E_{\text{Gap}}=0$: Leiter
\item $0<E_{\text{Gap}}<6\,\text{eV}$: Halbleiter; Reine Halbleiter sind
bei $T=0\,\text{K}$ isolierend.
\item $E_{\text{Gap}}>6\,\text{eV}$: Isolator
\end{itemize}
\emph{Elementhalbleiter} sind beispielsweise Si oder Ge (IV. Hauptgruppe).
Diamant C ist wegen $E_{\text{Gap}}\approx9\,\text{eV}>6\,\text{eV}$
kein Halbleiter. Beispiele für \emph{Verbindungshalbleiter} sind:
\begin{itemize}
\item GaAs, InAs, GaSb, InP, GaP (III-V-Halbleiter)
\item CdSe, ZnS, CdTe, ZnSe (II-VI-Halbleiter)
\end{itemize}
Si und Ge haben Diamantstruktur: fcc-Gitter mit 2-atomiger Basis
\begin{align*}
\left(0,0,0\right) & \ \left(\frac{a}{4},\frac{a}{4},\frac{a}{4}\right)\\
\text{Si}\quad & \qquad\ \text{Si}
\end{align*}
Die meisten III-V-Halbleiter liegen in Zinkblendestruktur vor: fcc-Gitter
mit 2-atomiger Basis
\begin{align*}
\left(0,0,0\right) & \ \left(\frac{a}{4},\frac{a}{4},\frac{a}{4}\right)\\
\text{Ga}\quad & \qquad\ \text{As}
\end{align*}



\section{Die Bandlücke}

Die \emph{Bandlücke} ist der energetisch kleinste Abstand zwischen
Valenz- und Leitungsband. Man unterscheidet \emph{direkte} und \emph{indirekte}
Bandlücken. (Details kommen später.)
\begin{itemize}
\item GaAs: direkte Bandlücke ($E_{\text{Gap}}=1{,}42\,\text{eV}$ bei $T=300\,\text{K}$)\\
\textcolor{green}{TODO: Abb96}
\item Si: indirekte Bandlücke ($E_{\text{Gap}}=1{,}12\,\text{eV}$ bei $T=300\,\text{K}$)\\
\textcolor{green}{TODO: Abb97}
\end{itemize}
\begin{align*}
800\,\text{nm}\, & \hat{=}\,1{,}55\,\text{eV}\\
400\,\text{nm}\, & \hat{=}\,3{,}1\,\text{eV}
\end{align*}
\begin{align*}
\mbox{\fbox{\ensuremath{{\displaystyle \lambda\left[\text{nm}\right]=\frac{1240}{E\left[\text{eV}\right]}}}}}
\end{align*}



\subsubsection*{Direkte und indirekte optische Übergänge}
\begin{itemize}
\item Die Lichtwellenlänge $\lambda_{\text{opt}}\gg a$ ist sehr viel größer
als die Gitterkonstante $a$.
\begin{align*}
\frac{2\pi}{\lambda_{\text{opt}}} & \approx10^{7}\frac{1}{\text{m}}\ll\frac{2\pi}{a}\approx10^{18}\frac{1}{\text{m}}
\end{align*}
\begin{framed}Der Photonenimpuls ist also vernachlässigbar gegenüber
der Ausdehnung der Brillouin-Zone. Optische Übergänge verbinden also
Zustände, die in der Bandstruktur (nahezu) vertikal übereinander liegen.
Man spricht von vertikalen Übergängen.\end{framed}
\item Direkter Halbleiter:\\
\textcolor{green}{TODO: Abb98}\\
Indirekter Halbleiter:\\
\textcolor{green}{TODO: Abb99}
\item Berechnung der optischen Absorption mittels Störungstheorie.
\begin{align*}
\alpha\left(\omega\right) & \sim\text{kombinierte Zustandsdichte }D^{*}\left(\omega\right)
\end{align*}
Übergangswahrscheinlichkeiten sind an den Bandextrema besonders groß
(van Hove"=Singularitäten).\\
\textcolor{green}{TODO: Abb100, Abb101}
\end{itemize}

\section{Elektronen und Löcher}

\begin{framed}

Ein \emph{Loch} ist ein unbesetzter Zustand in einem ansonsten besetzten
Band (im Normalfall im Valenzband des Halbleiters). Es trägt die Ladung
$+e$, hat den Wellenvektor
\begin{align}
\vec{k}_{h} & =-\vec{k}_{e}
\end{align}
und die Energie:
\begin{align}
E_{h}\left(\vec{k}_{h}\right) & =-E_{e}\left(\vec{k}_{e}\right)
\end{align}
Dabei wird der Energienullpunkt an die Valenzband-Oberkante gesetzt.

\end{framed}

\textcolor{green}{TODO: Abb102}

Impulserhaltung:
\begin{align*}
0\approx k_{\text{Photon}} & =k_{e}+k_{h}\\
k_{h} & \approx-k_{e}
\end{align*}


Die Dynamik eines Lochs wird durch Bildung eines \emph{Lochbandes}
(Inversion des Valenzbandes am Energienullpunkt) simuliert.

\textcolor{green}{TODO: Abb103}


\section{Die effektive Masse}
\begin{itemize}
\item Betrachte die Gruppengeschwindigkeit eines Teilchens, das heißt die
Ausbreitungsgeschwindigkeit seines Wellenpakets:
\begin{align}
v_{g} & =\frac{\dd\omega}{\dd k}=\frac{1}{\hbar}\frac{\dd E}{\dd k}
\end{align}

\item Bewegungsgleichung:
\begin{align*}
\frac{\dd v_{g}}{\dd t} & =\frac{1}{\hbar}\frac{\dd^{2}E}{\dd t\dd k}=\frac{1}{\hbar}\frac{\dd^{2}E}{\dd k^{2}}\cdot\frac{\dd k}{\dd t}
\end{align*}
Mit
\begin{align*}
F & =\frac{\dd p}{\dd t}=\hbar\frac{\dd k}{\dd t}
\end{align*}
folgt:
\begin{align*}
\frac{\dd v_{g}}{\dd t} & =\underbrace{\frac{1}{\hbar^{2}}\frac{\dd^{2}E}{\dd k^{2}}}_{=:\frac{1}{m^{*}}}\cdot F
\end{align*}

\item \emph{Effektive Masse}:
\begin{align}
\fbox{\ensuremath{{\displaystyle \frac{1}{m^{*}}:=\frac{1}{\hbar^{2}}\frac{\dd^{2}E}{\dd k^{2}}}}}
\end{align}
Für eine parabolische Bandstruktur, das heißt $E\left(k\right)=ak^{2}$,
folgt $a=\frac{\hbar^{2}}{2m^{*}}$ und $m^{*}$ ist konstant, sonst
hängt $m^{*}$ von $k$ ab. Im Allgemeinen ist $m^{*}$ richtungsabhängig,
das heißt ein Tensor:
\begin{align}
\fbox{\ensuremath{{\displaystyle \left(\frac{1}{m^{*}}\right)_{\mu\nu}=\frac{1}{\hbar^{2}}\frac{\dd^{2}E}{\dd k_{\mu}\dd k_{\nu}}}}}
\end{align}
Anmerkung: Für die meisten Metalle ist $m^{*}\approx m_{0}$, das
heißt die Wechselwirkung mit dem Gitter ist schwach. Für Halbleiter
gilt im Allgemeinen $m^{*}<m_{0}$, das heißt die Bandstruktureffekte
sind wichtig.
\item Einfache Vorstellung zur Bildung der Bandstruktur im Halbleiter:\\
$s$- und $p$-Orbitale benachbarter Atome überlappen und bilden wie
im Wasserstoffmolekül bindende und anti-bindende Molekülorbitale.
Hier haben wir aber $N$ Atome, weswegen sich die Orbitale zu Bändern
verbreitern.\\
\textcolor{green}{TODO: Abb104}\\
Beispiel: In Si gibt es 2 Si-Atome je Einheitszelle 8 Elektronen.
Dies gibt einen Halbleiter.\\
GaAs: 1 Ga und 1 As-Atom geben ebenfalls 8 Elektronen pro Einheitszelle.
Dies gibt einen Halbleiter.
\item \emph{Genauere Betrachtung der Bandkanten:}

\begin{enumerate}[label=\alph*)]
\item Halbleiter mit direkter Bandlücke bei $\vec{k}=0$ (zum Beispiel
GaAs, InAs, InP, $\ldots$).\\
\textcolor{green}{TODO: Abb105 (CB: conduction band)}\\
Leitungsband besteht aus antibindenden $s$-Orbitalen:
\begin{align*}
E_{cb} & =E_{G}+\frac{\hbar^{2}k^{2}}{2m_{e}^{*}}
\end{align*}
\begin{align*}
l & =0 & s & =\frac{1}{2}\\
\vec{j} & \sr ={\text{Russel-}}{\text{Saunders}}\vec{L}+\vec{S} & j & =\abs{l\pm s}=\frac{1}{2}
\end{align*}
Valenzband besteht aus bindenden $p$-Orbitalen.
\begin{align*}
l & =1 & s & =\frac{1}{2}\\
\vec{j} & \sr ={\text{Russel-}}{\text{Saunders}}\vec{L}+\vec{S} & j & =\abs{l\pm s}\in\left\{ \frac{1}{2},\frac{3}{2}\right\} 
\end{align*}
heavy hole:
\begin{align*}
E_{hh} & =-\frac{\hbar^{2}k^{2}}{2m_{hh}^{*}}
\end{align*}
light hole:
\begin{align*}
E_{lh} & =-\frac{\hbar^{2}k^{2}}{2m_{lh}^{*}}
\end{align*}
split off band:
\begin{align*}
E_{so} & =-\Delta-\frac{\hbar^{2}k^{2}}{2m_{so}^{*}}
\end{align*}
Um $\vec{k}=0$ können die Bänder parabolisch genähert werden! Dies
bedeutet eine konstante effektive Masse der Elektronen bei $k=0$.
Für GaAs ist $m_{e}^{*}\left(k=0\right)\approx0,07\cdot m_{e}$.\\
Die Fläche konstanter Energie ist:\\
\textcolor{green}{TODO: Abb106}%DATE: Mi 9.1.13
\item Silizium (indirekter Halbleiter):\\
\textcolor{green}{TODO: Bandstruktur siehe Folie (in Plot mit 2 Pseudo-Potentialen
gerechnet)}\\
Spin orbit coupling: Das Band ist wegen der schwachen Spin-Bahn-Wechselwirkung
um $\Delta=44\,\text{meV}$ verschoben.\\
Silizium hat 6 äquivalente Leitungsbandminima in den $\vec{k}$-Richtungen
$\left[100\right]$, $\left[010\right]$, $\left[001\right]$, $\left[\overline{1}00\right]$,
$\left[0\overline{1}0\right]$ und $\left[00\overline{1}\right]$
nahe dem $X$-Punkt.

\begin{itemize}
\item Die effektive Masse ist stark anisotrop und die Flächen konstanter
Energie sind daher keine Kugeln, sondern Ellipsoide!
\item Germanium hat 6 äquivalente Leitungsbandminima in Richtung $\left[111\right]$
und äquivalente Richtungen (am $L$-Punkt).
\end{itemize}
\end{enumerate}
\end{itemize}

\section{Eigenleitung}

Freie Ladungsträger entstehen nur aufgrund thermischer Anregung. Deshalb
ist die Zahl der Elektronen gleich der Zahl der Löcher! Behandlung
von Elektronen und Löchern jeweils als freie Teilchen.

\textcolor{green}{TODO: Abb107; Node1: $D_{e}\left(E_{e}\right)\sim\sqrt{E_{e}-E_{\text{Gap}}}$;
blau: $D_{h}\left(E_{h}\right)\sim\sqrt{E_{h}}$}


\subsubsection*{Ladungsträgerkonzentration bei Eigenleitung}
\begin{itemize}
\item Elektronen im Leitungsband: $E-\mu\gg k_{B}T$
\begin{align}
f_{e} & =\frac{1}{e^{\frac{E-\mu}{k_{B}T}}+1}\approx e^{\frac{\mu-E}{k_{B}T}}
\end{align}
Zustandsdichte (siehe (\ref{eq:D(E)})) pro Volumen:
\begin{align}
D_{e}\left(E\right) & =\frac{1}{2\pi^{2}}\left(\frac{2m_{e}}{\hbar^{2}}\right)^{\frac{3}{2}}\sqrt{E-E_{\text{gap}}}
\end{align}
Elektronendichte im Leitungsband:
\begin{align}
n & =\int_{E_{\text{gap}}}^{\infty}D_{e}\left(E\right)f_{e}\left(E,T\right)\dd E=\nonumber \\
 & =\frac{1}{2\pi^{2}}\left(\frac{2m_{e}^{*}}{\hbar^{2}}\right)^{\frac{3}{2}}e^{\frac{\mu}{k_{B}T}}\int_{E_{\text{gap}}}^{\infty}\sqrt{E-E_{\text{gap}}}e^{\frac{-E}{k_{B}T}}\dd E=\nonumber \\
 & =2\left(\frac{m_{e}^{*}k_{B}T}{2\pi\hbar^{2}}\right)^{\frac{3}{2}}e^{\frac{\mu-E_{\text{gap}}}{k_{B}T}}\label{eq:n}
\end{align}

\item Löcher sind fehlende Elektronen, das heißt es gilt:
\begin{align}
f_{h} & =1-f_{e}=1-\frac{1}{e^{\frac{E-\mu}{k_{B}T}}+1}\approx e^{\frac{E-\mu}{k_{B}T}}
\end{align}
Lochdichte im Valenzband:
\begin{align}
p & =2\left(\frac{m_{h}^{*}k_{B}T}{2\pi\hbar^{2}}\right)^{\frac{3}{2}}e^{-\frac{\mu}{k_{B}T}}\label{eq:p}
\end{align}

\item Beachte:
\begin{align}
\fbox{\ensuremath{{\displaystyle np=4\left(\frac{k_{B}T}{2\pi\hbar^{2}}\right)^{3}\left(m_{e}^{*}m_{h}^{*}\right)^{\frac{3}{2}}e^{\frac{-E_{\text{gap}}}{k_{B}T}}}}}
\end{align}
Dies ist unabhängig von $\mu$! (Massenwirkungsgesetz)
\item Bei Eigenleitung gilt $n=p$:
\begin{align}
n & =p=2\left(\frac{k_{B}T}{2\pi\hbar^{2}}\right)^{3}\left(m_{e}^{*}m_{h}^{*}\right)^{\frac{3}{2}}e^{\frac{-E_{\text{gap}}}{k_{B}T}}
\end{align}

\item Setze (\ref{eq:n}) gleich (\ref{eq:p}) und erhalte:
\begin{align*}
e^{\frac{2\mu}{k_{B}T}} & =\left(\frac{m_{h}^{*}}{m_{e}^{*}}\right)^{\frac{3}{2}}e^{\frac{E_{\text{gap}}}{k_{B}T}}
\end{align*}
\begin{align}
\fbox{\ensuremath{{\displaystyle \mu=\frac{E_{\text{gap}}}{2}+\frac{3}{4}k_{B}T\ln\left(\frac{m_{h}^{*}}{m_{e}^{*}}\right)}}}
\end{align}
$T=0$: Die Fermienergie liegt in der Bandlückenmitte:
\begin{align*}
\mu & =E_{F}=\frac{E_{\text{gap}}}{2}
\end{align*}
$T>0$: Falls $m_{e}^{*}=m_{h}^{*}$ gilt, folgt $\mu=\frac{E_{\text{gap}}}{2}$,
was nicht von $T$ abhängt.\\
Falls $m_{e}^{*}\not=m_{h}^{*}$ gilt, ist $\mu$ temperaturabhängig.\\
In der Halbleiterphysik wird das chemische Potential $\mu$ oft auch
als Fermienergie bezeichnet, selbst für $T>0$.
\item Elektrische Leitfähigkeit bei Eigenleitung: Driftgeschwindigkeit (vergleiche
(\ref{eq:Driftgeschwindigkeit})).
\begin{align*}
\norm{\vec{v}_{D}} & =\frac{e\tau}{m}\norm{\vec{E}}
\end{align*}
Die Beweglichkeit wird definiert als:
\begin{align*}
\mu_{b} & =\frac{e\tau}{m^{*}}
\end{align*}
In intrinsischen Halbleiter stoßen Elektronen und Löcher hauptsächlich
mit \emph{Phononen}.
\begin{align*}
\mu_{b,e} & =\frac{e\tau_{e}}{m_{e}^{*}} & \mu_{b,h} & =\frac{e\tau_{h}}{m_{h}^{*}}
\end{align*}
Elektrische Leitfähigkeit:
\begin{align*}
\sigma & =ne\mu_{b,e}+pe\mu_{b,n}\\
\vec{j} & =\sigma\vec{E}
\end{align*}

\end{itemize}

\section{Störstellenleitung}

Durch \emph{Dotieren}, also durch den gezielten Einbau von Fremdatomen,
kann man die Leitfähigkeit verändern.
\begin{itemize}
\item $n$-Dotierung: Ein Donator hat \emph{ein} Valenzeelektron \emph{mehr}
als Kristallatom.\\
\textcolor{green}{TODO: Abb108, 109}\\
Durch Ionisierung kann das Donatoratom ein Elektron in das Leitungsband
(conduction band, CB) abgeben. Dadurch erhält man ein freies Elektron
im Leitungsband.
\item $p$-Dotierung:\\
\textcolor{green}{TODO: Abb110, Abb111}\\
Durch Ionisierung nimmt der Akzeptor ein Elektron aus dem Valenzband
auf und man erhält ein freies Loch im Valenzband!
\end{itemize}
\begin{framed}

\begingroup \leftskip0.4em \rightskip\leftskip

$-$ Störstellenleitung: Die elektrische Leitfähigkeit wird durch
die Dotierung eingebrachter

\hspace{1em}Ladungsträger dominiert.

\endgroup

\end{framed}
\begin{itemize}
\item Beschreibung eines Donators analog zum Bohrschen Atommodell: Elektron
am Donator ist in etwa ein Wasserstoffatom mit dielektrischer Umgebung
$\varepsilon_{r}$ und effektiver Elektronenmasse $m^{*}$.\\
Bohrradius:
\begin{align*}
a^{*} & =0{,}052\,\text{nm}\cdot\frac{\varepsilon_{r}m_{0}}{m_{*}}\approx10{,}5\,\text{nm}\quad\text{(für GaAs)}
\end{align*}
Ionisierungsenergie:
\begin{align*}
E_{I} & =13{,}6\,\text{eV}\cdot\frac{m^{*}}{m_{0}\varepsilon^{2}}\approx0{,}0054\,\text{eV}\quad\text{(für GaAs)}
\end{align*}
Generell ist $E_{I}\approx\text{meV}$ und $k_{B}T=25\,\text{meV}$
bei Raumtemperatur. Also sind praktisch alle Donatoren und Akzeptoren
ionisiert. Daher hat man freie Ladungsträger im Leitungsband beziehungsweise
Valenzband.
\item Zusätzlich findet eine thermische Anregung von Elektronen aus dem
Valenzband ins Leitungsband statt.
\item Unterscheidung:

\begin{enumerate}[label=\alph*)]
\item nicht entartete Halbleiter: Für $T\to0$ sind die Donatoren (Akzeptoren)
nicht ionisiert und man hat keine freien Ladungsträger.
\item entartete Halbleiter: Dotierung ist so groß, dass die Abschirmung
und Coulombabstoßung ausreichen, um Elektronen (Löcher) ins Leitungsband
(Valenzband) anzuheben. Selbst für $T=0$ hat man freie Elektronen
Löcher.
\end{enumerate}
\item Temperaturabhängigkeit der Elektronenkonzentration:\\
\textcolor{green}{TODO: siehe Folien}\\
\textcolor{green}{TODO: Abb112}
\item Lage des chemischen Potentials in GaAs als Funktion der Temperatur:\\
\textcolor{green}{TODO: siehe Folien}
\end{itemize}
%DATE: Mi 23.1.13


\section{Exzitonen}

Betrachte einen intrinsischen Halbleiter:

\textcolor{green}{TODO: Abb113}

Erzeuge ein Elektron-Loch-Paar durch Absorption eines Photons hinreichender
Energie.

Die Coulomb-Wechselwirkung verbindet dieses Paar zu einem Exziton.

\begin{framed}

Ein \emph{Exziton} ist ein gebundener Zustand von Elektron und Loch
(analog zum Wasserstoffatom) aufgrund der Coulomb-Wechselwirkung.

\end{framed}
\begin{enumerate}[label=\roman*)]
\item Frenkel-Exziton: Elektron und Loch sind in einer Einheitszelle lokalisiert.
Die Bindungsenergie ist groß, zum Beispiel in ionischen Kristallen
und Isolatoren.
\item Wannier-Mott-Exziton: Die Bindung ist schwache und die Dielektrizitätskonstante
groß. Dies gibt es bei vielen Halbleitern.
\end{enumerate}
Betrachte die Relativbewegung von Elektron und Loch analog zum Wasserstoffatom:
\begin{align}
E_{r}\left(n\right) & =\underbrace{E_{r}\left(\infty\right)}_{\text{hier: }E_{\text{Gap}}}-\frac{E_{\text{Ry}}^{*}}{n^{2}}
\end{align}
Hierbei ist:
\begin{align*}
E_{\text{Ry}}^{*} & =13{,}6\,\text{eV}\cdot\left(\frac{\mu}{m_{0}\varepsilon^{2}}\right) & \frac{1}{\mu} & =\frac{1}{m_{e}^{*}}+\frac{1}{m_{h}^{*}}
\end{align*}
Gesamtenergie:
\begin{align}
E & =E_{\text{Gap}}-\underbrace{\frac{E_{\text{Ry}}}{n^{2}}}_{\text{Bindungsenergie}}+\underbrace{\frac{\hbar^{2}k^{2}}{2M}}_{\text{Schwerpunktsenergie}}
\end{align}
\begin{align*}
\vec{k} & =\vec{k}_{e}+\vec{k}_{n} &  & \text{Schwerpunktsimpuls}\\
M & =m_{e}^{*}+m_{n}^{*} &  & \text{Gesamtmasse}
\end{align*}
\textcolor{green}{TODO: Abb114; Nur die eingekreisten Exzitonen können
strahlend rekombinieren.}

Wegen der Impulserhaltung ist $\vec{k}_{\text{Photon}}=\vec{k}\approx0$.

\noindent \begin{center}
\begin{tabular}{c|c|c}
HL$\vphantom{\Big|}$ & $E_{\text{Ry}}^{*}$ in meV & $a_{0}^{*}$ in $\angs$\tabularnewline
\hline 
\hline 
GaAs$\vphantom{\Big|}$ & 4,9 & 112\tabularnewline
\hline 
InP$\vphantom{\Big|}$ & 5,1 & 113\tabularnewline
\hline 
CaTe$\vphantom{\Big|}$ & 11 & 12,2\tabularnewline
\hline 
ZnTe$\vphantom{\Big|}$ & 13 & 11,5\tabularnewline
\hline 
ZnSe$\vphantom{\Big|}$ & 19,9 & 10,7 \tabularnewline
\end{tabular}
\par\end{center}

Das Absorptionsspektrum ist durch Exzitonen modifiziert (bei tiefen
Temperaturen).

\textcolor{green}{TODO: Abb115}

Falls $k_{B}T>E_{\text{Ry}}^{*}$ ist, wird das Exziton thermisch
ionisiert in ein freies Elektron-Loch-Paar.


\chapter[Dielektrik und Optik]{Dielektrische und optische Eigenschaften von Festkörpern}


\section{Die dielektrische Funktion}
\begin{itemize}
\item Wiederholung Elektrodynamik: Beschreibung elektromagnetischer Wechselwirkung
mit Materie durch die Maxwell-Gleichung und Materialgleichungen:
\begin{align}
\vec{D} & =\varepsilon_{0}\vec{E}+\vec{P}=\varepsilon\varepsilon_{0}\vec{E} &  & \text{elektrische Flussdichte}\\
\vec{P} & =\varepsilon_{0}\left(\varepsilon-1\right)\vec{E} &  & \text{Polarisation}
\end{align}
$\varepsilon$ beschreibt den Einfluss des Mediums. Nehme zur Einfachheit
$\mu=1$ an. Im Vakuum gilt $\varepsilon=1$. In allgemeiner Materie
ist
\begin{align}
\varepsilon & =\varepsilon\left(\omega,\vec{q}\right)
\end{align}
ein Tensor, der von (Kreis-)Frequenz $\omega$ und Wellenvektor $\vec{q}$
der elektromagnetischen Strahlung abhängt.
\item Beachte:

\begin{itemize}
\item $\varepsilon\left(0,\vec{q}\right)$ ist die elektrostatische Abschirmung
der Elektron-Elektron- und der Elektron-Gitter-Wechselwirkung.
\item $\varepsilon_{L}\left(\omega,0\right)$ beschreibt kollektive longitudinale
Anregung des Fermisees (Plasmonen) oder des Gitters (LO-Phononen).
Bei Frequenzen der longitudinalen Anregung gilt:
\begin{align}
\fbox{\ensuremath{{\displaystyle \varepsilon\left(\omega,0\right)=0}}}
\end{align}
Begründung:
\begin{align*}
\vec{\nabla}\cdot\vec{D} & =\varrho=0\\
\Rightarrow\qquad\varepsilon\left(\omega\right)\frac{\vec{\nabla}\cdot\vec{P}}{\left(\varepsilon\left(\omega\right)-1\right)} & =0
\end{align*}
Für eine longitudinale Welle, zum Beispiel $\vec{P}=\vec{P}_{x_{0}}e^{\ii kx-\ii\omega t}$,
gilt $\vec{\nabla}\cdot\vec{P}\not=0$ und somit muss $\varepsilon\left(\omega\right)\stackrel{!}{=}0$
gelten.
\item Isolatoren: $\varepsilon$ wird durch Phononen bestimmt.\\
Daher ist die frequenzabhängige Antwort von $\varepsilon$ im Bereich
der Phononenfrequenz, also Infrarot oder fernes Infrarot.
\item Metalle: $\varepsilon$ wird durch freie Elektronen bestimmt. Die
Frequenzabhängigkeit ist im Bereich der Plasmafrequenz der freien
Elektronen. (sichtbares Licht bis ultraviolette Strahlung)
\end{itemize}
\end{itemize}

\section{Dielektrische Funktion eines freien Elektronengases}

\emph{Plasma} ist ein ladungsneutrales Gas aus geladenen Teilchen.

Hier betrachten wir den positiven Hintergrund des Ionengitters und
negativ geladene freie Elektronen für $\lambda\to\infty$, das heißt
$\varepsilon\left(\omega,\vec{q}\approx0\right)=\varepsilon\left(\omega\right)$.
Elektronen im elektrischen Wechselfeld $E\left(\omega\right)$ erfüllen
diese Bewegungsleichung:
\begin{align*}
m\cdot\frac{\dd^{2}x}{\dd t^{2}} & =-eE
\end{align*}
$x$ und $E$ haben die gleiche Zeitabhängigkeit proportional zu $e^{-\ii\omega t}$,
womit folgt:
\begin{align*}
-\omega^{2}m\cdot x & =-eE\\
x & =\frac{eE}{m\omega^{2}}
\end{align*}

\begin{itemize}
\item Die \emph{Polarisation} ist das induzierte Dipolmoment pro Volumeneinheit.
\begin{align*}
P & =-nex=-\frac{ne^{2}}{m\omega^{2}}E
\end{align*}
\begin{align}
\fbox{\ensuremath{{\displaystyle \varepsilon\left(\omega\right)=1+\frac{P\left(\omega\right)}{\varepsilon_{0}E\left(\omega\right)}=1-\frac{ne^{2}}{\varepsilon_{0}m\omega^{2}}=1-\frac{\omega_{p}^{2}}{\omega^{2}}}}}\label{eq:eps-omega}
\end{align}
Die \emph{Plasmafrequenz} ergibt sich zu:
\begin{align}
\fbox{\ensuremath{{\displaystyle \omega_{p}=\sqrt{\frac{ne^{2}}{\varepsilon_{0}m}}}}}
\end{align}
\begin{align*}
0 & \stackrel{!}{=}\varepsilon\left(\omega\right)=1-\frac{\omega_{p}^{2}}{\omega^{2}}\\
\Leftrightarrow\qquad\omega & =\omega_{p}
\end{align*}

\item Nach (\ref{eq:eps-omega}) kann $\varepsilon\left(\omega\right)$
auch negativ sein. Bedeutung?
\item Aus der Wellengleichung für elektromagnetische Felder folgt, dass
$E$ und $B$ proportional zu $e^{\ii\left(kx-\omega t\right)}$ sind,
das heißt propagierende Wellen mit Phasengeschwindigkeit:
\begin{align*}
c & =\frac{c_{0}}{n}
\end{align*}
($c_{0}$: Vakuumlichtgeschwindigkeit, $n=\sqrt{\varepsilon\mu}$:
Brechungsindex) 
\begin{align*}
\lambda & =\frac{\lambda_{0}}{n}\\
k & =\frac{2\pi}{\lambda}=n\cdot\frac{2\pi}{\lambda_{0}}
\end{align*}
Für $\varepsilon>0$ ist $n$ reell und somit $e^{\ii\left(kx-\omega t\right)}$
eine ebene Welle.\\
Für $\varepsilon<0$ ist $n$ imaginär und somit $e^{\ii\left(kx-\omega t\right)}=e^{-\abs kx-\ii\omega t}$
eine exponentiell gedämpfte Welle (evaneszente Welle).\\
\textcolor{green}{TODO: Abb116; Im Dämpfungsgebiet: elektromagnetische
Welle wird von der Oberfläche total reflektiert.}
\end{itemize}
%DATE: Do 24.1.13
\begin{itemize}
\item In realen Festkörpern (Metallen) liefern die positiv geladene Ionen
des Gitters den Hintergrund $\varepsilon\left(\infty\right)$. Ersetze
$m$ durch die effektive Masse $m^{*}$ und erhalte:
\begin{align*}
\varepsilon\left(\omega\right) & =\varepsilon\left(\infty\right)-\frac{\omega_{p}^{2}}{\omega^{2}}
\end{align*}
Aus
\begin{align*}
\varepsilon\left(\omega\right) & \stackrel{!}{=}0
\end{align*}
ergibt sich:
\begin{align}
\fbox{\ensuremath{{\displaystyle \omega^{2}=\tilde{\omega}_{p}^{2}=\frac{\omega_{p}^{2}}{\varepsilon\left(\infty\right)}=\frac{ne^{2}}{\varepsilon_{0}\varepsilon\left(\infty\right)m^{*}}}}}
\end{align}
Dies ist die Plasmafrequenz von Metallelektronen. Für Metalle liegt
$\tilde{\omega}_{p}$ im UV-Spektralbereich. Sichtbares Licht wird
also reflektiert, weshalb alle Metalle nahezu gleich aussehen.\\
\textcolor{green}{TODO: Abb117}\\
Mit $n=n_{R}+\ii\kappa$:
\begin{align*}
R & =\frac{\abs{n_{R}-1}^{2}+\kappa^{2}}{\abs{n_{R}+1}^{2}+\kappa^{2}}
\end{align*}
Falls $n$ imaginär ist, folgt $R=1$.
\end{itemize}

\section{Dielektrische Funktion eines Ionenkristalls}
\begin{itemize}
\item $\varepsilon\left(\omega\right)$ wird durch die Polarisierbarkeit
des Gitters (Phononen) bestimmt.
\begin{align}
\fbox{\ensuremath{{\displaystyle \varepsilon\left(\omega\right)=\varepsilon\left(\infty\right)\cdot\frac{\omega_{\text{LO}}^{2}-\omega^{2}}{\omega_{\text{TO}}^{2}-\omega^{2}}}}}
\end{align}
Dies ist die dielektrische Funktion eines Ionengitters.
\item $\varepsilon\left(\omega\right)$ hat eine Nullstelle bei $\omega_{\text{LO}}$
und einen Pol bei $\omega_{\text{TO}}$.\\
\textcolor{green}{TODO: Abb118}\\
$\varepsilon\left(0\right)$ ist die \emph{statische Dielektrizitätskonstante}
und $\varepsilon\left(\infty\right)$ die \emph{optische Dielektrizitätskonstante}.
\item Es gilt die \emph{Lydane-Sachs-Teller-Relation}:
\begin{align}
\fbox{\ensuremath{{\displaystyle \frac{\varepsilon\left(0\right)}{\varepsilon\left(\infty\right)}=\frac{\omega_{\text{LO}}^{2}}{\omega_{\text{TO}}^{2}}}}}
\end{align}
\textcolor{green}{TODO: Abb119}
\end{itemize}

\chapter{Supraleitung}

\begin{framed}

Mit \emph{Supraleitung} bezeichnet man das Verschwinden des elektrischen
Restwiderstandes unterhalb einer kritischen Temperatur $T_{c}$.

\end{framed}

Die Entdeckung erfolgte 1911 durch Kammerlingh-Onnes an Quecksilber.
(Nobelpreis 1913)

Viele Metalle sind\emph{ Konventionelle Supraleiter}:

\noindent \begin{center}
\begin{tabular}{c||c|c|c|c|c|c}
Elemente$\vphantom{\Big|}$ & Al & In & Sn & Hg & Ta & Pb\tabularnewline
\hline 
$T_{c}\left(\text{K}\right)$$\vphantom{\Big|}$ & 1,14 & 3,4 & 3,72 & 4,16 & 4,48 & 7,9 \tabularnewline
\end{tabular}
\par\end{center}

\noindent \begin{center}
\begin{tabular}{c||c|c|c|c}
Verbindung$\vphantom{\Big|}$ & $\text{Nb}_{3}\text{N}$ & $\text{Nb}_{3}\text{Ge}$ & $\text{Ca}_{3}\text{In}$ & $\text{V}_{3}\text{Si}$\tabularnewline
\hline 
$T_{c}\left(\text{K}\right)$$\vphantom{\Big|}$ & 18,05 & 23,2 & 10,4 & 17,1\tabularnewline
\end{tabular}
\par\end{center}


\section{Meissner-Ochsenfeld-Effekt und London-Gleichungen}


\subsubsection*{W. Meissner und R. Ochsenfeld (1933)}

Unterhalb von $T_{c}$ werden magnetische Flusslinien wie in einem
idealen Diamagneten vollständig aus dem Supraleiter verdrängt (Meissner-Ochsenfeld-Effekt).

\textcolor{green}{TODO: Abb121}

\begin{align*}
\chi_{\text{Dia}} & \approx-10^{-5}\\
\chi_{\text{Para}} & \approx10^{-5}
\end{align*}
\textcolor{green}{TODO: Abb122}


\subsubsection*{
\begin{align*}
\chi_{s} & =-1
\end{align*}
Max von Laues Vermutung}

\begin{align}
B & =\mu_{0}\left(H+M\right)=0
\end{align}
Also ist die magnetische Suszeptibilität für einen idealen Diamagneten:
\begin{align}
\fbox{\ensuremath{{\displaystyle \chi_{s}=\frac{M}{H}=-1}}}
\end{align}



\subsubsection*{Fritz und Heinz London (1935)}
\begin{itemize}
\item Im Supraleiter verschwindet der Widerstand und das Magnetfeld:
\begin{align}
R & =0 & B & =0
\end{align}
Nach dem Drude-Modell gilt:
\begin{align}
\sigma_{\text{Drude}} & =\frac{n_{s}q_{s}^{2}}{m_{s}}\tau_{s}\to\infty & \Rightarrow\qquad\tau_{s} & \to\infty
\end{align}
Die Träger des Suprastroms erleiden keine Stöße, das heißt:
\begin{align}
m_{s}\dot{\vec{v}}_{s}+\underbrace{\frac{m_{s}\vec{v}_{s}}{\tau_{s}}}_{=0} & =q_{s}\dot{\vec{E}}
\end{align}

\item Die Stromdichte ist:
\begin{align}
\vec{j}_{s} & =n_{s}q_{s}\vec{v}_{s}\nonumber \\
\dot{\vec{j}}_{s} & =n_{s}q_{s}\dot{\vec{v}}_{s}=\frac{n_{s}q_{s}^{2}}{m_{s}}\cdot\vec{E}
\end{align}
Es folgt die 1. London-Gleichung:
\begin{align}
\fbox{\ensuremath{{\displaystyle \frac{\dd\vec{j}_{s}}{\dd t}=\frac{1}{\mu_{0}\lambda_{L}^{2}}\vec{E}}}}\label{eq:erste-London-Gleichung}
\end{align}
Dabei ist $\lambda_{L}$ die \emph{Londonsche Eindringtiefe}:
\begin{align}
\lambda_{L} & =\sqrt{\frac{m_{s}}{\mu_{0}n_{s}q_{s}^{2}}}
\end{align}

\item Es gilt die Maxwell-Gleichung:
\begin{align}
\vec{\nabla}\times\vec{E}+\frac{\partial\vec{B}}{\partial t} & =0
\end{align}
Mit (\ref{eq:erste-London-Gleichung}) folgt:
\begin{align}
\frac{\partial}{\partial t}\left(\mu_{0}\lambda_{L}^{2}\vec{\nabla}\times\vec{j}_{s}+\vec{B}\right) & =0
\end{align}
Integriere über die Zeit und setze ad hoc die Integrationskonstante
Null. Damit folgt die zweite London-Gleichung:
\begin{align}
\fbox{\ensuremath{{\displaystyle \vec{\nabla}\times\vec{j}_{s}=-\frac{1}{\mu_{0}\lambda_{L}^{2}}\vec{B}}}}
\end{align}
London-Gleichungen statt Drude-Gleichungen!\\
Die physikalische Bedeutung von $\lambda_{L}$ ergibt sich aus den
Maxwell-Gleichungen:
\begin{align}
\vec{\nabla}\times\vec{B} & =\mu_{0}\vec{j}\label{eq:rot_B}
\end{align}
\begin{align}
-\Delta\vec{B}\stackrel{\nabla\vec{B}=0}{=}\vec{\nabla}\times\vec{\nabla}\times\vec{B} & =\mu_{0}\vec{\nabla}\times\vec{j}_{s}=-\frac{1}{\lambda_{L}^{2}}\vec{B}\nonumber \\
\vec{B} & =\lambda_{L}^{2}\Delta\vec{B}\label{eq:Laplacegleichung_B}
\end{align}

\item Beispiel: Betrachte einen in $x$-Richtung unendlich ausgedehnten
supraleitenden Halbraum im Feld $\vec{B}=\left(0,0,B_{0}\right)$.
Die Lösung von (\ref{eq:Laplacegleichung_B}) ist:
\begin{align}
B\left(x\right) & =B_{0}e^{-\frac{x}{\lambda_{L}}}
\end{align}
\textcolor{green}{TODO: Abb123}
\item Der Feldverlauf am Rand des Supraleiters wird durch einen supraleitenden
Abschirmstrom erzeugt:
\begin{align}
\vec{j}_{s}\left(x\right) & \stackrel{\eqref{eq:rot_B}}{=}\frac{1}{\mu_{0}}\vec{\nabla}\times\vec{B}=-\frac{1}{\mu_{0}}\frac{\partial B}{\partial x}\vec{e}_{y}=\frac{B_{0}}{\mu_{0}\lambda_{L}}e^{-\frac{x}{\lambda_{L}}}\vec{e}_{y}=\vec{j}_{s}\left(0\right)e^{-\frac{x}{\lambda_{L}}}
\end{align}

\end{itemize}
%DATE: Mi 30.1.13


\section{Die kritische Feldstärke \texorpdfstring{$H_{c}$}{Hc}}

Die Supraleitung wird durch Magnetfelder $H>H_{c}$ zerstört. $H_{c}$
ist temperaturabhängig:
\begin{align}
H_{c}\left(T\right) & =H_{c}\left(0\right)\left(1-\left(\frac{T}{T_{c}}\right)^{2}\right)\label{eq:Hc(T)}
\end{align}
\textcolor{green}{TODO: Abb124}

Der supraleitende Zustand ist eine thermodynamische Phase, die durch
einen Ordnungsparameter charakterisiert wird, nämlich die supraleitende
Dichte $n_{s}\left(T\right)$. Typische kritische Felder sind:

\noindent \begin{center}
\begin{tabular}{c|c|c|c|c|c}
Element$\vphantom{\Big|}$ & Al & In & Sn & Hg & Nb\tabularnewline
\hline 
\hline 
$T_{c}$ in K$\vphantom{\Big|}$ & 1,14 & 3,4 & 3,72 & 4,16 & 9,2\tabularnewline
\hline 
$\mu_{0}H_{c}\left(0\right)$ in mT$\vphantom{\Big|}$ & 10 & 28,8 & 20,4 & 39,4 & 195\tabularnewline
\end{tabular}
\par\end{center}

\noindent \begin{center}
\begin{tabular}{c|c|c|c|c|c}
Verbindung$\vphantom{\Big|}$ & NbN & $\text{Nb}_{3}\text{Sn}$ & $\text{Nb}_{3}\text{Ge}$ & $\text{V}_{3}\text{Si}$ & $\text{Nb}_{3}\left(\text{AlGe}\right)$\tabularnewline
\hline 
\hline 
$T_{c}$ in K$\vphantom{\Big|}$ & 15,7 & 18,0 & 23,2 & 17,1 & 20,7\tabularnewline
\hline 
$\mu_{0}H_{c}\left(0\right)$ in T$\vphantom{\Big|}$ & 1,5 & 24,5 & 38 & 2,35 & 44\tabularnewline
\end{tabular}
\par\end{center}


\section{Supraleitung als thermodynamischer Zustand}
\begin{itemize}
\item Bedingung für die thermodynamische Phase: Der Endzustand ist unabhängig
vom Weg.\\
\textcolor{green}{TODO: Abb125}\\
Der Meissner-Ochsenfeld-Effekt ist ein Unterschied zum idealen Leiter,
für den nur $R=0$, das heißt $\sigma\to\infty$, gilt.
\item Stabilisierungs- oder Kondensationsenergie der supraleitendend Phase:\\
Die freie Enthalpie (Gibbs-Potential) $G\left(T,P,H\right)$ muss
an der Phasengrenze stetig sein.
\begin{align}
\dd G\left(T,H\right) & =-S\dd T-\mu_{0}M\dd H\label{eq:dG(T,H)}\\
G_{N}\left(T,H_{c}\left(T\right)\right) & =G_{S}\left(T,H_{c}\left(T\right)\right)
\end{align}
Die magnetische Antwort des Systems ist:
\begin{align*}
\chi & =\frac{\partial M}{\partial H}
\end{align*}
Für den Normalleiter ist $M=\chi_{N}H$ mit $\chi_{N}\approx10^{-5}$
(\foreignlanguage{english}{linear response}). Für Supraleiter gilt
$M=-H$ (idealer Diamagnet).\stepcounter{equation}
\begin{align}
G_{N}\left(T,H\right) & =G_{N}\left(T,0\right)-\mu_{0}\underbrace{\chi_{N}}_{\approx10^{-5}}\frac{H^{2}}{2}\tag{\arabic{chapter}.\arabic{equation}a}\\
G_{S}\left(T,H\right) & =G_{S}\left(T,0\right)+\mu_{0}\frac{H^{2}}{2}\tag{\arabic{chapter}.\arabic{equation}b}
\end{align}
\textcolor{green}{TODO: Abb126}
\item An der Phasengrenze gilt also:
\begin{align}
G_{N}\left(T,0\right) & \approx G_{S}\left(T,0\right)+\mu_{0}\frac{H^{2}}{2}
\end{align}

\item Für $H<H_{c}$ ist $G_{S}<G_{N}$ und somit die supraleitende Phase
energetisch günstiger!
\item Es gilt:
\begin{align}
G_{N}\left(T=0\right) & \approx G_{N}\left(T,H_{c}\right)=G_{S}\left(T,H_{c}\right)=G_{S}\left(T,0\right)+\frac{\mu_{0}}{2}H_{c}^{2}\left(T\right)\label{eq:G_N}
\end{align}
\begin{align*}
\Delta G\left(T,H\right) & =G_{N}\left(T,H\right)-G_{S}\left(T,H\right)\approx G_{N}\left(T,0\right)-G_{S}\left(T,0\right)-\frac{\mu_{0}}{2}H^{2}=\\
 & \stackrel{\eqref{eq:G_N}}{=}\frac{\mu_{0}}{2}\left(H_{c}^{2}\left(T\right)-H^{2}\right)
\end{align*}
\begin{align}
\fbox{\ensuremath{{\displaystyle \Delta G\left(T,H\right)=\frac{\mu_{0}}{2}\left(H_{c}^{2}\left(T\right)-H^{2}\right)}}}\label{eq:DeltaG(T,H)}
\end{align}
\textcolor{green}{TODO: Abb127}\\
Die supraleitende Phase wird durch $\Delta G=G_{N}-G_{S}$ stabilisiert.
\end{itemize}

\section{Spezifische Wärme des supraleitenden Zustands}
\begin{itemize}
\item Für Supraleiter:
\begin{align}
c\left(T,H>H_{c}\right) & =c_{N}\left(T\right)=\alpha T^{3}+\gamma T\\
c\left(T,H<H_{c}\right) & =c_{S}\left(T\right)=\alpha T^{3}+\underbrace{\gamma T\cdot a\cdot e^{-\frac{\Delta'}{T}}}_{=c_{S,\text{el}}}
\end{align}
\textcolor{green}{TODO: Abb128}
\item Beachte: Es gibt einen Sprung von $c_{S}\left(T\right)$ bei $T=T_{c}$.
Der elektronische Beitrag zu $c_{S}\left(T\right)$ fällt für $T\to0$
exponentiell ab.\\
\textcolor{green}{TODO: Abb129}\\
Man findet:
\begin{align}
\fbox{\ensuremath{{\displaystyle \Delta c\left(T=T_{c}\right)=c_{S}\left(T_{c}\right)-c_{N}\left(T_{c}\right)=4\mu_{0}\frac{H_{c}^{2}}{T_{c}}}}}
\end{align}
Dies ist in sehr guter Übereinstimmung mit Experimenten! Der Grund
ist:
\begin{align*}
c_{S}-c_{N} & =\frac{\partial U_{S}}{\partial T}-\frac{\partial U_{N}}{\partial T}=T\frac{\partial}{\partial T}\left(S_{S}-S_{N}\right)=\\
 & \stackrel{\eqref{eq:dG(T,H)}}{=}-T\frac{\partial}{\partial T}\left(\frac{\partial G_{S}\left(T,H\right)}{\partial T}-\frac{\partial G_{N}\left(T,H\right)}{\partial T}\right)=\\
 & =-T\frac{\partial^{2}}{\partial T^{2}}\left(G_{S}\left(T,H\right)-G_{N}\left(T,H\right)\right)=\\
 & \stackrel{\eqref{eq:DeltaG(T,H)}}{=}T\frac{\partial^{2}}{\partial T^{2}}\left(\frac{\mu_{0}}{2}\left(H_{c}^{2}\left(T\right)-H^{2}\right)\right)=\\
 & =T\frac{\mu_{0}}{2}\cdot\frac{\partial}{\partial T}\left(2H_{c}\left(T\right)\cdot\frac{\partial H_{c}}{\partial T}\right)=\\
 & =T\mu_{0}\left(\left(\frac{\partial H_{c}}{\partial T}\right)^{2}+H_{c}\left(T\right)\frac{\partial^{2}H_{c}}{\partial T^{2}}\right)
\end{align*}
Dies ist die Rutgers-Formel:
\begin{align}
\fbox{\ensuremath{{\displaystyle c_{S}-c_{N}=T\mu_{0}\left(\left(\frac{\partial H_{c}}{\partial T}\right)^{2}+H_{c}\left(T\right)\frac{\partial^{2}H_{c}}{\partial T^{2}}\right)}}}
\end{align}
Speziell gilt:
\begin{align*}
\left[c_{S}-c_{N}\right]_{T=T_{c}} & \sr ={H_{c}\left(T_{c}\right)=0}{}T_{c}\mu_{0}\left(\frac{\partial H_{c}}{\partial T}\right)^{2}\bigg|_{T=T_{c}}=\\
 & \stackrel{\eqref{eq:Hc(T)}}{=}T_{c}\mu_{0}\left(-H_{c}\left(0\right)\cdot2\frac{T}{T_{c}^{2}}\right)^{2}\bigg|_{T=T_{c}}=4\mu_{0}\frac{H_{c}^{2}\left(0\right)}{T_{c}}
\end{align*}
\qqed
\end{itemize}

\section{Energielücke und Elektronenpaarung}
\begin{itemize}
\item Der Sprung von $c_{S}\left(T_{c}\right)$ gibt einen Hinweis auf eine
Energielücke.\\
Dies ermöglicht den direkten Nachweis durch Mikrowellenabsorption.
(Experiment von Glores und Tinkham, 1956)\\
\textcolor{green}{TODO: Abb130; Messung der Oberflächenströme $I_{N}$
für $T>T_{c}$ beziehungsweise $I_{S}$ für $T<T_{c}$.}\\
\textcolor{green}{TODO: Abb131}\\
Energielücke:
\begin{align}
\fbox{\ensuremath{{\displaystyle h\nu_{G}=2\Delta}}}
\end{align}
Die Energielücke skaliert mit der Übergangstemperatur. Experimentell
findet man:
\begin{align}
\fbox{\ensuremath{{\displaystyle 2\Delta\approx3{,}5\cdot k_{B}T_{c}}}}
\end{align}



\noindent \begin{center}
\begin{tabular}{c||c|c|c|c|c}
Element$\vphantom{\Big|}$ & Al & In & Sn & Pb & Nb\tabularnewline
\hline 
${\displaystyle \frac{n\nu_{G}}{k_{B}T_{c}}\vphantom{\bigg|}}$ & 3,4 & 3,5 & 3,55 & 4,11 & 3,5 - 3,8\tabularnewline
\end{tabular}
\par\end{center}

\item Die Energielücke ist temperaturabhängig:
\begin{align}
\fbox{\ensuremath{{\displaystyle \Delta\left(T\right)=\Delta\left(0\right)\sqrt{1-\frac{T}{T_{c}}}}}}
\end{align}
\textcolor{green}{TODO: Abb132 (BCS: Bardeen, Cooper, Schrieffer)}
\end{itemize}

\subsubsection*{Zustandsdichte}

\textcolor{green}{TODO: Abb133, Abb134}
\begin{align*}
\fbox{\ensuremath{{\displaystyle \frac{2\Delta}{E_{F}}\approx10^{-4}}}}
\end{align*}



\section{Die BCS-Theorie}

Bardeen, Cooper und Schrieffer (Nobelpreis 1972)


\subsubsection*{Experimentelle Vorarbeiten}
\begin{enumerate}
\item spezifische Wärme
\item Energielücke
\item Der Isotopieeffekt bestätigt die BCS-Theorie.
\begin{align*}
M^{\frac{1}{2}}\cdot T_{c} & =\text{konst.}
\end{align*}

\item Flussquantisierung (Nachweis der Cooper Paare)\\
Der magnetischer Fluss $B$ durch den supraleitenden Ring kann nur
ganzzahlige Vielfache von $\frac{h}{q}$ annehmen, wobei $q=2e$ (Cooper-Paar)
ist.
\end{enumerate}

\subsubsection*{Erklärung der Supraleitung in konventionellen Supraleitern}
\begin{itemize}
\item Für $T<T_{c}$ bilden sich Cooper-Paare. Dies sind zwei Elektronen
mit gegensätzlichem Impuls und Spin.
\item Es handelt sich um einen gebundenen Zustand, wobei die anziehende
Wechselwirkung durch Austausch virtueller Phononen (dynamische Gitterdeformation)
verursacht wird.\\
\textcolor{green}{TODO: Abb135}
\item Cooper-Paare sind Bosonen, haben also einen ganzzahligen Spin. Für
$T<T_{c}$ „kondensieren“ (einige) freie Elektronen als Cooper-Paare
in einen bosonischen Grundzustand. Dieser kann durch eine makroskopische
Wellenfunktion beschrieben werden. Für $k_{B}T\ge2\Delta$ werden
die Cooper-Paare aufgebrochen.
\end{itemize}

\subsubsection*{Anwendungen für Supraleiter}

SQUID (\foreignlanguage{english}{Superconducting QUantum-Interference
Device}), MRI (\foreignlanguage{english}{Magnetic Resonance Imaging})

%DATE: Do 31.1.13


\chapter{Magnetismus}


\section{Klassifizierung magnetischer Eigenschaften}

Magnetisierung:
\begin{align}
\vec{M} & =\chi\vec{H}=\frac{1}{\mu_{0}\mu_{r}}\chi\vec{B}\label{eq:Magnetisierung}
\end{align}
Die magnetische Suszeptibilität ist im Allgemeinen ein Tensor: 
\begin{align*}
\chi & =\chi\left(\vec{q},\omega\right)
\end{align*}
\begin{align}
\chi & =\mu_{r}-1\\
\vec{B} & =\mu_{r}\mu_{0}\vec{H}=\mu_{0}\left(\vec{H}+\vec{M}\right)
\end{align}
\begin{framed}
\begin{align*}
-1\le & \chi<0 &  & \text{diamagnetische Substanzen}\\
 & \chi>0 &  & \text{paramagnetische Substanzen}\\
 & \chi\gg1 &  & \text{ferromagnetische Substanzen}
\end{align*}
\end{framed}

Dabei sind ferromagnetische Substanzen generell geordnete Spinstrukturen.

Ursachen des magnetischen Moments des \emph{freien} Atoms:

\noindent \begin{center}
\begin{tabular}{c|c}
Paramagnetismus$\vphantom{\bigg|}$ & Diamagnetismus\tabularnewline
\hline 
Elektronenspin und$\vphantom{\Big|}$ & Änderung des\tabularnewline
Bahndrehimpuls$\vphantom{\Big|}$ & Bahndrehimpulses\tabularnewline
unvollständig gefüllter$\vphantom{\Big|}$ & durch äußeres\tabularnewline
Elektronenschalen$\vphantom{\Big|}$ & Magnetfeld\tabularnewline
\end{tabular}
\par\end{center}

Gefüllte Elektronenschalen haben $L_{\text{ges}}=S_{\text{ges}}=0$.
Die Kernmomente sind vernachlässigbar, da $\mu_{\text{Kern}}\approx10^{-3}\mu_{\text{Elektron}}$.


\section{Die Langevin-Gleichung für Diamagnetismus}
\begin{itemize}
\item Der Diamagnetismus wird durch ein äußeres Magnetfeld induziert.
\item Dies ist eine Eigenschaft aller Elektronen! (Normalerweise überwiegen
stärkere Effekte.)
\item \emph{Larmor-Theorem}: Das $\vec{B}$-Feld verursacht eine zusätzliche
Bewegung der Elektronen um den Atomkern mit Kreisfrequenz:
\begin{align}
\omega_{L} & =\frac{eB}{2m_{e}}
\end{align}

\item \emph{Lenzsche Regel}: Das $\vec{B}$-Feld verursacht einen Strom,
dessen magnetisches Moment dem äußeren Feld entgegen gerichtet ist.
\begin{align}
j & =-Ze\nu_{L}=-Ze\frac{1}{2\pi}\cdot\frac{eB}{2m_{e}}
\end{align}

\item In einem Kreis ohne Widerstand (Supraleiter oder Elektronenbahn im
Atom) fließt der Strom so lange, wie das Feld anliegt.\\
\textcolor{green}{TODO: Abb136; magnetisches Moment}
\begin{align}
\mu & =\text{Strom}\cdot\text{Fläche}=j\cdot\pi\left\langle \varrho^{2}\right\rangle =-\frac{Ze^{2}B}{4m_{e}}\left\langle \varrho^{2}\right\rangle \label{eq:mu}
\end{align}
\begin{align*}
\left\langle \varrho^{2}\right\rangle  & =\left\langle x^{2}\right\rangle +\left\langle y^{2}\right\rangle 
\end{align*}
$\left\langle \varrho^{2}\right\rangle $ ist das mittlere Abstandsquadrat
des Elektrons senkrecht zur Feldachse.
\begin{align*}
\left\langle r^{2}\right\rangle  & =\left\langle x^{2}\right\rangle +\left\langle y^{2}\right\rangle +\left\langle z^{2}\right\rangle 
\end{align*}
Beim Diamagnetismus liegt eine kugelsymmetrische Ladungsverteilung
vor, sonst erhält man wegen $L\not=0$ Paramagnetismus. Es folgt:
\begin{align}
\left\langle r^{2}\right\rangle  & =\frac{3}{2}\left\langle \varrho^{2}\right\rangle \label{eq:r2_mittel}
\end{align}

\item Aus (\ref{eq:Magnetisierung}), (\ref{eq:mu}) und (\ref{eq:r2_mittel})
folgt:
\begin{align}
\fbox{\ensuremath{{\displaystyle \chi_{\text{dia}}=\frac{M\mu_{r}\mu_{0}}{B}=\frac{\mu N\mu_{0}}{B}=-\mu_{0}\frac{ZNe^{2}}{6m_{e}}\left\langle r^{2}\right\rangle }}}
\end{align}
$N$ ist die Anzahl der Atome pro Volumeneinheit.
\item Dies liefert einen Beitrag zu allen Festkörpern (auch Metallen)!
\end{itemize}

\section{Diamagnetismus freier Elektronen}
\begin{itemize}
\item Ein klassisches freies Elektronengas zeigt nach dem \emph{Bohr-von
Leuwen-Theorem} keinen Diamagnetismus, das heißt:
\begin{align*}
\chi_{\text{dia}}^{\text{klass}} & =0
\end{align*}
Also müssen wir quantenmechanisch rechnen.
\item Betrachte freie Elektronen im Magnetfeld:
\begin{align*}
\vec{B} & =\left(0,0,B\right)
\end{align*}
Das Vektorpotential $\vec{A}$ in Coulomb Eichung ist:
\begin{align*}
\vec{A} & =\left(0,Bx,0\right)
\end{align*}
In der Schrödingergleichung muss $\vec{p}$ durch $\vec{p}+e\vec{A}$
ersetzt werden.
\begin{align*}
\frac{\left(\vec{p}+e\vec{A}\right)^{2}}{2m}\psi & =E\psi
\end{align*}
Dies führt zu den Eigenenergien:
\begin{align*}
E_{l}\left(k_{z},B\right) & =\frac{\hbar^{2}k_{z}^{2}}{2m^{*}}+\hbar\omega_{c}\left(l+\frac{1}{2}\right)
\end{align*}
Dabei ist $\omega_{c}=\frac{eB}{m^{*}}$ die \emph{Zyklotronfrequenz}
und $l\in\mathbb{N}_{\ge0}$. Die erlaubten Zustände im $k$-Raum
liegen auf \emph{Landau-Röhren}:\\
\textcolor{green}{TODO: Abb137}

\begin{itemize}
\item Die Zahl der Landau-Röhren innerhalb der Fermikugel nimmt mit wachsendem
$B$ ab, da $\omega_{c}\sim B$ ist.
\item Die Entartung der Zustände wächst mit $B$, da die Zahl der Elektronen
konstant ist.
\end{itemize}
\item Landaus Rechnung (Skizze):

\begin{itemize}
\item Freie Energie:
\begin{align*}
F_{\text{dia}}\left(T,B\right) & =-k_{B}T\ln\left(Z\right)
\end{align*}

\item Zustandssumme:
\begin{align*}
Z & =\sum_{l,m_{l}}\ln\left(1+e^{\frac{-E_{l}-E_{f}}{k_{B}T}}\right)
\end{align*}

\item Magnetisierung:
\begin{align*}
M_{\text{dia}}\left(T,B\right) & =\frac{1}{V}\left(\frac{\partial F_{\text{dia}}}{\partial B}\right)
\end{align*}

\item Landau Suszeptibilität:
\begin{align}
\fbox{\ensuremath{{\displaystyle \chi_{\text{dia}}^{\text{Landau}}=\frac{M_{\text{dia}}}{H}=-\frac{1}{2}n\frac{\mu_{B}^{2}\mu_{0}}{k_{B}T_{F}}\cdot\frac{m^{*}}{m_{e}}}}}\label{eq:chi-Landau}
\end{align}
Dabei ist
\begin{align}
\fbox{\ensuremath{{\displaystyle \mu_{B}=\frac{e\hbar}{2m_{e}}}}}
\end{align}
das Bohrsche Magneton und $n$ die Dichte freier Elektronen.
\end{itemize}
\end{itemize}

\section{Paramagnetismus gebundener Elektronen}

Paramagnetismus tritt auf bei
\begin{enumerate}
\item Atomen, Molekülen und Gitterfehlstellen mit einer ungeraden Anzahl
von Elektronen,
\item freien Atomen oder Ionen mit teilweise gefüllten inneren Schalen,
Übergangselemente, seltene Erden, Aktiniden und
\item wenigen Verbindungen mit gerader Elektronenzahl (z.B. $\text{O}_{2}$).\end{enumerate}
\begin{itemize}
\item Das magnetische Moment eines Atoms oder Ions ist:
\begin{align}
\vec{\mu} & =\gamma\hbar\vec{j}=-g\mu_{B}\vec{j}\label{eq:magn-Moment}
\end{align}
$\gamma$ ist das \emph{gyromagnetische Verhältnis} und $g$ der \emph{Landé-Faktor}.\\
Für den Elektronenspin folgt aus der QED:
\begin{align*}
g & =2,0023\approx2
\end{align*}
Für das freie Atom erhält man:
\begin{align}
g & =1+\frac{j\left(j+1\right)+s\left(s+1\right)-l\left(l+1\right)}{2j\left(j+1\right)}
\end{align}

\item Für die Energieniveaus im Magnetfeld folgt:
\begin{align}
\vec{E} & =-\vec{\mu}\cdot\vec{B}\stackrel{\eqref{eq:magn-Moment}}{=}g\mu_{B}\vec{J}\cdot\vec{B}=m_{J}g\mu_{B}B
\end{align}
Dabei ist $m_{J}\in\left\{ -J,-J+1,\ldots,J\right\} $ die magnetische
Quantenzahl.
\item Beispiel: Elektronenspin:
\begin{align*}
B_{J} & =\pm\frac{1}{2}
\end{align*}
\begin{align*}
m_{J} & =\pm\frac{1}{2} & g & =2 & E & =\pm\mu_{B}B
\end{align*}
\textcolor{green}{TODO: Abb138=Abb139}
\begin{align*}
N & =N_{1}+N_{2}
\end{align*}
ist die Gesamtzahl der Elektronen.%DATE: Mi 6.2.13
\begin{align}
\frac{N_{1}}{N} & =\frac{N_{1}}{N_{1}+N_{2}}=\frac{\exp\left(\frac{\mu_{B}B}{k_{B}T}\right)}{\exp\left(\frac{\mu_{B}B}{k_{B}T}\right)+\exp\left(-\frac{\mu_{B}B}{k_{B}T}\right)}\\
\frac{N_{2}}{N} & =\frac{N_{2}}{N_{1}+N_{2}}=\frac{\exp\left(-\frac{\mu_{B}B}{k_{B}T}\right)}{\exp\left(\frac{\mu_{B}B}{k_{B}T}\right)+\exp\left(-\frac{\mu_{B}B}{k_{B}T}\right)}
\end{align}
\textcolor{green}{TODO: Abb140}\\
Mit $x=\frac{\mu_{B}B}{k_{B}T}$ folgt:
\begin{align}
M & =N_{1}\mu_{B}-N_{2}\mu_{B}=N\mu_{B}\frac{e^{x}-e^{-x}}{e^{x}+e^{-x}}=N\mu_{B}\tanh\left(x\right)
\end{align}
Für $x\ll1$ gilt außerdem $\tanh\left(x\right)\approx x$ und es
folgt:
\begin{align}
\fbox{\ensuremath{{\displaystyle M\approx N\mu_{B}\cdot\frac{\mu_{B}B}{k_{B}T}}}}
\end{align}
Dies ist der \emph{Curie-Paramagnetismus}.
\item Allgemein für ein Atom mit Gesamtdrehimpuls $J$ erhält man $2J+1$
äquidistante Niveaus für $B>0$.\begin{framed}
\begin{align}
M & =Ng_{J}J\mu_{B}B_{J}\left(x\right) & x & =\frac{g_{J}J\mu_{B}B}{k_{B}T}
\end{align}
\end{framed}Dabei ist $B_{J}\left(x\right)$ die Brillouin-Funktion:
\begin{align}
B_{J}\left(x\right) & =\frac{2J+1}{2J}\coth\left(\frac{\left(2J+1\right)x}{2J}\right)-\frac{1}{2J}\coth\left(\frac{x}{2J}\right)
\end{align}
\textcolor{green}{TODO: Abb141}
\item Für $x\ll1$ gilt ${\displaystyle \coth\left(x\right)=\frac{1}{x}+\frac{x}{3}+\frac{x^{3}}{45}+\mathcal{O}_{0}\left(x^{5}\right)}$.
\begin{align}
\fbox{\ensuremath{{\displaystyle \chi_{\text{Para}}\approx\mu_{0}\frac{N\left(J+1\right)Jg_{J}^{2}\mu_{B}^{2}}{3k_{B}T}=\frac{C}{T}}}}
\end{align}
Dies ist das \emph{Curie-Gesetz}.\\
\textcolor{green}{TODO: Abb142}
\end{itemize}

\section{Paramagnetische Suszeptibilität der Leitungselektronen}

Dies nennt man auch die \emph{Pauli-Suszeptibilität}.

„Klassisch“ würde man auch für freie Elektronen ein $\frac{1}{T}$-Curie-Gesetz
erwarten (aufgrund des Spins $s=\frac{1}{2}$, $m_{s}=\pm\frac{1}{2}$).
Experimentell ergibt sich aber für nicht (anti-)ferromagnetische Metalle
meist ein $T$-unabhängiges $\chi$. Hier muss man also die Zustandsdichte
und die Fermiverteilung berücksichtigen!

\textcolor{green}{TODO: Abb143}

Man hat also einen Überschuss von Spin down Elektronen.
\begin{align*}
N_{\downarrow} & =\frac{1}{2}\int_{-\mu_{B}B}^{E_{F}}\dd Ef\left(E\right)\mathcal{D}\left(E+\mu_{0}B\right)\approx\frac{1}{2}\int_{0}^{E_{F}}\dd Ef\left(E\right)\mathcal{D}\left(E\right)+\frac{1}{2}\mu_{B}BD\left(E_{F}\right)\\
N_{\uparrow} & =\frac{1}{2}\int_{+\mu_{B}B}^{E_{F}}\dd Ef\left(E\right)\mathcal{D}\left(E+\mu_{0}B\right)\approx\frac{1}{2}\int_{0}^{E_{F}}\dd Ef\left(E\right)\mathcal{D}\left(E\right)-\frac{1}{2}\mu_{B}BD\left(E_{F}\right)
\end{align*}
Mit $d\left(E_{F}\right)=\frac{3}{2}\frac{N}{k_{B}T_{F}}$ und $M=\mu_{B}\left(N_{\downarrow}-N_{\uparrow}\right)$
folgt:
\begin{align*}
M & =\mu_{B}^{2}\mathcal{D}\left(E_{F}\right)\cdot B=\frac{3}{2}\frac{n\mu_{B}^{2}}{k_{B}T_{F}}\cdot B=\frac{\chi_{\text{Para}}^{\text{Pauli}}}{\mu_{0}}B
\end{align*}
Man erhält so die \emph{Pauli-Spin-Suszeptibilität}.
\begin{align}
\fbox{\ensuremath{{\displaystyle \chi_{\text{Para}}^{\text{Pauli}}=\frac{3}{2}\mu_{0}\frac{n\mu_{B}^{2}}{k_{B}T_{F}}\stackrel{n\text{ const.}}{=}\text{const.}}}}
\end{align}

\begin{itemize}
\item Zusammen mit der diamagnetischen Suszeptibilität $\chi_{\text{Dia}}^{\text{Landau}}$
in Gleichung (\ref{eq:chi-Landau}) erhält man die Gesamtsuszeptibilität
freier Elektronen im Festkörper:
\begin{align}
\fbox{\ensuremath{{\displaystyle \chi^{\text{el}}=\chi_{\text{Dia}}^{\text{Landau}}+\chi_{\text{Para}}^{\text{Pauli}}=\left(-\frac{1}{2}+\frac{3}{2}\right)\mu_{0}\frac{n\mu_{B}^{2}}{k_{B}T_{F}}=\mu_{0}\frac{n\mu_{B}^{2}}{k_{B}T_{F}}}}}
\end{align}
$\chi_{\text{Dia}}^{\text{Landau}}$ kommt von der Bahnbewegung und
$\chi_{\text{Para}}^{\text{Pauli}}$ kommt vom Spin der Elektronen.\\
Bis jetzt wurde noch nicht berücksichtigt:

\begin{itemize}
\item diamagnetischer Beitrag der Gitterionen
\item paramagnetischer Beitrag der Gitterionen
\item Bandstruktureffekte (verändern Zustandsdichte, effektive Masse!)
\item Elektron-Elektron-Wechselwirkung
\end{itemize}
\end{itemize}

\section{Ferromagnetismus}

\begin{framed}

Ein Ferromagnet besitzt ein spontanes magnetisches Moment $\vec{M}$,
das heißt ein Nettomoment auch ohne äußeres Feld. (innerhalb einer
magnetischen Domäne)

\end{framed}

\textcolor{green}{TODO: Abb144, 145}

Paramagnetismus:
\begin{align*}
\chi & =\frac{C}{T}
\end{align*}
Ferromagnetismus:
\begin{align*}
\chi & =\frac{C'}{T-T_{c}}
\end{align*}
Dieses Curie-Weiss-Gesetz gilt für Temperaturen $T$ oberhalb der
Curie-Temperatur $T_{c}$.

Antiferromagnet:
\begin{align*}
\chi & =\frac{2C''}{T+\theta}
\end{align*}
Dies gilt für Temperaturen $T$ über der Neél-Temperatur $T_{N}$.

Einige Ferromagnete:

\noindent \begin{center}
\begin{tabular}{c|c}
$\vphantom{\Big|}$ & $T_{c}$ in K\tabularnewline
\hline 
\hline 
Fe$\vphantom{\Big|}$ & 1043\tabularnewline
\hline 
Co$\vphantom{\Big|}$ & 1388\tabularnewline
\hline 
Ni$\vphantom{\Big|}$ & 627\tabularnewline
\end{tabular}\qquad{}%
\begin{tabular}{c|c}
$\vphantom{\Big|}$ & $T_{c}$ in K\tabularnewline
\hline 
\hline 
MnAs$\vphantom{\Big|}$ & 318\tabularnewline
\hline 
$\text{CrO}_{2}$$\vphantom{\Big|}$ & 386\tabularnewline
\hline 
FeO, $\text{Fe}_{2}\text{O}_{3}$$\vphantom{\Big|}$ & 858\tabularnewline
\end{tabular}
\par\end{center}
\begin{itemize}
\item Temperaturabhängigkeit der Sättigungsmagnetisierung:\\
\textcolor{green}{TODO: Abb146; $H_{c}$: Koerzitivfeld; $M_{R}$:
Remanenz}\\
\textcolor{green}{TODO: Abb147}
\begin{align*}
M & \sim\left(1-\frac{T}{T_{c}}\right)^{\beta}
\end{align*}
Es ist $\beta=0{,}125$ nach dem Ising-Modell, $\beta=0{,}5$ nach
der \foreignlanguage{english}{Mean Field theory}.
\item Einige Antiferromagnete:
\end{itemize}
\noindent \begin{center}
\begin{tabular}{c|c}
$\vphantom{\Big|}$ & $T_{N}$ in K\tabularnewline
\hline 
\hline 
MnO$\vphantom{\Big|}$ & 116\tabularnewline
\hline 
Cr$\vphantom{\Big|}$ & 308\tabularnewline
\hline 
NiO$\vphantom{\Big|}$ & 525\tabularnewline
\hline 
FeO$\vphantom{\Big|}$ & 198\tabularnewline
\hline 
MnS$\vphantom{\Big|}$ & 160\tabularnewline
\hline 
CoO$\vphantom{\Big|}$ & 291\tabularnewline
\end{tabular}
\par\end{center}

%DATE: Do 7.2.13


\section{Energiebeiträge im ferromagnetischen Systemen}
\begin{enumerate}[label=\roman*)]
\item Die \emph{Austauschenergie} wird verursacht durch die Wechselwirkung
der magnetischen Momente der Elektronenspins (allgemeiner des Gesamtdrehimpuls)
mittels Austauschwechselwirkung durch direkten Überlapp der Wellenfunktionen
(nicht Dipol-Dipol-Wechselwirkung). Dies bezeichnet man als \emph{Heisenberg-Modell}:
\begin{align}
\fbox{\ensuremath{{\displaystyle E_{\text{ex}}=-2\sum_{i\not=j}I\left(\vec{r}_{ij}\right)\vec{S}_{i}\vec{S}_{j}}}}
\end{align}
$\vec{S}_{i}$ ist der Spin am Ort $i$ und $I\left(\vec{r}_{ij}\right)$
ist das Austauschintegral. Ist $I\left(\vec{r}_{ij}\right)>0$, so
erhält man eine parallele Orientierung der Spins und für $I\left(\vec{r}_{ij}\right)<0$
erhält man eine anti-parallele Orientierung der Spins.

\begin{itemize}
\item Dies setzt einen direkten Überlapp der Wellenfunktionen voraus (oft
nächste Nachbarn).
\item Seltenes Phänomen!\\
Superaustausch (Austausch über Ionen)\\
RKKY-Wechselwirkung
\end{itemize}
\item Die \emph{Zeeman-Energie} ist die potentielle Energie im äußeren Magnetfeld:
\begin{align}
E_{\text{Z}} & =-\mu_{0}\int_{V}\vec{H}\cdot\vec{M}\dd V
\end{align}

\item Die Dipol-Dipol-Wechselwirkung ruft die \emph{Demagnetisierungsenergie},
also die Energie, die im Streufeld des Magneten gespeichert ist, hervor:
\begin{align*}
E_{\text{D}} & =-\frac{1}{2}\mu_{0}\int_{V}\vec{H}_{\text{eigen}}\cdot\vec{M}\dd V
\end{align*}
$\vec{H}_{\text{eigen}}$ ist das von der Magnetisierung erzeugte
Feld.\\
\textcolor{green}{TODO: Abb148}
\item Die \emph{Anisotropieenergie} $E_{A}$ beschreibt, dass die Richtung
von $\vec{M}$ von den Kristallrichtungen abhängen kann, also eine
Anisotropie vorliegt.\\
\textcolor{green}{TODO: Abb149}
\end{enumerate}
Um die Domänenkonfiguration zu finden, minimiere die Summe aller Energien:
\begin{align*}
\fbox{\ensuremath{{\displaystyle E_{\text{ex}}+E_{\text{Z}}+E_{\text{D}}+E_{\text{A}}\to\text{Minimum!}}}}
\end{align*}
Den Rest erfährt man in einer Magnetismus-Vorlesung.


\section{Heisenberg-Operator}



Ursprung: Coulomb-Wechselwirkung und Pauli-Prinzip

Einfaches Beispiel: zwei Wasserstoff-Atome

\textcolor{green}{TODO: Abb150}

$\vec{r}_{1}$ ist die Position von $e_{1}^{-}$ und $\vec{r}_{2}$
ist die Position von $e_{2}^{-}$ und $\vec{R}_{1}$ und $\vec{R}_{2}$
sind die Positionen der Atomkerne. Der \emph{Hamiltonoperator} ist:
\begin{align*}
H & =-\frac{\hbar^{2}}{2m}\left(\vec{\nabla}_{1}^{2}+\vec{\nabla}_{2}^{2}\right)+V\left(\vec{r}_{1},\vec{r}_{2}\right)
\end{align*}
Ohne Überlapp folgt:
\begin{align*}
V\left(\vec{r}_{1},\vec{r}_{2}\right) & =-\frac{e^{2}}{\norm{\vec{r}_{1}-\vec{R}_{1}}}-\frac{e^{2}}{\norm{\vec{r}_{2}-\vec{R}_{2}}}
\end{align*}
\textcolor{green}{TODO: Abb151}

Die Gesamtwellenfunktion des Systems ist antisymmetrisch.
\begin{align*}
\psi & =\phi_{1}\chi_{1}\cdot\phi_{2}\chi_{2}
\end{align*}
Falls der Überlapp nicht verschwindet, so enthält $V\left(\vec{r}_{1},\vec{r}_{2}\right)$
auch die Wechselwirkung zwischen $e_{1}^{-}$ und $e_{2}^{-}$. Man
definiert:
\begin{align*}
\psi_{\alpha} & =\phi_{s}\left(\vec{r}_{1},\vec{r}_{2}\right)\chi_{a}\left(s_{1},s_{2}\right)\\
\psi_{\beta} & =\phi_{a}\left(\vec{r}_{1},\vec{r}_{2}\right)\chi_{s}\left(s_{1},s_{2}\right)
\end{align*}
$s$ steht für symmetrisch und $a$ für antisymmetrisch:
\begin{align*}
\phi_{\sr{}sa}\left(\vec{r}_{1},\vec{r}_{2}\right) & =\pm\phi_{\sr{}sa}\left(\vec{r}_{2},\vec{r}_{1}\right)\\
\chi_{\sr{}sa}\left(\vec{r}_{1},\vec{r}_{2}\right) & =\pm\chi_{\sr{}sa}\left(\vec{r}_{2},\vec{r}_{1}\right)
\end{align*}
Somit ergeben sich vier Wellenfunktionen für die antisymmetrische
Gesamtwellenfunktion:
\begin{align*}
\KET{\uparrow\uparrow},\ \KET{\uparrow\downarrow},\ \KET{\downarrow\uparrow},\ \KET{\downarrow\downarrow}
\end{align*}
Symmetrische Ortswellenfunktion: Singulett
\begin{align*}
\psi_{S} & =\frac{1}{\sqrt{2}}\left(\phi_{1}\left(\vec{r}_{1}\right)\phi_{2}\left(\vec{r}_{2}\right)+\phi_{1}\left(\vec{r}_{2}\right)\phi_{2}\left(\vec{r}_{1}\right)\right)\cdot\frac{1}{\sqrt{2}}\underbrace{\left(\KET{\uparrow\downarrow}-\KET{\downarrow\uparrow}\right)}_{\text{Singulett, }s=0,\, s_{z}=0}
\end{align*}
Antisymmetrische Ortswellenfunktion: Triplett
\begin{align*}
\psi_{T} & =\frac{1}{\sqrt{2}}\left(\phi_{1}\left(\vec{r}_{1}\right)\phi_{2}\left(\vec{r}_{2}\right)-\phi_{1}\left(\vec{r}_{2}\right)\phi_{2}\left(\vec{r}_{1}\right)\right)\cdot\left(\begin{array}{c}
\KET{\uparrow\uparrow}\\
\frac{1}{\sqrt{2}}\left(\KET{\uparrow\downarrow}+\KET{\downarrow\uparrow}\right)\\
\KET{\downarrow\downarrow}
\end{array}\right)\begin{array}{c}
s=1,\ s_{z}=1\\
s=1,\ s_{z}=0\\
s=1,\ s_{z}=-1
\end{array}
\end{align*}
Löse die Schrödingergleichung um die Energieeigenwerte zu finden.
\begin{align*}
\hat{H}\psi & =E\psi
\end{align*}
Dies liefert $E_{S}$ und $E_{T}$.
\begin{align*}
E_{S}-E_{T} & =\frac{\BraKet{\psi_{S}|\hat{H}|\psi_{S}}}{\left\langle \psi_{S}|\psi_{S}\right\rangle }-\frac{\BraKet{\psi_{T}|\hat{H}|\psi_{T}}}{\left\langle \psi_{T}|\psi_{T}\right\rangle }
\end{align*}



\subsubsection*{Definition}

Das \emph{Austausch-Integral} ist definiert als:
\begin{align*}
J_{12} & =\frac{1}{2}\left(E_{S}-E_{T}\right)=\int\dd^{3}r_{1}\int\dd^{3}r_{2}\left(\phi_{1}^{*}\left(\vec{r}_{1}\right)\phi_{2}^{*}\left(\vec{r}_{2}\right)\,\hat{H}_{12}\,\phi_{2}\left(\vec{r}_{1}\right)\phi_{1}\left(\vec{r}_{2}\right)\right)
\end{align*}
Dabei ist:
\begin{align*}
\hat{H}_{12} & =\frac{e^{2}}{\norm{\vec{r}_{1}-\vec{r}_{2}}}+\frac{e^{2}}{\norm{\vec{R}_{1}-\vec{R}_{2}}}-\frac{e^{2}}{\norm{\vec{r}_{1}-\vec{R}_{2}}}-\frac{e^{2}}{\norm{\vec{r}_{2}-\vec{R}_{1}}}
\end{align*}
Das \emph{Coulomb-Integral} ist definiert als:
\begin{align*}
K_{12} & =\int\dd^{3}r_{1}\int\dd^{3}r_{2}\left(\phi_{1}^{*}\left(\vec{r}_{1}\right)\phi_{2}^{*}\left(\vec{r}_{2}\right)\,\hat{H}_{12}\,\phi_{1}\left(\vec{r}_{1}\right)\phi_{2}\left(\vec{r}_{2}\right)\right)
\end{align*}
Man erhält:
\begin{align*}
E_{S} & =K_{12}+J_{12} & E_{T} & =K_{12}-J_{12}
\end{align*}
\begin{align*}
\Rightarrow\qquad E_{S}-E_{T} & =2J_{12}
\end{align*}
Bisher ist dies unabhängig vom Spin. Die Spinabhängigkeit kommt durch
das Pauli-Prinzip.
\begin{align*}
H_{12}\psi & =H_{12}\left(\phi\cdot\chi\right)=\chi\left(H_{12}\phi\right)=\chi E_{S,T}\phi=E_{S,T}\psi
\end{align*}
Es muss einen Spinoperator geben, der die gleichen Eigenwerte liefert.
Man weiß:
\begin{align*}
\vec{S}_{i}^{2} & =\frac{1}{2}\left(\frac{1}{2}+1\right)=\frac{3}{4}\\
\vec{S}^{2} & =\left(\vec{S}_{1}+\vec{S}_{2}\right)^{2}=\frac{3}{2}+2\cdot\vec{S}_{1}\cdot\vec{S}_{2}
\end{align*}
$\vec{S}^{2}$ hat den Eigenwert $S\left(S+1\right)$ und der Operator
$\vec{S}_{1}\cdot\vec{S}_{2}=\frac{1}{2}\left(\vec{S}^{2}-\frac{3}{2}\right)$
hat die Eigenwerte $-\frac{3}{4}$ (Singulett, $S=0$) und $\frac{1}{4}$
(Triplett, $S=1$). Also liefert der Operator
\begin{align*}
\fbox{\ensuremath{{\displaystyle H^{\text{Spin}}=\frac{1}{4}\left(E_{S}+3E_{T}\right)-\left(E_{S}-E_{T}\right)\vec{S}_{1}\cdot\vec{S}_{2}}}}
\end{align*}
die Eigenwerte $E_{S}$ im Singulett-Zustand und $E_{T}$ im Triplett-Zustand.
Verschiebe den Energienullpunkt:
\begin{align*}
\fbox{\ensuremath{{\displaystyle H^{\text{Spin}}=-2J_{12}\vec{S}_{1}\cdot\vec{S}_{2}}}}
\end{align*}
\begin{align*}
2J_{12} & =E_{S}-E_{T}
\end{align*}
Für $n$-Elektronen-System erhält man den \emph{Heisenberg-Dirac-Operator}:
\begin{align*}
\fbox{\ensuremath{{\displaystyle H_{\text{Heis.-Dirac}}=-2\sum_{i\not=j}J_{ij}\vec{S}_{i}\cdot\vec{S}_{j}}}}
\end{align*}

\begin{itemize}
\item Typischerweise verwendet man nur die nächste-Nachbar-Wechselwirkung.
\item Die Wechselwirkung wird isotrop angenommen.
\end{itemize}
Damit folgt:
\begin{align*}
\fbox{\ensuremath{{\displaystyle H_{\text{Heis.}}=-2J\sum_{\text{n.n.}}\vec{S}_{i}\cdot\vec{S}_{j}}}}
\end{align*}
Dies funktioniert gut für $4f$- und $3d$-Systeme. In der \foreignlanguage{english}{Mean
field approximation} ist der Heisenberg-Operator:
\begin{align*}
H_{\text{Heis.}} & =-2J\sum_{\text{n.n.}}\vec{S}_{i}\underbrace{\left\langle \vec{S}_{j}\right\rangle _{T}}_{\text{wie Feld}}
\end{align*}



\part*{Anhang\thispagestyle{empty}}

\addcontentsline{toc}{part}{Anhang}

\fancyhead[R]{Index}
\fancyhead[C]{Anhang}


\chapter*{Danksagungen}

\addcontentsline{toc}{section}{\hspace*{2.7em}Danksagungen}

\fancyhead[R]{Danksagungen}

Mein besonderer Dank geht an Professor Back, der diese Vorlesung hielt
und es mir gestattete, diese Vorlesungsmitschrift zu veröffentlichen.

Außerdem möchte ich mich ganz herzlich bei allen bedanken, die durch
aufmerksames Lesen Fehler gefunden und mir diese mitgeteilt haben.

\vspace{1cm}


\hfill{}Andreas Völklein

\label{END}
\end{document}
